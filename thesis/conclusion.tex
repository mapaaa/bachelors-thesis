\chapter{Concluzii}
Această lucrare prezintă o metodă computațională aplicată cu succes în reconstrucția cuvintelor 
cognate în seturi de date incomplete și în producerea etimonului latinesc în cazul limbii române.
Pentru acestea se folosesc cinci limbi romanice moderne: spaniola, italiana, franceza, portugheza
și româna. Algoritmul are la bază idei preluate din lingvistica comparativă încercând să automatizeze
procesul de detecție al șabloanelor de modificare ortografică în evoluția cuvintelor. Aceasta nu ia
în considerare alți factori precum cei sociali, economici, tehnologici etc, factori ce ar fi dificil
de modelat matematic.

Astfel, producțiile obținute vin în ajutorul experților în domeniul lingvisticii istorice pentru
studiul evoluției și reconstrucției artificiale de limbaj.

\section{Îmbunătățiri}
În primul rând, o idee ce merită a fi cercetată este adăugarea dimensiunii fonetice în analiza 
cuvintelor cognate din limbile romanice moderne și latina. Așadar, sistemul va fi adaptat pentru a 
combina deducțiile bazate pe schimbările ortografice cu cele bazate pe transcrierea fonetică.

În al doilea rând, am putea extinde algoritmul pentru a lua în considerare toate limbile romanice
moderne precum \textit{catalana}, \textit{siciliana} și dialectele lor. Cel mai probabil, ar trebui
grupate în funcție de similaritate pentru a obține cele mai bune rezultate, cum am procedat și în 
experimentele rulate (italiană, română și latină). Spre exemplu, catalana este mai apropiată de 
spaniolă deci schimbările produse într-una dintre ele au șanse mai mari sa fie similare decât
cele produse în română.

În al treilea rând, merită cercetată adaptabilitatea metodei prezentate în adăugarea dimensiunii 
semantice a cuvintelor. Putem oare adaugă sensul cuvintelor ca un alt set de atribute în învățarea
sistemelor CRF?


