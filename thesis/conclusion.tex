\chapter{Concluzii}
Aceasta lucrare prezinta o metoda computationala aplicata cu succes in reconstructia cuvintelor 
cognate in seturi de date incomplete si in producerea etimonului latinesc in cazul limbii romane.
Pentru acestea se folosesc cinci limbi romanice moderne: spanionla, italiana, franceza, portugheza
si romana. Algoritmul are la baza idei preluate din lingvistica comparativa incercand sa automatizeze
procesul de detectie al sabloanelor de modificare ortografica in evolutia cuvintelor. Aceasta nu ia
in considerare alti factori precum cei sociali, economici, tehnologici etc, factori ce ar fi dificili
de modelat matematic.

Astfel, productiile obtinute vin in ajutorul expertilor in domeniul lingvisticii istorice pentru
studiul evolutiei si recostructiei artificiale de limbaj.

\section{Imbunatatiri}
In primul rand, o idee ce merita a fi cercetata este adaugarea dimensiunii fonetice in analiza 
cuvintelor cognate din limbile romanice moderne si latina. Asadar, sistemul va fi adaptat pentru a 
combina deductiile bazate pe schimbarile ortografice cu cele bazate pe transcrierea fonetica.

In al doilea rand, am putae extinde algoritmul pentru a lua in considerare toate limbile romanice
moderne precun \textit{catalana}, \textit{siciliana} si dialectele lor. Cel mai probabil, ar trebui
grupate in functie de similaritate pentru a obtine cele mai bune rezultate, cum am procedat si in 
experimentele rulate (italiana, romana si latina). Spre exemplu, catalana este mai apropiata de 
spaniola deci schimbarile produse intr-una dintre ele au sanse mai mari sa fie similare decat
cele produse in romana.

In al treilea rand, merita cercetata adaptabilitatea metodei prezentate in adaugarea dimensiunii 
semantice a cuvintelor. Putem oare adauga sensul cuvintelor ca un alt set de atribute in invatarea
sistemelor CRF?


