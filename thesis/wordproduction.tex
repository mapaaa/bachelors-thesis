\chapter{Reconstrucție de cuvinte latinești}
\label{chap:two}
Pornim de la formele cuvintelor din limbile romanice moderne. Fiind date mai multe perechi de cuvinte
cognate vrem să deducem forma latinescului strămoș comun. Aplicăm o metodă similară cu cea folosită
în \cite{theone}. Ca și în \cite{theone}, ne bazăm pe faptul că modificările ortografice sunt strâns
legate de evoluția cuvintelor. Deci încercam să reconstruim proto-cuvinte din forma ortografică a
cuvintelor moderne.

În final, dorim să obținem o listă cu $n$ cele mai bune predicții pe care mai apoi sa le prelucrăm
într-o manieră atât automată cât și manuală pentru a obține cele mai bune rezultate.

\section{Prezentare generala}
Dat fiind mai multe seturi de date de cuvinte cognate din limbi romanice moderne, metoda de 
reconstrucție va încerca să aproximeze forma latinescului de proveniență. Vom folosi: \textit{Română},
\textit{Italiană}, \textit{Franceză}, \textit{Spaniolă}, \textit{Portugheză}, iar seturile de date
vor avea forma $(\textit{cuvânt modern}, \textit{cuvânt latinesc})$. Din acestea, modelul va învăța 
pe baza câmpurilor condiționate aleatoare (\textit{conditional random fields} sau prescurtat CRF) 
diverse schimbări de ortografie suferite de cuvintele latinești pentru a le forma pe cele moderne. 
Apoi, vom aplica o tehnica de agregare a acestor rezultate pentru a combina informație din toate 
limbile. În final, vom folosi aceste schimbări pentru a oferi variante de cuvinte ce 
completează anumite seturi de date.

Pașii algoritmului pentru o anumită limbă romanică modernă sunt:

\begin{enumerate}
  \item Pentru fiecare pereche $(\textit{cuvânt modern}, \textit{cuvânt latinesc})$, vom alinia
    cele două cuvinte pentru a înțelege ce semne ortografice s-au păstrat, schimbat sau elidat.
  \item Pregătim antrenarea sistemului CRF: extragem caracteristici din alinierile fiecărei perechi.
  \item Rulăm sistemul CRF și obținem liste de $n$ cele mai bune producții sortate în funcție de
    probabilitatea lor.
\end{enumerate}

\section{Aliniere}
Avem perechi de tipul $(\textit{cuvânt modern}, \textit{cuvânt latinesc})$ pe care vrem să le aliniem.
Nu orice aliniere ne oferă informație validă. Avem nevoie de așa numitele alinieri optime, în care
numărul de diferențe dintre cele două cuvinte este minim. Vom aplica algoritmul de aliniere 
Needleman-Wunsch\cite{needle} din bioinformatică, folosit cu succes și în probleme de procesare al
limbajului natural. 

\subsection{Needleman-Wunsch}
Algoritmul de aliniere Needleman-Wunsch provine din bioinformatică, mai exact din alinierea secvențelor
de proteine sau nucleotide. Determinarea alinierilor se face printr-o tehnică
de programare dinamică. Așadar, problema inițială va fi împărțită în subprobleme fie deja calculate,
fie mai ușor de calculat. De fiecare dată când vom spune alinieri ne vom referi doar la alinieri de tip
Needleman-Wunsch.

Avem doua șiruri $a=a_1a_2...a_n$ și $b=b_1b_2...b_m$ de caractere de lungime $n$ respectiv $m$. 
Vrem sa aliniem șirul $b$ pentru a se potrivi cu șirul $a$.  

Există 3 tipuri de operații într-o aliniere la o anumită poziție $i$:
\begin{enumerate}
  \item \textbf{Potrivire}, caracterele aliniate se potrivesc: $a_i=b_i$
  \item \textbf{Nepotrivire}, caracterele aliniate nu se potrivesc: $a_i \neq b_i$
  \item \textbf{Spațiu}, caracterul din primul șir nu se aliniază cu niciun caracter din al doilea șir
    sau invers
\end{enumerate}

Se observă că în cazul operației de tipul 3, caracterele din dreapta poziției curente $i$ se vor
deplasa cu o poziție.

Fiecare dintre operațiile de mai sus are un anumit cost. În problema noastră de aliniere a unui
cuvânt latinesc cu un cuvânt modern, vom considera costul unei operații de \textbf{Potrivire} ca fiind $0$,
iar costul unei operații de \textbf{Nepotrivire} sau \textbf{Spațiu} ca fiind 1. Astfel, alinierea
optimă va fi cea cu costul minim. În procesul de potrivire nu  vom lua în considerare diacriticele.
Literele ce conțin astfel de semne vor fi considerate la fel cu literele de bază (spre exemplu,
caracterul \textit{\`{e}} poate fi potrivit cu \textit{e}).

Problema se rezolvă folosind tehnica programării dinamice. Definim
două matrici bidimensionale cu $n+1$ și $m+1$ coloane numerotate de $0$ la $n$ respectiv de la $0$
la $m$:

\begin{gather*}
  D_{i,j} = \text{costul minim pentru a alinia prefixul } a_1a_2...a_i \text{ din șirul a} \\
  \text{cu prefixul } b_1b_2...b_j \text{ din șirul b} \\
  P_{i,j} = \text{ultima operație efectuată } \textbf{Potrivire }, \textbf{Nepotrivire } \text{sau} 
  \textbf{ Spațiu}
\end{gather*}

Considerăm faptul că prefixul vid va fi reprezentat de poziția fie cu linia $0$ (în cazul în care
prefixul vid provine din șirul $a$), fie cu coloana $0$ (în cazul în care prefixul vid provine
din șirul $b$).

Relația de recurență este:
\begin{gather*}
  D_{0,j} = j, \forall j=0,...,m \\
  D_{i,0} = i, \forall i=0,...,n \\
  D_{i,j} = \min \begin{dcases}
      D_{i-1,j-1}       &, a_i=b_j \textbf{ Potrivire} \\
      D_{i-1,j-1}+1     &, a_i \neq b_j \textbf{ Nepotrivire} \\
      D_{i-1,j}         &, \textbf{ Spațiu} \\
      D_{i,j-1}         &, \textbf{ Spațiu} \\
    \end{dcases}
\end{gather*}

Scorul alinierii se va afla pe poziția $D_{n,m}$. Pentru a reconstitui alinierea, la fiecare pas din 
recurență trebuie să reținem în matricea $P$ ultima operație efectuată. Acum, pornim cu doi
indici $i = n$ și $j = m$. Dacă pentru a rezolva subproblema determinată de prefixele $a_1a_2...a_i$
și $b_1b_2...b_j$ am folosit ca și ultimă operație:

\begin{itemize}
  \item \textbf{Potrivire}: aliniem caracterul de pe poziția $i$ din $a$ cu cel de pe poziția $j$ 
    din $b$; decrementăm indicii $i$ și $j$  
  \item \textbf{Spațiu}: deducem dacă spațiul este în șirul $a$ sau $b$ și refacem indicii corespunzător
    (dacă spațiul provine din șirul $a$ atunci decrementăm indicele $j$, iar dacă spațiul provine
    din șirul $b$ decrementăm indicele $i$
  \item \textbf{Nepotrivire}: aliniem caracterele diferite de pe pozițiile $i$ din $a$ și $j$ din $b$;
    decrementăm indicii $i$ și $j$
\end{itemize}

    
\section{Câmpuri condiționate aleatoare}
În continuare vrem să învățăm care sunt schimbările ortografice produse în evoluția unui cuvânt 
modern știind etimonul său. În secțiunea precedentă am aliniat cuvintele pentru a determina ce  
modificări au suferit caracterele. Folosim un algoritm de învățare automată pentru a studia
șabloane de schimbări ortografice dintre fiecare limbă modernă și limba latină.

Propunem o metodă bazată de câmpuri condiționate aleatoare, întrucât acestea au dat rezultate satisfăcătoare în generarea transliterațiilor \cite{ganesh} și în producția de cuvinte cognate \cite{crfciobanu}.

Vom explica mai întâi câteva noțiuni despre câmpurile condiționate aleatoare iar apoi le vom aplica 
pe problema reconstrucției de cuvinte.

\subsection{Generalități}
Câmpurilor condiționate aleatoare (\textit{conditional random fields} sau prescurtat CRF) sunt o
metodă de modelare statistică pentru a face predicții structurate.

Lafferty, McCallum si Pereira\cite{crf} sunt primii care explică această structură. Ei introduc un 
nou cadru pentru construirea modelelor probabilistice pentru segmentarea și etichetarea datelor
secvențiale. Până în acel moment, astfel de probleme erau rezolvate prin modele Markov ascunse și 
gramatici stocastice.

Conform \cite{crf}, fie $X$ o variabilă aleatoare peste secvențe de date ce trebuie etichetate și 
$Y$ o variabilă aleatoare peste etichetele corespunzătoare. Construim un model de probabilități
condiționate $P(Y|X)$ din perechile de secvențe de date și etichete.

\begin{definition}
Fie $G=(V, E)$ un graf astfel încât $Y=(Y_i)_{i \in V}$ (nodurile grafului reprezintă
indecșii etichetelor $Y$). $(X, Y)$ se numește câmp condiționat aleator dacă variabila aleatoare
$Y_v$ condiționată de $X$ respectă proprietatea Markov raportat la graful G: 
  \begin{gather*}
    P(Y_v | X, Y_w, w \neq v) = P(Y_v | X, Y_w, w \sim v)
  \end{gather*}
unde $w \sim v$ înseamnă că există muchia $(w, v)$ în $E$.
\end{definition}

Observăm că CRF-ul este global condiționat de variabila $X$. Graful $G$ este nedirecționat, construit
diferit în funcția de atributele secvențelor de etichetat dar de cele mai multe ori are forma unui
lanț sau arbore.

Pentru atribute, vom defini o funcție pentru a reprezenta caracteristici ale secvențelor de date. 
Valoarea acestor funcții se calculează în funcție de problema pe care dorim să o rezolvăm. Le vom nota 
cu $f_k$, $g_k$ etc. Spre exemplu $g_k(v, y|S, x)$ va fi adevărat dacă cuvântul X începe cu o majusculă, 
iar eticheta $Y$ este "substantiv propiu".\cite{crf} $y|S$ reprezintă  setul de componente ale lui $Y$ 
asociate cu nodurile subgrafului $G$. Funcțiile de atribute trebuie calculate înaintea aplicării 
sistemului CRF. 

Atribuim fiecărui atribut o anumită pondere $\lambda_i$ (pentru atributele în raport cu muchile grafului $G$)
și $\mu_i$ (pentru atributele reportate la nodurile grafului $G$). Acestea sunt valorile pe care algoritmul
trebuie să le învețe pentru a face mai apoi predicțiile.

Exista doua structuri de grafuri pe care se pretează să aplicăm sistemul CRF: cele lanț și cele
arborescente.

Formula probabilităților condiționate pentru structura de arbore este:
\begin{gather*}
  P(Y|X) \propto \exp(\smashoperator{\sum_{e \in E, k}} \lambda_k f_k(e, y|e, x) + \smashoperator{\sum_{v \in V, k}} \mu_k g_k(v, y|v, x))
\end{gather*}

Pentru a determina formula probabilităților condiționate pentru structura de lanț vom nota $start = Y_0$ și $stop = Y_{n+1}$
Probabilitățile condiționate le vom calcula într-o matrice $M$ astfel:
\begin{gather*}
  M_i(y', y|x) = exp(\Lambda_i(y', y|x)) \\
  \Lambda_i(y', y|x) = \smashoperator{\sum_k} \lambda_k f_k(e_i, Y|e_i = (y', y), x) +
                       \smashoperator{\sum_k} \mu_k g_k(v_i, Y|v_i = y, x).
\end{gather*}
unde, $e_i$ este muchia cu etichetele $(Y_{i-1}, Y_i)$ și $v_i$ este nodul corespondent lui $Y_i$. 

Pentru a estima parametrii $\lambda_i$ și $\mu_i$ se folosesc algoritmi iterativi pentru a maximiza
rezultatele pe un set de date de antrenare. Algoritmii sunt prezentați în \cite{crf}.

Acum pute scrie formula pentru probabilitățile condiționate pentru un graf de tip lanț:
\begin{gather*}
  P(Y|X) \propto \frac{\smashoperator{\prod_{i=1}^{n+1}} M_i(y_{i-1}, y_i | x)}{(\smashoperator{\prod_{i=1}^{n+1}} M_i(x))_{start,stop}}
\end{gather*}

Precum și în alte modele statistice ce folosesc probabilități condiționate (spre exemplu: Naive Bayes),
predicția se face prin:

\begin{gather*}
  \hat{Y} = argmax_Y(P(Y|X))
\end{gather*}

Putem lua chiar și primele $n$ cele mai bune probabilități pentru face o analiză în profunzime.

\subsection{Aplicate pe problema reconstrucției cuvintelor}
\label{subs:one}
Ne întoarcem la problema de predicție a cuvintelor latinești pornind de la cuvinte moderne folosind 
sisteme CRF. Secvențele de date vor fi cuvintele moderne. Atributele vor fi $n$-grame din cuvintele
de intrare, extrase din ferestre de dimensiune $w$. 

Etichetele se calculează folosind algoritmul de aliniere, utilizând caracterele ce se potrivesc în 
cuvântul modern și cuvântul latinesc. Dar, pentru că cele două cuvinte nu sunt complet identice,
în cazul unei inserări vom pune caracterul adăugat la eticheta precedentă (nu putem asocia aceasta
literă cu niciuna din cuvântul de intrare). În cazul unui spațiu, apărut în cuvântul latinesc se 
produce practic un fenomen de elidare. Asociem caracterului respectiv din cuvântul sursă, o etichetă
nouă (spre exemplu $-$) pentru a semnifica dispariția acestuia.

Sistemul are nevoie de margini la capetele cuvintelor, în cazul în care avem inserții produse la 
începutul sau sfârșitul cuvântului, adică o eticheta "precedentă" cu care să asociem aceste litere noi.
Așadar, fiecare cuvânt va fi extins prin adăugarea a două caractere \textbf{B} și \textbf{E} la 
începutul și sfârșitul acestuia.\cite{theone} Orice literă nouă adăugata la început sau la sfârșit, va
fi asociată cu aceste 2 litere speciale.


Folosim 5 astfel de sisteme CRF pentru fiecare limba modernă \textit{română}, \textit{italiană}, 
\textit{franceză}, \textit{spaniolă}, \textit{portugheză} pusa în raport cu \textit{limba latină}.
Fiecare sistem va calcula liste de cele mai bune $n$ cuvinte latinești sortate în funcție de 
probabilitate. Pentru a profita de toate limbile moderne propunem o metodă de combinare a rezultatelor
sistemelor CRF bazată pe agregări folosind distanța rank.
