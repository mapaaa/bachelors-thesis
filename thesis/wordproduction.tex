\chapter{Reconstrucție de cuvinte latinesti}
Pornim de la formele cuvintelor din limbile romanice moderne. Fiind date mai multe perechi de cuvinte
cognate vrem sa deducem forma latinescului stramos comun. Aplicam o metoda similara cu cea folosita
in \cite{theone}. Ca si in \cite{theone}, ne bazam pe faptul ca modificarile ortografice sunt strans
legate de evolutia cuvintelor. Deci incercam sa reconstruim proto-cuvinte din forma ortografica a
cuvintelor moderne.

In final, dorim sa obtinem o lista cu $n$ cele mai bune predictii pe care mai apoi sa le prelucam
intr-o maniera atat automata cat si manuala pentru a obtine cele mai bune rezultate.

\section{Prezentare generala}
Dat fiind mai multe seturi de date de cuvinte cognate din limbi romanice moderne, metoda de 
reconstructie va incerca sa aproximeze forma latinescului de provenienta. Vom folosi: \textit{Romana},
\textit{Italiana}, \textit{Franceza}, \textit{Spaniola}, \textit{Portugheza}, iar seturile de date
vor avea forma $(\textit{cuvant modern}, \textit{cuvant latinesc})$. Din acestea, modelul va invata 
pe baza campurilor conditionate aleatoare (\textit{conditional random fields} sau prescurtat CRF) 
diverse schimbari de ortografie suferite de cuvintele latinesti pentru a le forma pe cele moderne. 
Apoi, vom aplica o tehnica de agregare a acestor rezultate pentru a combina informatie din toate 
limbile. In final, vom folosi mai apoi aceste schimbari pentru a oferi variante de cuvinte ce 
completeaza anumite seturi de date.

Pasii algoritmului pentru o anumita limba romanica moderna sunt:

\begin{enumerate}
  \item Pentru fiecare pereche $(\textit{cuvant modern}, \textit{cuvant latinesc})$, vom alinia
    cele doua cuvinte pentru a intelege ce semne ortografice s-au pastrat, schimbat sau elidat.
  \item Pregatim antrenarea sistemului CRF: extragem caracteristici din alinierile fiecarei perechi.
  \item Rulam sistemul CRF si obtinem liste de $n$ cele mai bune productii sortate in functie de
    probabilitatea lor.
\end{enumerate}

\section{Aliniere}
Avem perechi de tipul $(\textit{cuvant modern}, \textit{cuvant latinesc})$ pe care vrem sa le aliniem.
Nu orice aliniere ne ofera informatie valida. Avem nevoie de asa numitele alinieri optime, in care
numarul de diferente dintre cele doua cuvinte este minim. Vom aplica algoritmul de aliniere 
Needleman-Wunsch\cite{needle} din bioinfomatica, folosit cu succes si in probleme de procesare al
limbajului natural. 

\subsection{Needleman-Wunsch}
Algoritmul de aliniere Needleman-Wunsch provine din bioinfomatica, mai exact din alinierea secventelor
de proteine sau nucleotide. Determinarea alinierilor se face printr-o tehnica
de programare dinamica. Asadar, problema initiala va fi impartita in subprobleme fie deja calculate,
fie mai usor de calculat. De fiecare cand vom spune alinieri ne vom referi doar la alinieri de tip
Needleman-Wunsch.

Avem doua siruri $a=a_1a_2...a_n$ si $b=b_1b_2...b_m$ de caractere de lungime $n$ respectiv $m$. 
Vrem sa aliniem sirul $b$ pentru a se potrivi cu sirul $a$.  

Exista 3 tipuri de operatii intr-o aliniere la o anumita pozitie $i$:
\begin{enumerate}
  \item \textbf{Potrivire}, caracterele aliniate se potrivesc: $a_i=b_i$
  \item \textbf{Nepotrivire}, caracterele aliniate nu se potrivesc: $a_i \neq b_i$
  \item \textbf{Spatiu}, caracterul din primul sir nu se aliniaza cu niciun caracter din al doilea sir
    sau invers;
\end{enumerate}

Se observa ca in cazul operatiei de tipul 3, caracterele din dreapta pozitiei curente $i$ se vor
deplasa cu o pozitie.

Fiecare dintre operatiile de mai sus are un anumit cost. In problema noastra de aliniere a unui
cuvant latinesc cu un cuvant modern, vom considera costul unei operatii de \textbf{Potrvire} ca fiind $0$,
iar costul unei operatii de \textbf{Nepotrivire} sau \textbf{Spatiu} ca fiind 1. Atfel, alinierea
optima va fi cea cu costul minim. In procesul de potrivire nu  vom lua in considerare diacriticele.
Literele ce contin astfel de semne vor fi considerate la fel cu literele de baza (spre exemplu,
caracterul \textit{\`{e}} poate fi potrivit cu \textit{e}).

Problema se rezolva folosind tehnica programarii dinamice. Definim
doua matrici bidimensionale cu $n+1$ si $m+1$ coloane numerotate de $0$ la $n$ respectiv de la $0$
la $m$:

\begin{gather*}
  D_{i,j} = \text{costul minim pentru a alinia prefixul } a_1a_2...a_i \text{ din sirul a} \\
  \text{cu prefixul } b_1b_2...b_j \text{ din sirul b} \\
  P_{i,j} = \text{ultima operatie efectuata } \textbf{Potrivire }, \textbf{Nepotrivire } \text{sau} 
  \textbf{ Spatiu}
\end{gather*}

Consideram faptul ca prefixul vid va fi reprezentat de pozitia fie cu linia $0$ (in cazul in care
prefixul vid provine din sirul $a$), fie cu coloana $0$ (in cazul in care prefixul vid provine
din sirul $b$).

Relatia de recurenta este:
\begin{gather*}
  D_{0,j} = j, \forall j=0,...,m \\
  D_{i,0} = i, \forall i=0,...,n \\
  D_{i,j} = \min \begin{dcases}
      D_{i-1,j-1}       &, a_i=b_j \textbf{ Potrivire} \\
      D_{i-1,j-1}+1     &, a_i \neq b_j \textbf{ Nepotrivire} \\
      D_{i-1,j}         &, \textbf{ Spatiu} \\
      D_{i,j-1}         &, \textbf{ Spatiu} \\
    \end{dcases}
\end{gather*}

Scorul alinierii se va afla pe pozitia $D_{n,m}$. Pentru a reconstitui alinierea, la fiecare pas din 
recurenta trebuie sa retinem in matricea $P$ matrice ultima operatie efectuata. Acum, pornim cu doi
indici $i = n$ si $j = m$. Daca pentru a rezolva subproblema determinata de prefixele $a_1a_2...a_i$
si $b_1b_2...b_j$ am folosit ca si ultima operatie:

\begin{itemize}
  \item \textbf{Potrivire}: aliniem caracterul de pe pozitia $i$ din $a$ cu cel de pe pozitia $j$ 
    din $b$; decrementam indicii $i$ si $j$  
  \item \textbf{Spatiu}: deducem daca spatiul este in sirul $a$ sau $b$ si refacem indicii corespunzator
    (daca spatiul provine din sirul $a$ atunci decrementam indicele $j$, iar daca spatiul provine
    din sirul $b$ decrementam indicele $i$
  \item \textbf{Nepotrivire}: aliniem caracterele diferite de pe pozitiile $i$ din $a$ si $j$ din $b$;
    decrementam indicii $i$ si $j$
\end{itemize}

    
\section{Campuri conditionate aleatoare}
\subsection{Generalitati}
\subsection{Aplicate pe problema reconstructiei cuvintelor}

