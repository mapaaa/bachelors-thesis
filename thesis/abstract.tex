\begin{abstract}
În această lucrare ne propunem să introducem câteva noțiuni despre lingvistica computațională și 
producerea de cuvinte. Pornind de la tehnici deja existente, examinăm o metodă prin care să putem 
produce cuvinte cognate ce lipsesc din anumite limbi romanice, cu precadere pe limba română. Algoritmul
prezentat se bazează pe modelări statistice ale unor alinieri de perechi de cuvinte și etimonul lor.
Limbile folosite sunt română, italiană, franceză, spaniolă, portugheză și latină. Informația redată
de aceste limbi asupra modificărilor structurii cuvintelor este combinată folosind agregarări pe baza
dinstanței rank.
\end{abstract}

