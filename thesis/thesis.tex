% Template version used: v1.4
%
% Largely adapted from Adrian Nievergelt's template for the ADPS
% (lecture notes) project.
% Adapted for a bachelor's thesis


%% We use the memoir class because it offers a many easy to use features.
\documentclass[11pt,a4paper,titlepage]{memoir}

%% Packages
%% ========

\usepackage{algorithm}
\usepackage{algorithmic}
%% LaTeX Font encoding -- DO NOT CHANGE
\usepackage[OT1]{fontenc}

%% Babel provides support for languages.  'english' uses British
%% English hyphenation and text snippets like "Figure" and
%% "Theorem". Use the option 'ngerman' if your document is in German.
%% Use 'american' for American English.  Note that if you change this,
%% the next LaTeX run may show spurious errors.  Simply run it again.
%% If they persist, remove the .aux file and try again.
\usepackage[romanian]{babel}

%% Input encoding 'utf8'. In some cases you might need 'utf8x' for
%% extra symbols. Not all editors, especially on Windows, are UTF-8
%% capable, so you may want to use 'latin1' instead.
\usepackage[utf8]{inputenc}

%% This changes default fonts for both text and math mode to use Herman Zapfs
%% excellent Palatino font.  Do not change this.
\usepackage[sc]{mathpazo}

%% The AMS-LaTeX extensions for mathematical typesetting.  Do not
%% remove.
\usepackage{amsmath,amssymb,amsfonts,mathrsfs}

%% NTheorem is a reimplementation of the AMS Theorem package. This
%% will allow us to typeset theorems like examples, proofs and
%% similar.  Do not remove.
%% NOTE: Must be loaded AFTER amsmath, or the \qed placement will
%% break
\usepackage[amsmath,thmmarks]{ntheorem}

%% LaTeX' own graphics handling
\usepackage{graphicx}

%% We unfortunately need this for the Rules chapter.  Remove it
%% afterwards; or at least NEVER use its underlining features.
\usepackage{soul}

%% This allows you to add .pdf files. It is used to add the
%% declaration of originality.
\usepackage{pdfpages}

%% Some more packages that you may want to use.  Have a look at the
%% file, and consult the package docs for each.
%% See the TeXed file for more explanations

%% [OPT] Multi-rowed cells in tabulars
%\usepackage{multirow}

%% [REC] Intelligent cross reference package. This allows for nice
%% combined references that include the reference and a hint to where
%% to look for it.
\usepackage{varioref}

%% [OPT] Easily changeable quotes with \enquote{Text}
%\usepackage[german=swiss]{csquotes}

%% [REC] Format dates and time depending on locale
\usepackage{datetime}

%% [OPT] Provides a \cancel{} command to stroke through mathematics.
%\usepackage{cancel}

%% [NEED] This allows for additional typesetting tools in mathmode.
%% See its excellent documentation.
\usepackage{mathtools}

%% [ADV] Conditional commands
%\usepackage{ifthen}

%% [OPT] Manual large braces or other delimiters.
%\usepackage{bigdelim, bigstrut}

%% [REC] Alternate vector arrows. Use the command \vv{} to get scaled
%% vector arrows.
\usepackage[h]{esvect}

%% [NEED] Some extensions to tabulars and array environments.
\usepackage{array}

%% [OPT] Postscript support via pstricks graphics package. Very
%% diverse applications.
%\usepackage{pstricks,pst-all}

%% [?] This seems to allow us to define some additional counters.
%\usepackage{etex}

%% [ADV] XY-Pic to typeset some matrix-style graphics
%\usepackage[all]{xy}

%% [OPT] This is needed to generate an index at the end of the
%% document.
%\usepackage{makeidx}

%% [OPT] Fancy package for source code listings.  The template text
%% needs it for some LaTeX snippets; remove/adapt the \lstset when you
%% remove the template content.
\usepackage{listings}
\lstset{language=TeX,basicstyle={\normalfont\ttfamily}}

%% [REC] Fancy character protrusion.  Must be loaded after all fonts.
\usepackage[activate]{pdfcprot}

%% [REC] Nicer tables.  Read the excellent documentation.
\usepackage{booktabs}

%% Up greek letters
\usepackage{upgreek}


%% Our layout configuration.  DO NOT CHANGE.
%% Memoir layout setup

%% NOTE: You are strongly advised not to change any of them unless you
%% know what you are doing.  These settings strongly interact in the
%% final look of the document.

\usepackage{UNIBUClogo}

% Turn extra space before chapter headings off.
\setlength{\beforechapskip}{0pt}

\nonzeroparskip
\parindent=0pt
\defaultlists

% Chapter style redefinition
\makeatletter

\pagestyle{ruled}
\makeevenfoot{ruled}{}{}{\thepage}
\makeoddfoot{ruled}{}{}{\thepage}
\copypagestyle{chapter}{ruled}

\makeoddhead{chapter}{}{}{}
\makeevenhead{chapter}{}{}{}
\makeheadrule{chapter}{\textwidth}{0pt}
\copypagestyle{abstract}{empty}

\makechapterstyle{bianchimod}{%
  \chapterstyle{default}
  \renewcommand*{\chapnamefont}{\normalfont\Large}
  \renewcommand*{\chapnumfont}{\normalfont\Large}
  \renewcommand*{\printchaptername}{%
    \chapnamefont\centering\@chapapp}
  \renewcommand*{\printchapternum}{\chapnumfont {\thechapter}}
  \renewcommand*{\chaptitlefont}{\normalfont\huge}
  \renewcommand*{\printchaptertitle}[1]{%
    \hrule\vskip\onelineskip \centering \chaptitlefont\textbf{\vphantom{gyM}##1}\par}
  \renewcommand*{\afterchaptertitle}{\vskip\onelineskip \hrule\vskip
    \afterchapskip}
  \renewcommand*{\printchapternonum}{%
    \vphantom{\chapnumfont {9}}\afterchapternum}}

% Use the newly defined style
\chapterstyle{bianchimod}

\setsecheadstyle{\Large\bfseries}
\setsubsecheadstyle{\large\bfseries}
\setsubsubsecheadstyle{\bfseries}
\setparaheadstyle{\normalsize\bfseries}
\setsubparaheadstyle{\normalsize\itshape}
\setsubparaindent{0pt}

% Set captions to a more separated style for clearness
\captionnamefont{\bfseries\footnotesize}
\captiontitlefont{\footnotesize}
\setlength{\intextsep}{16pt}
\setlength{\belowcaptionskip}{1pt}

% Set section and TOC numbering depth to subsection
\setsecnumdepth{subsection}
\settocdepth{subsection}

%% Titlepage adjustments
\pretitle{\vspace{0pt plus 0.7fill}\begin{center}\HUGE\normalfont\bfseries}
\posttitle{\end{center}\par}
\preauthor{\par\begin{center}\let\and\\\Large\normalfont}
\postauthor{\end{center}}
\predate{\par\begin{center}\Large\normalfont}
\postdate{\end{center}}

\def\@advisors{}
\newcommand{\advisors}[1]{\def\@advisors{#1}}
\def\@department{}
\newcommand{\department}[1]{\def\@department{#1}}
\def\@thesistype{}
\newcommand{\thesistype}[1]{\def\@thesistype{#1}}

\renewcommand{\maketitlehooka}{\noindent\UNIBUClogo[2in]}

\renewcommand{\maketitlehookb}{\vspace{1in}%
  \par\begin{center}\Large\normalfont\@thesistype\end{center}}

\renewcommand{\maketitlehookd}{%
  \vfill\par
  \begin{flushright}
    \normalfont
    \@advisors\par
    \@department, UNIBUC 
  \end{flushright}
}

\checkandfixthelayout

\setlength{\droptitle}{-48pt}

\makeatother

% This defines how theorems should look. Best leave as is.
\theoremstyle{plain}
\setlength\theorempostskipamount{0pt}

%%% Local Variables:
%%% mode: latex
%%% TeX-master: "thesis"
%%% End:



%% Theorem environments.  You will have to adapt this for a German
%% thesis.
%% Theorem-like environments

%% This can be changed according to language. You can comment out the ones you
%% don't need.

\numberwithin{equation}{chapter}

%% German theorems
%\newtheorem{satz}{Satz}[chapter]
%\newtheorem{beispiel}[satz]{Beispiel}
%\newtheorem{bemerkung}[satz]{Bemerkung}
%\newtheorem{korrolar}[satz]{Korrolar}
%\newtheorem{definition}[satz]{Definition}
%\newtheorem{lemma}[satz]{Lemma}
%\newtheorem{proposition}[satz]{Proposition}

%% English variants
\newtheorem{theorem}{Teoremă}[chapter]
\newtheorem{example}[theorem]{Exemplu}
\newtheorem{remark}[theorem]{Propoziție}
\newtheorem{corollary}[theorem]{Corolar}
\newtheorem{definition}[theorem]{Definiție}
\newtheorem{lemma}[theorem]{Lemă}
\newtheorem{proposition}[theorem]{Propoziție}
\newtheorem{problem}[theorem]{Problemă}

%% Proof environment with a small square as a "qed" symbol
\theoremstyle{nonumberplain}
\theorembodyfont{\normalfont}
\theoremsymbol{\ensuremath{\square}}
\newtheorem{proof}{Proof}
%\newtheorem{beweis}{Beweis}



%% Helpful macros.
%% Custom commands
%% ===============

%% Special characters for number sets, e.g. real or complex numbers.
\newcommand{\C}{\mathbb{C}}
\newcommand{\K}{\mathbb{K}}
\newcommand{\N}{\mathbb{N}}
\newcommand{\Q}{\mathbb{Q}}
\newcommand{\R}{\mathbb{R}}
\newcommand{\Z}{\mathbb{Z}}
\newcommand{\X}{\mathbb{X}}

%% Fixed/scaling delimiter examples (see mathtools documentation)
\DeclarePairedDelimiter\abs{\lvert}{\rvert}
\DeclarePairedDelimiter\norm{\lVert}{\rVert}

%% Use the alternative epsilon per default and define the old one as \oldepsilon
\let\oldepsilon\epsilon
\renewcommand{\epsilon}{\ensuremath\varepsilon}

%% Also set the alternate phi as default.
\let\oldphi\phi
\renewcommand{\phi}{\ensuremath{\varphi}}



%% Make document internal hyperlinks wherever possible. (TOC, references)
%% This MUST be loaded after varioref, which is loaded in 'extrapackages'
%% above.  We just load it last to be safe.
\usepackage[linkcolor=black,colorlinks=true,citecolor=black,filecolor=black]{hyperref}


%% Document information
%% ====================

\title{Aspecte computaționale în producerea de cuvinte}
\author{Maria-Smaranda Pandele}
\thesistype{Lucrare de Licență}
\advisors{Coordonator: Prof.\ Dr.\ Liviu Dinu}
\department{Facultatea de Matematică și Informatică}
\date{22 Iunie, 2019}
\begin{document}

\frontmatter

%% Title page is autogenerated from document information above.  DO
%% NOT CHANGE.
\begin{titlingpage}
  \calccentering{\unitlength}
  \begin{adjustwidth*}{\unitlength-24pt}{-\unitlength-24pt}
    \maketitle
  \end{adjustwidth*}
\end{titlingpage}

%% The abstract of your thesis.  Edit the file as needed.
\begin{abstract}
  This example thesis briefly shows the main features of our thesis
  style, and how to use it for your purposes.
\end{abstract}



%% TOC with the proper setup, do not change.
\cleartorecto
\tableofcontents
\mainmatter

%% Your real content!
\chapter{Introducere}
Limbajul unei țări se află în continuă schimbare datorită mai multor cauze precum cauze sociale, 
economice, migrația populației, progresele în știință, tehnologie și medicină. De-a lungul timpului,
modificările s-au petrecut inevitabil și nu au fost neapărat reglementate de experți în domeniu.
Astfel a apărut o nouă ramură a lingvisticii ce studiază sistematic aceste transformări încercând
să găsească șabloane, reguli și să pună o ordine asupra schimbărilor lexicale, fonetice, semantice
și sintactice. Câteva exemple clasice ar fi determinarea etimologiei unui cuvânt (românescul
\textit{genunchi} provine din latinescul \textit{genuclum}) sau determinarea similarității
limbajelor.

De cele mai multe ori, popoare ce vorbeau aceeași limbă s-au despărțit și au evoluat diferit, apărând
limbaje derivate. Ne putem da seama de gradul de rudenie a acestora prin identificarea
formelor \textbf{cognate} (grupuri de cuvinte ce au derivat din același etimon). În alte cazuri,
limbile au împrumutat cuvinte între ele fie luându-le ca atare (cuvântul japonez \textit{sushi}),
fie modificând forma lor (romanescul \textit{cafea} din turcescul \textit{kahve}). Astfel, se pot 
descoperi chiar relații de natură istorică între limbi și popoarele ce le folosesc. Spre exemplu,
există peste 200 de cuvinte ce se regăsesc în toate limbile romanice moderne mai puțin limba
română. Este greu de crezut ca o parte din aceste cuvinte nu au existat în lexicul limbii române
la un moment dat. Fischer\cite{fischer} a identificat câteva cauze ce au condus la dispariția acestor cuvinte
și înlocuirea acestora prin cuvinte cu alte etimologi: cauze externe, social-economice, schimbarea
ocupațiilor romanilor, întreruperea României cu lumea Occidentala, dezvoltarea limbii române departe
de nucleul romanic și asa mai departe.

\section{Reconstrucția limbajelor folosind metode comparative}
Lingvisticii istorici se ocupă cu cercetarea schimbărilor limbilor de-a lungul timpului. 
O preocupare majora a acestora o reprezintă producerea de cuvinte înrudite. Ei folosesc
metode comparative prin care analizează modificările limbajelor efectuând comparații între limbaje
înrudite pentru a deduce, printr-o inginerie inversă, proprietăți ale limbii strămoș comune.\cite{weissbook}
Practic, se încearcă determinarea grupurilor de cuvinte cognate și găsirea unor reguli prin care
au fost obținute din limba strămoș. Acestea nu sunt neapărat atestate, cele mai multe dintre ele 
denumindu-se \textbf{proto-limbaje}. În consecința, rezultatele obținute sunt foarte greu de demonstrat
pentru ca nu exista dovezi arheologice concrete.

Asemenea metode au dat rezultate satisfăcătoare, reușind să determine structura familiilor de limbi
europene. Mai mult, reconstrucția proto-limbajul \textbf{Indo-European} considerat cel mai vechi
limbaj cunoscut, a fost posibilă prin punerea în corelație cu limbajele derivate din acesta 
(proto-limbajul German, proto-limbajul Indo-Iranian).\cite{protostuff} În plus, metodele comparative
au avut avut succes și în analiza altor familii de limbaje de pe alte continente.

\section{Metode computaționale}
În principal, tehnicile comparative au mai multi pași efectuați manual de lingviști și presupun 
multe ore de muncă. În perioada în care calculatoarele nu erau încă inventate, căutarea cuvintelor 
în cărți, dicționare și prelucrarea acestora era o muncă nu numai obositoare și repetitivă dar și 
dificilă, fiind nevoie de atenție continuă. Însă, odată cu apariția metodelor computaționale, se pune accentul
pe determinarea automată cuvintelor înrudite precum în \cite{kondrak} sau \cite{list}. Aceste
soluții sunt departe de a înlocui un expert în domeniul lingvisticii și își propun mai mult să vină
în ajutorul acestuia pentru a facilita dezvoltarea în profunzime a domeniului.

Metodele computaționale doar automatizează procesul păstrând ideea pornirii de la formele moderne
ale cuvintelor pentru a le reconstrui pe cele vechi din care provin. Există numeroase date întreținute
activ de către specialiști ce determină conexiuni între cuvinte din mai multe limbi moderne. Spre
exemplu, corpusul Europarl pentru limbile oficiale vorbite în Uniunea Europeana sau WordNet (Fellbaum, 1998)
pentru limba engleza. Asemenea resurse vaste sunt esențiale în aplicarea unor tehnici comparative 
automate. Pe baza lor se pot determina și aplica reguli de producție a cuvintelor într-un mod formal,
fără prea mari intervenții din partea lingvisticilor pentru asa zisele \textit{excepții}.

Este dificil de prezis cum anume a fost modificat un cuvânt pentru a ajunge în forma lui actuală.
Deși se presupune că ar exista șabloane și reguli în evoluția unui cuvânt din etimonul sau, sunt
și exemple care s-au detașat semnificativ de strămoșul lor: latinescul \textit{umbilicu(lu)s} a ajuns
la forma de \textit{buric} în română, \textit{nombril} în franceză și \textit{umbigo} în portugheză.

Detecția automată a cuvintelor cognate poate fi formulată ca o problemă de clasificare prin învățare
automată. O serie de atribute s-au dovedit a fi de succes de-a lungul timpului precum $n$-gramele,
distanțe de editare, cel mai lung subșir comun etc. Jager si Sofroniev \cite{svmclass} determină
perechi cognate folosind un clasificator SVM iar apoi probabilități și distanțe pentru a grupa
cuvinte în grupuri cognate. Rama \cite{cnn} aplica o rețea neurală convoluțională. Ciobanu și Dinu
\cite{theone} folosesc câmpuri aleatoare condiționate reușind să reconstruiască proto-cuvinte din
seturi cognate incomplete.

\section{Latina și limbi romanice moderne}
Română(ro), Italiană(it), Franceză(fr), Spaniolă(es), Portugheză(pt) sunt doar câteva dintre limbile
moderne ce au evoluat din latina, în special din dialectul vulgar (latina vorbita de oamenii de rand 
în Imperiul Roman). Bineînțeles, provenind din aceeași limbă, se pot pune foarte multe probleme de 
natură lingvistică precum gradul de similaritate dintre acestea, găsirea grupurilor de cuvinte cognate,
reconstrucția de cuvinte latinești neatestate și asa mai departe.

Tranziția de la latină la o limbă romanică moderna s-a efectuat prin schimbări majore de vocabular, 
sintaxă și fonologie. Aceste schimbări sunt mai mult sau mai puțin similare pentru fiecare limbă
romanică modernă. În principal, s-a urmărit simplificarea vocabularului prin eliminarea arhaismelor
și a excepțiilor în favoarea regulilor clare, reducerea sinonimelor, stabilirea clară a sensurilor
cuvintelor (conform \cite{sala}).

\section{Obiective si abordare}
Pentru limbile romanice \textit{ro, it, fr, es, pt} există un dicționar etimologic în \cite{ripeanubook}
pornind de a limba latină. Acest set este incomplet și ne propunem sa-l completam folosind o metoda 
computațională bazata pe lingvistica comparativă\cite{sub}. Evaluarea rezultatelor va fi făcută în 
mare parte manual, aceste cuvinte nefiind atestate. Metoda computațională are la baza ideea prezentată în 
\cite{theone}.

Astfel, pornind de la mai multe perechi cognate din limbi romanice moderne, se va aplica o metodă
de reconstrucție a cuvintelor bazată pe etichetarea secvențelor și câmpuri aleatoare condiționate. 
Fiecare limba modernă va fi analizată separat în raport cu limba latină. Apoi, rezultatele din toate 
limbile romanice vor fi combinate folosind agregări pe baza distanței rank \cite{rankdistance}.
Toate acestea vor fi detaliate pe larg în următoarele capitole.

\chapter{Reconstrucție de cuvinte latinești}
\label{chap:two}
Pornim de la formele cuvintelor din limbile romanice moderne. Fiind date mai multe perechi de cuvinte
cognate vrem să deducem forma latinescului strămoș comun. Aplicăm o metodă similară cu cea folosită
în \cite{theone}. Ca și în \cite{theone}, ne bazăm pe faptul că modificările ortografice sunt strâns
legate de evoluția cuvintelor. Deci încercam să reconstruim proto-cuvinte din forma ortografică a
cuvintelor moderne.

În final, dorim să obținem o listă cu $n$ cele mai bune predicții pe care mai apoi sa le prelucrăm
într-o manieră atât automată cât și manuală pentru a obține cele mai bune rezultate.

\section{Prezentare generala}
Dat fiind mai multe seturi de date de cuvinte cognate din limbi romanice moderne, metoda de 
reconstrucție va încerca să aproximeze forma latinescului de proveniență. Vom folosi: \textit{Română},
\textit{Italiană}, \textit{Franceză}, \textit{Spaniolă}, \textit{Portugheză}, iar seturile de date
vor avea forma $(\textit{cuvânt modern}, \textit{cuvânt latinesc})$. Din acestea, modelul va învăța 
pe baza câmpurilor condiționate aleatoare (\textit{conditional random fields} sau prescurtat CRF) 
diverse schimbări de ortografie suferite de cuvintele latinești pentru a le forma pe cele moderne. 
Apoi, vom aplica o tehnica de agregare a acestor rezultate pentru a combina informație din toate 
limbile. În final, vom folosi aceste schimbări pentru a oferi variante de cuvinte ce 
completează anumite seturi de date.

Pașii algoritmului pentru o anumită limbă romanică modernă sunt:

\begin{enumerate}
  \item Pentru fiecare pereche $(\textit{cuvânt modern}, \textit{cuvânt latinesc})$, vom alinia
    cele două cuvinte pentru a înțelege ce semne ortografice s-au păstrat, schimbat sau elidat.
  \item Pregătim antrenarea sistemului CRF: extragem caracteristici din alinierile fiecărei perechi.
  \item Rulăm sistemul CRF și obținem liste de $n$ cele mai bune producții sortate în funcție de
    probabilitatea lor.
\end{enumerate}

\section{Aliniere}
Avem perechi de tipul $(\textit{cuvânt modern}, \textit{cuvânt latinesc})$ pe care vrem să le aliniem.
Nu orice aliniere ne oferă informație validă. Avem nevoie de așa numitele alinieri optime, în care
numărul de diferențe dintre cele două cuvinte este minim. Vom aplica algoritmul de aliniere 
Needleman-Wunsch\cite{needle} din bioinformatică, folosit cu succes și în probleme de procesare al
limbajului natural. 

\subsection{Needleman-Wunsch}
Algoritmul de aliniere Needleman-Wunsch provine din bioinformatică, mai exact din alinierea secvențelor
de proteine sau nucleotide. Determinarea alinierilor se face printr-o tehnică
de programare dinamică. Așadar, problema inițială va fi împărțită în subprobleme fie deja calculate,
fie mai ușor de calculat. De fiecare dată când vom spune alinieri ne vom referi doar la alinieri de tip
Needleman-Wunsch.

Avem doua șiruri $a=a_1a_2...a_n$ și $b=b_1b_2...b_m$ de caractere de lungime $n$ respectiv $m$. 
Vrem sa aliniem șirul $b$ pentru a se potrivi cu șirul $a$.  

Există 3 tipuri de operații într-o aliniere la o anumită poziție $i$:
\begin{enumerate}
  \item \textbf{Potrivire}, caracterele aliniate se potrivesc: $a_i=b_i$
  \item \textbf{Nepotrivire}, caracterele aliniate nu se potrivesc: $a_i \neq b_i$
  \item \textbf{Spațiu}, caracterul din primul șir nu se aliniază cu niciun caracter din al doilea șir
    sau invers
\end{enumerate}

Se observă că în cazul operației de tipul 3, caracterele din dreapta poziției curente $i$ se vor
deplasa cu o poziție.

Fiecare dintre operațiile de mai sus are un anumit cost. În problema noastră de aliniere a unui
cuvânt latinesc cu un cuvânt modern, vom considera costul unei operații de \textbf{Potrivire} ca fiind $0$,
iar costul unei operații de \textbf{Nepotrivire} sau \textbf{Spațiu} ca fiind 1. Astfel, alinierea
optimă va fi cea cu costul minim. În procesul de potrivire nu  vom lua în considerare diacriticele.
Literele ce conțin astfel de semne vor fi considerate la fel cu literele de bază (spre exemplu,
caracterul \textit{\`{e}} poate fi potrivit cu \textit{e}).

Problema se rezolvă folosind tehnica programării dinamice. Definim
două matrici bidimensionale cu $n+1$ și $m+1$ coloane numerotate de $0$ la $n$ respectiv de la $0$
la $m$:

\begin{gather*}
  D_{i,j} = \text{costul minim pentru a alinia prefixul } a_1a_2...a_i \text{ din șirul a} \\
  \text{cu prefixul } b_1b_2...b_j \text{ din șirul b} \\
  P_{i,j} = \text{ultima operație efectuată } \textbf{Potrivire }, \textbf{Nepotrivire } \text{sau} 
  \textbf{ Spațiu}
\end{gather*}

Considerăm faptul că prefixul vid va fi reprezentat de poziția fie cu linia $0$ (în cazul în care
prefixul vid provine din șirul $a$), fie cu coloana $0$ (în cazul în care prefixul vid provine
din șirul $b$).

Relația de recurență este:
\begin{gather*}
  D_{0,j} = j, \forall j=0,...,m \\
  D_{i,0} = i, \forall i=0,...,n \\
  D_{i,j} = \min \begin{dcases}
      D_{i-1,j-1}       &, a_i=b_j \textbf{ Potrivire} \\
      D_{i-1,j-1}+1     &, a_i \neq b_j \textbf{ Nepotrivire} \\
      D_{i-1,j}         &, \textbf{ Spațiu} \\
      D_{i,j-1}         &, \textbf{ Spațiu} \\
    \end{dcases}
\end{gather*}

Scorul alinierii se va afla pe poziția $D_{n,m}$. Pentru a reconstitui alinierea, la fiecare pas din 
recurență trebuie să reținem în matricea $P$ ultima operație efectuată. Acum, pornim cu doi
indici $i = n$ și $j = m$. Dacă pentru a rezolva subproblema determinată de prefixele $a_1a_2...a_i$
și $b_1b_2...b_j$ am folosit ca și ultimă operație:

\begin{itemize}
  \item \textbf{Potrivire}: aliniem caracterul de pe poziția $i$ din $a$ cu cel de pe poziția $j$ 
    din $b$; decrementăm indicii $i$ și $j$  
  \item \textbf{Spațiu}: deducem dacă spațiul este în șirul $a$ sau $b$ și refacem indicii corespunzător
    (dacă spațiul provine din șirul $a$ atunci decrementăm indicele $j$, iar dacă spațiul provine
    din șirul $b$ decrementăm indicele $i$
  \item \textbf{Nepotrivire}: aliniem caracterele diferite de pe pozițiile $i$ din $a$ și $j$ din $b$;
    decrementăm indicii $i$ și $j$
\end{itemize}

    
\section{Câmpuri condiționate aleatoare}
În continuare vrem să învățăm care sunt schimbările ortografice produse în evoluția unui cuvânt 
modern știind etimonul său. În secțiunea precedentă am aliniat cuvintele pentru a determina ce  
modificări au suferit caracterele. Folosim un algoritm de învățare automată pentru a studia
șabloane de schimbări ortografice dintre fiecare limbă modernă și limba latină.

Propunem o metodă bazată de câmpuri condiționate aleatoare, întrucât acestea au dat rezultate satisfăcătoare în generarea transliterațiilor \cite{ganesh} și în producția de cuvinte cognate \cite{crfciobanu}.

Vom explica mai întâi câteva noțiuni despre câmpurile condiționate aleatoare iar apoi le vom aplica 
pe problema reconstrucției de cuvinte.

\subsection{Generalități}
Câmpurilor condiționate aleatoare (\textit{conditional random fields} sau prescurtat CRF) sunt o
metodă de modelare statistică pentru a face predicții structurate.

Lafferty, McCallum si Pereira\cite{crf} sunt primii care explică această structură. Ei introduc un 
nou cadru pentru construirea modelelor probabilistice pentru segmentarea și etichetarea datelor
secvențiale. Până în acel moment, astfel de probleme erau rezolvate prin modele Markov ascunse și 
gramatici stocastice.

Conform \cite{crf}, fie $X$ o variabilă aleatoare peste secvențe de date ce trebuie etichetate și 
$Y$ o variabilă aleatoare peste etichetele corespunzătoare. Construim un model de probabilități
condiționate $P(Y|X)$ din perechile de secvențe de date și etichete.

\begin{definition}
Fie $G=(V, E)$ un graf astfel încât $Y=(Y_i)_{i \in V}$ (nodurile grafului reprezintă
indecșii etichetelor $Y$). $(X, Y)$ se numește câmp condiționat aleator dacă variabila aleatoare
$Y_v$ condiționată de $X$ respectă proprietatea Markov raportat la graful G: 
  \begin{gather*}
    P(Y_v | X, Y_w, w \neq v) = P(Y_v | X, Y_w, w \sim v)
  \end{gather*}
unde $w \sim v$ înseamnă că există muchia $(w, v)$ în $E$.
\end{definition}

Observăm că CRF-ul este global condiționat de variabila $X$. Graful $G$ este nedirecționat, construit
diferit în funcția de atributele secvențelor de etichetat dar de cele mai multe ori are forma unui
lanț sau arbore.

Pentru atribute, vom defini o funcție pentru a reprezenta caracteristici ale secvențelor de date. 
Valoarea acestor funcții se calculează în funcție de problema pe care dorim să o rezolvăm. Le vom nota 
cu $f_k$, $g_k$ etc. Spre exemplu $g_k(v, y|S, x)$ va fi adevărat dacă cuvântul X începe cu o majusculă, 
iar eticheta $Y$ este "substantiv propiu".\cite{crf} $y|S$ reprezintă  setul de componente ale lui $Y$ 
asociate cu nodurile subgrafului $G$. Funcțiile de atribute trebuie calculate înaintea aplicării 
sistemului CRF. 

Atribuim fiecărui atribut o anumită pondere $\lambda_i$ (pentru atributele în raport cu muchile grafului $G$)
și $\mu_i$ (pentru atributele reportate la nodurile grafului $G$). Acestea sunt valorile pe care algoritmul
trebuie să le învețe pentru a face mai apoi predicțiile.

Exista doua structuri de grafuri pe care se pretează să aplicăm sistemul CRF: cele lanț și cele
arborescente.

Formula probabilităților condiționate pentru structura de arbore este:
\begin{gather*}
  P(Y|X) \propto \exp(\smashoperator{\sum_{e \in E, k}} \lambda_k f_k(e, y|e, x) + \smashoperator{\sum_{v \in V, k}} \mu_k g_k(v, y|v, x))
\end{gather*}

Pentru a determina formula probabilităților condiționate pentru structura de lanț vom nota $start = Y_0$ și $stop = Y_{n+1}$
Probabilitățile condiționate le vom calcula într-o matrice $M$ astfel:
\begin{gather*}
  M_i(y', y|x) = exp(\Lambda_i(y', y|x)) \\
  \Lambda_i(y', y|x) = \smashoperator{\sum_k} \lambda_k f_k(e_i, Y|e_i = (y', y), x) +
                       \smashoperator{\sum_k} \mu_k g_k(v_i, Y|v_i = y, x).
\end{gather*}
unde, $e_i$ este muchia cu etichetele $(Y_{i-1}, Y_i)$ și $v_i$ este nodul corespondent lui $Y_i$. 

Pentru a estima parametrii $\lambda_i$ și $\mu_i$ se folosesc algoritmi iterativi pentru a maximiza
rezultatele pe un set de date de antrenare. Algoritmii sunt prezentați în \cite{crf}.

Acum pute scrie formula pentru probabilitățile condiționate pentru un graf de tip lanț:
\begin{gather*}
  P(Y|X) \propto \frac{\smashoperator{\prod_{i=1}^{n+1}} M_i(y_{i-1}, y_i | x)}{(\smashoperator{\prod_{i=1}^{n+1}} M_i(x))_{start,stop}}
\end{gather*}

Precum și în alte modele statistice ce folosesc probabilități condiționate (spre exemplu: Naive Bayes),
predicția se face prin:

\begin{gather*}
  \hat{Y} = argmax_Y(P(Y|X))
\end{gather*}

Putem lua chiar și primele $n$ cele mai bune probabilități pentru face o analiză în profunzime.

\subsection{Aplicate pe problema reconstrucției cuvintelor}
\label{subs:one}
Ne întoarcem la problema de predicție a cuvintelor latinești pornind de la cuvinte moderne folosind 
sisteme CRF. Secvențele de date vor fi cuvintele moderne. Atributele vor fi $n$-grame din cuvintele
de intrare, extrase din ferestre de dimensiune $w$. 

Etichetele se calculează folosind algoritmul de aliniere, utilizând caracterele ce se potrivesc în 
cuvântul modern și cuvântul latinesc. Dar, pentru că cele două cuvinte nu sunt complet identice,
în cazul unei inserări vom pune caracterul adăugat la eticheta precedentă (nu putem asocia aceasta
literă cu niciuna din cuvântul de intrare). În cazul unui spațiu, apărut în cuvântul latinesc se 
produce practic un fenomen de elidare. Asociem caracterului respectiv din cuvântul sursă, o etichetă
nouă (spre exemplu $-$) pentru a semnifica dispariția acestuia.

Sistemul are nevoie de margini la capetele cuvintelor, în cazul în care avem inserții produse la 
începutul sau sfârșitul cuvântului, adică o eticheta "precedentă" cu care să asociem aceste litere noi.
Așadar, fiecare cuvânt va fi extins prin adăugarea a două caractere \textbf{B} și \textbf{E} la 
începutul și sfârșitul acestuia.\cite{theone} Orice literă nouă adăugata la început sau la sfârșit, va
fi asociată cu aceste 2 litere speciale.


Folosim 5 astfel de sisteme CRF pentru fiecare limba modernă \textit{română}, \textit{italiană}, 
\textit{franceză}, \textit{spaniolă}, \textit{portugheză} pusa în raport cu \textit{limba latină}.
Fiecare sistem va calcula liste de cele mai bune $n$ cuvinte latinești sortate în funcție de 
probabilitate. Pentru a profita de toate limbile moderne propunem o metodă de combinare a rezultatelor
sistemelor CRF bazată pe agregări folosind distanța rank.

\chapter{Agregarea rezultatelor folosind distanța rank}
\label{chap:three}
Am văzut cum putem obține producții de cuvinte combinând câte o singură limbă romanică cu 
limba latină. Pentru a îmbunătății rezultatele vrem să folosim informația din mai multe 
limbi romanice moderne. Astfel, fiecare clasificator întoarce o listă ordonată de cuvinte latinești, 
pe prima poziție aflându-se etimonul latinesc cu cea mai mare probabilitate. Prin agregarea 
acestora cu o anumită metrică vom obține o lista sortata cu mai probabile cuvinte latinești. Metrica
folosita este distanta rank \cite{rankdistance} întrucât s-au obținut rezultate bune în alte 
probleme de natură lingvistică precum determinarea similitudinii silabice a limbilor romanice 
\cite{syllabicsim}, \cite{simnat}.

În primul rând vom defini ce înseamna o listă ordonată de elemente. În al doilea rând, vom explica
distanța rank între două clasamente și între un clasament și o mulțime de clasamente. Apoi vom
prezenta o metodă de aflare a tuturor agregărilor unei mulțimi de mai multe clasamente folosind
distanța rank. În final, vom prelucra mulțimea de agregări bazat pe un sistem de vot pentru a 
determina \textbf{o singură listă ordonată} de posibile etimoane latinești.

\section{Clasamente și distanța rank}
Un \textit{clasament} este o listă ordonată de obiecte după un anumit criteriu, pe prima poziție 
aflându-se cel cu cea mai mare importanță. În unele situații se pune problema găsirii unui clasament
cât mai apropiat de o mulțime de mai multe clasamente. Pentru a rezolva această problemă trebuie să
definim mai întâi ce înseamnă distanța dintre două clasamente sau dintre un singur clasament și o
mulțime de clasamente.

Există mai multe metrici folosite cu succes în diverse aplicații: distanța \textit{Kedall tau}, 
\textit{Spearman footrule}, \textit{Levenshtein}, dar noi vom folosi distanța \textit{rank}
introdusa in articolul \cite{rankdistance}. În întregul capitol vom folosi următoarele notații:
\begin{itemize}
    \item $U = \{1, 2, ..., n\}$ o mulțime finită de obiecte numită univers 
    \item $\tau = (x_1 > x_2 > ... > x_d)$ un clasament peste universul $U$ 
    \item $>$ o relație de ordine strictă reprezentând criteriul de ordonare 
    \item $\tau(x)$ = poziția elementului $x \in U$ în clasamentul $\tau$ dacă $x \in \tau$, 
      numerotând pozițiile de la 1 începând cu cel mai important obiect din clasament
\end{itemize}

Dacă un clasament conține toate elementele din univers, atunci el se va numi 
\textit{clasament total}. Asemănător, dacă conține doar o submulțime de obiecte din univers, atunci
îl vom numi \textit{clasament parțial}.

Notăm ordinea elementului $x$ în $\tau$ astfel:
\begin{gather}
\label{ord}
  ord(\tau, x)= \begin{dcases}
    |n + 1 - \tau(x)|    &, x \in \tau \\
    0                    &, x \in U \setminus \tau
  \end{dcases}
\end{gather}

\begin{definition}
Fie $\tau$ și $\sigma$ două clasamente parțiale peste același univers $U$. Atunci distanța rank va fi
\begin{gather}
  \Updelta(\tau, \sigma) = \smashoperator{\sum_{x \in \tau \cup \sigma}} |ord(\tau, x) - ord(\sigma, x)|
\end{gather}
\end{definition}

Se observă faptul că, în calculul distanței rank, se ia în considerare ordinea definită mai sus
și nu poziția. În primul rând, cum primele poziții sunt cele mai importante, distanța dintre două
clasamente trebuie sa fie cu atât mai mare cu cât diferă mai mult începutul lor.\cite{linguisticstructuresmarcus}
În al doilea rând, definiția funcției $ord$ pune accentul pe lungimea clasamentelor întrucăt putem
presupune că un clasament mai lung a fost obținut în urma comparării mai multor obiecte din univers.
Deci ordinea elementelor este mai solidă. Spre exemplu, dacă două clasamente de lungimi diferite au
același element pe prima poziție, există totuși o diferență a ordinii obiectului în cele două liste,
diferență ce contribuie la calculul distanței rank totale.\cite{rankaggregationproblem}

\subsection{Agregări cu distanța rank}
O \textit{agregare de clasamente} reprezintă un singur clasament $\sigma$ astfel încât o anumită 
metrică de la acesta la mulțimea de liste de agregat $T$ este minimă. Raportându-ne la distanța
rank avem\cite{rankdistance}:

\begin{definition}
Fie un set de clasamente $T = \{\tau_1, \tau_2, ..., \tau_m\}$ dintr-un univers $U$ și
$\sigma = (\sigma_1 > \sigma_2 > ... > \sigma_k)$ un clasament astfel încât $\sigma_i \in U, 
\forall 1 \leqslant i \leqslant k$. Definim distanța rank de la $\sigma$ la $T$ astfel:
\begin{gather}
  \Updelta(\sigma, T) = \smashoperator{\sum_{i = 1}^{m}} \Updelta(\sigma, \tau_i)
\end{gather}
\end{definition}

\begin{definition}
\label{def:Aset}
Se numește mulțime de agregari de lungime $k$ a mulțimii $T$ folosind distanța rank, setul
$
  A(T, k) = \{\sigma=(\sigma_1 > \sigma_2 > ... > \sigma_k) | \sigma_i \in U, 
  \forall 1 \leqslant i \leqslant k$, si 
  $\Updelta(\sigma, T)$ este minim posibila \}
\end{definition}


\begin{problem}
Fie $U$ un set de obiecte și $T = \{\tau_1, \tau_2, ..., \tau_m\}$ o mulțime de clasamente peste
universul $U$. Vrem să determinăm mulțimea de agregări $A(T, k)$ pentru un k fixat. 
\end{problem}

Construim următoarele matrici bidimensionale $W^k(i, j)$ cu $n$ linii și $n$ coloane. Fiecare celulă
din fiecare matrice reprezintă costul total din distanța rank de la un clasament $\sigma$, de 
lungime $l$, către o mulțime $T = \{\tau_1, \tau_2, ..., \tau_m\}$ fixată indus de plasarea 
elementului $x_i \in U$ pe poziția $j$ în $\sigma$ \cite{rankaggregationproblem}. Se observă faptul
că un clasament peste universul $U$ definit mai sus poate avea lungimea maxim $n$. Rezultă că 
numărul de coloane al matricilor $W^t$ este egal cu $n$.
\begin{gather}
  \label{eq:wmatrix}
  W^k(i, j) = \begin{dcases}
    \smashoperator{\sum_{p=1}^{m}} | ord(p, i) - k + j |    &, j \leqslant k \\
    \smashoperator{\sum_{p=1}^{m}} | ord(p, i) |            &, j > k
  \end{dcases}
\end{gather}

\begin{remark}
Distanța de la un clasament $\sigma=(\sigma_1 > \sigma_2 > ... \sigma_k)$ la mulțimea $T$ este
\[
  \Updelta(\sigma, T) = \smashoperator{\sum_{x_i \in U \cap \sigma}} W^k(i, \sigma(x_i)) +
      \smashoperator{\sum_{x_i \in U \setminus \sigma}} W^k(i, k + 1)
\]unde $n$ reprezintă numărul de obiecte din univers, iar $k < n$.
\end{remark}
Se observă faptul că, în cazul în care $\sigma$ conține toate elementele din $U$, deci cazul $k = n$
, formula se devine
\begin{gather}
  \label{eq:minimize}
  \Updelta(\sigma, T) = \smashoperator{\sum_{x_i \in U \cap \sigma}} W^k(i, \sigma(x_i))
\end{gather}

\subsection{Reducerea la o problemă de cuplaj perfect de cost minim}
Fiecare matrice $W^l$ din secțiunea precedentă este calculată în mod independent de celelalte
Deci putem determina doar o singură matrice pentru o anumită lungime fixată $l$. 
Astfel, problema se reduce la găsirea unui clasament $\sigma$ ce 
minimizează formula \eqref{eq:minimize}. Formal:

\begin{problem}
\label{def:problem}
Fiind dată o matrice pătratică $W$, $W = (w_{i, j})_{1 \leqslant i,j \leqslant n}$ vrem să
determinăm următoarea mulțime:
\[
  S = \{(i_1, i_2, ..., i_k) | (i_p \neq i_j, \forall p \neq j), (i_j \in U) \text{ și } \smashoperator{\sum_{j=1}^n} w_{i_j, j} \text{ este minim}\}
\]
\end{problem}

Problema de mai sus se aseamănă cu o problemă de cuplaj de dimensiune $k$ de cost minim întrucât 
vrem să formăm perechi între obiectele dintr-un univers și pozițiile unui clasament de tip agregare, 
iar fiecare combinație are un anumit cost. Practic $(i_1, i_2, ..., i_n)$ reprezintă o permutare a 
elementelor din $U$.

O soluție pentru a rezolva problema precedentă este aplicarea algoritmul Ungar prezentat de Khun \cite{hungarianmethod}.
Altfel, putem considera matricea $W$ ca fiind o matrice de costuri într-un graf bipartit $G$ pe care
aplicăm un algoritm clasic de găsire a cuplajului maxim de cost minim (din care luam doar $k$ perechi). 
Conform \cite{flowassignment} această problemă poate fi rezolvată în timp polinomial 
$\mathcal{O}(n^3)$ construind o rețea de flux cu capacități convenabile și prin găsirea unor drumuri 
de augmentare minime, din punct de vedere al costului, folosind algoritmul lui Dijkstra\cite{dijkstra}.

Toate aceste rezolvări determină o singură agregare dar nu și pe toate, adică mulțimea $A(T, k)$ din
definiția \ref{def:Aset}. În continuare prezentăm o metoda de determinare a tuturor agregărilor 
bazată pe găsirea tuturor cuplajelor \textit{perfecte} de cost minim dintr-un graf, metoda 
prezentată în \cite{allmatchings}. Algoritmul rulează într-un timp polinomial. Particularizăm 
problemele și algoritmii din articolul  \textit{A generalization of the assignment problem, and its
application to the rank aggregation  problem} \cite{allmatchings} pentru Problema \ref{def:problem}.


\subsection{Calcularea tuturor agregărilor optime}
Reamintim faptul că dorim să calculăm mulțimea de agregări $A(T, k)$, știind costul plasării 
fiecărui element pe fiecare poziție, memorat în matricea $W^k$ calculată la \eqref{eq:wmatrix}. 
Reformulăm problema în elemente de teoria grafurilor. Astfel, asociem Problemei \ref{def:problem} 
un graf $G = (V, E, c, w)$, unde $V$ reprezintă mulțimea de noduri, $E$ este mulțimea de muchii iar
$c \colon E \to \mathbb{N}$ și $w \colon E \to \mathbb{N}$ reprezintă capacitatea unei muchii
respectiv costul acesteia. Legăturile între Problema \ref{def:problem} și graful $G$ sunt:
\begin{itemize}
  \item $V = \{src, dst\} \cup U \cup \{1, 2,..., k, k+1\}$
  \item $E = \{(src, x_i) | x_i \in U\} \cup \{(x_i, j) | x_i \in U \text{ si } j = 1,...,k\} \cup 
    \{(j, dst) | j = 1,...,k\}$  
  \item $c(muchie) = 1, \forall muchie = (x, j) \in E, j \neq k+1$
  \item $c(muchie) = \infty, \forall muchie = (x, k+1) \in E$
  \item funcția $w$ astfel:
  \begin{itemize}
    \item $w((src, x_i)) = 0, \forall x_i \in U$
    \item $w((x_i, j)) = W^k(i, j), \forall x_i \in U, j = 1,...,k+1$
    \item $w((j, dst)) = 0, j = 1,...,k+1$
  \end{itemize}
\end{itemize}

Potrivirea unui element $x$ cu poziția $k+1$ va semnifica excluderea acestuia din agregare ce 
afectează distanța rank dintre un clasament ce nu conține elementul $x$ și mulțimea întreagă $T$.
Se poate calcula ușor în acest graf un flux maxim de cost minim folosind algoritmi clasici\cite{flowassignment}).
Notăm prin $solve(W)$ un asemenea algoritm.
Fie soluția $M = \{(x, j) | x \in U \text{ și } j = 1,...,k\}$. Următorul pas este aflarea unei
soluții $M'$ diferite de $M$.

\begin{proposition}
Doua cuplaje $M$ și $M'$ sunt diferite dacă există cel puțin o pereche $(x, y)$ care se află în $M$
și nu se află în $M'$.
\end{proposition}

Astfel, propunem următorul algoritm, adaptat din \cite{allmatchings}, prin care căutăm o a doua
soluție $M'$ fixând câte o muchie candidat $(x, y)$ prin care $M'$ sa difere de $M$. Setând costul
muchiei $(x, y)$ pe o valoare infinită, avem garanția ca aceasta nu va fi luată în considerare în 
construcția lui $M'$.
\begin{algorithm}
\label{P}
\caption{anotherSolution}
\begin{algorithmic}[1]
\REQUIRE $W, M$
\ENSURE $M'$
  \STATE $s \gets \sum_{(u, v) \in M} w_{uv}$
  \FORALL{$(x, y) \in M$}
    \STATE $temp \gets w_{xy}$
    \STATE $w_{xy} \gets \infty$
    \STATE $M' \gets solve(W)$
    \IF{$M' \neq \emptyset \text{ si } \sum_{(u, v) \in M'} w_{uv} = s$}
    \RETURN $M'$
    \ELSE
    \STATE $w_{xy} \gets temp$
    \ENDIF
  \ENDFOR
  \RETURN $\emptyset$
\end{algorithmic}
\end{algorithm}

Algoritmul returnează fie mulțimea vidă, fie o soluție $M'$ astfel încât exista o pereche $(x, y)
\in M \setminus M'$. Se poate împărți problema inițială în două subprobleme disjuncte $P_1$ și
$P_2$:
\begin{itemize}
  \item $P_1\colon$ mulțimea tuturor cuplajelor ce conțin muchia $(x, y)$ \\
    În acest caz forțăm păstrarea perechii în soluție prin setarea tuturor celorlalte valori de pe
    linia $x$, coloana $y$ pe o valoare infinita în matricea $W$:
    $w_{iy} = w_{xj}, \forall i = 1,..,n \text{, }i \neq x  \text{ și } j = 1,...,n \text{, }j \neq y$
  \item $P_2\colon$ mulțimea tuturor cuplajelor ce \textbf{nu} conțin muchia $(x, y)$ \\
    În acest caz perechea $(x, y)$ nu va mai fi niciodată aleasă într-o soluție dacă costul acesteia
    este infinit:
    $w_{xy} = \infty$
\end{itemize}

Evident, există deja cate o soluție calculată pentru cele 2 subprobleme și anume $M \in P_1$ și 
$M' \in P_2$. Prim urmare, se poate aplica Algoritmul 1 în mod recursiv pentru fiecare dintre 
aceste subprobleme pentru a determina întreaga mulțime de soluții. Această abordare conduce la
construirea unei structuri de căutare arborescente în care rădăcina reprezintă problema inițială
\ref{def:problem}, iar fiecare nod intern constituie o împărțire pe subprobleme după o pereche 
$(x, y)$. Soluția finală se construiește traversând arborele în adâncime și păstrând toate soluțiile
parțiale calculate la fiecare pas. Nu se va genera aceeași soluție de mai multe ori prin faptul că
problemele $P_1$ și $P_2$ sunt complet disjuncte.

\begin{algorithm}
\caption{dfsAgregare}
\begin{algorithmic}[1]
\REQUIRE $S, M, W$
  \STATE $s \gets \sum_{(u, v) \in M} w_{uv}$
  \STATE $M' \gets another_solution(W, M)$
  \IF{$M' \neq \emptyset$}
    \RETURN
  \ELSE
    \STATE $S \gets M'$
    \STATE $(x, y) \in M \setminus M'$
    \STATE $w_{xy} \gets \infty$
    \STATE $dfsAgregare(S, M', W)$
    \STATE $w_{iy} \gets w_{xj}, \forall i = 1,..,n \text{, }i \neq x  \text{ si } j = 1,...,n \text{, }j \neq y$
    \STATE $dfsAgregare(S, M' \setminus (x, y), W)$
  \ENDIF
\end{algorithmic}
\end{algorithm}

\begin{algorithm}
\caption{Calculează toate cuplajele perfecte de cost minim}
\begin{algorithmic}[1]
\REQUIRE $W$
\ENSURE $S$
  \STATE $S \gets \emptyset$
  \STATE $M \gets solve(W)$
  \STATE $S \gets S \cup M$
  \STATE $dfsAgregare(S, M, W)$
  \RETURN $S$
\end{algorithmic}
\end{algorithm}

Algoritmul 3 determină toate cuplajele perfecte de cost minim pornind de la o matrice de costuri
$W$. Calculând matricea $W$ după formula \eqref{eq:wmatrix}, atunci mulțimea $S$ determinată de
algoritm este chiar soluția căutată în Problema \ref{def:problem}. Câteva aspecte legate de
complexitatea metodei prezentate: \\
Fie $x = |S|$, numărul total de soluții pentru o anumita problema.
\begin{itemize}
  \item Pentru fiecare soluție nou calculată, se construiesc două noi alte probleme. În total se vor
    rezolva maxim $2*x+1$ subprobleme.
  \item O subproblemă necesită găsirea unui cuplaj de cost minim ce se poate calcula într-un timp
    polinomial folosind metoda Ungară\cite{hungarianmethod} ori un algoritm clasic de determinare
    a fluxului maxim de cost minim într-un graf bipartit\cite{flowassignment}.
\end{itemize}
Intuitiv, putem afirma ca Algoritmul 3 rulează într-un timp polinomial. Complexitatea acestuia a
fost demonstrată în \cite{allmatchings} ca fiind $\mathcal{O}((2x+1)k^3n\log(n+k))$.

Subproblemele sunt complet independente de complementarele lor. Prin urmare, rezolvările acestora se
pot rula în paralel pe mai multe fire de execuție. În experimentele rulate am ales să pornim
un nou fir de execuție pentru fiecare subproblemă de tipul $P_2$ până la un anumit nivel din
arborele de căutare determinat în funcție de procesorul folosit. Subproblema de tipul $P_1$
a rămas să ruleze pe firul de execuție principal.


\section{Determinarea tuturor agregărilor producțiilor de cuvinte}
În capitolul precedent am prezentat o metoda de a combina o limbă romanică modernă și limba latină
pentru a automatiza procesul de determinare a etimonului latinesc. Metoda returna primele $n$
cuvinte posibile ordonate de la cel cu probabilitatea cea mai mare la cel cu probabilitatea cea mai
mică. Vom considera aceste liste de cuvinte ca fiind clasamente. Pentru fiecare cuvânt latinesc vom
agrega clasamentele produse din fiecare limbă romanică modernă \textit{(ro, it, fr, es, pt)}.
Se observă faptul că pot exista mai multe astfel de agregări așa că ne propunem să le aflăm pe toate
într-un mod eficient din punct de vedere al complexității timp. Alegem să luăm în considerare
doar primele 5 cele mai bune cuvinte din fiecare set. Astfel, pentru un singur cuvânt latinesc, vom 
avea:
\begin{gather*}
  R = \{r_1, r_2, ..., r_k\}, \text{clasamentele produse din ro, it, fr, es, pt} \\
  k = |R|, 1 \leqslant k \leqslant 5 \\
  U = \smashoperator{\bigcup_{i = 1}^{k}} r_i \text{ universul de cuvinte} \\
  n = |U|
\end{gather*}
Definim o matrice bidimensională de $k$ linii și $n$ coloane în care calculăm ordinea fiecărui 
cuvânt din universul unui cuvânt în fiecare clasament dat:
\[
  ord[i][j] = \begin{dcases}
    |6 - r_i(x_j)|    &, x_j \in r_i \\
    0                 &, x_j \in U \setminus r_i 
  \end{dcases}
\]

Calculăm apoi matricea de costuri $W^5$ conform formulei de la \ref{eq:wmatrix} pentru a afla cum
este afectat costul final dacă un anumit cuvânt $x_i \in U$ este plasat pe poziția $j$. Aceasta 
poziție poate sa fie mai mare decât $5$ dar conform formulei construite, nu se va face nicio 
distincție între a plasa un cuvânt pe poziția $6$ spre exemplu, sau poziția $7$, considerându-se
că respectivul element nu va aparține în agregarea finala de lungime $5$.

Este limpede ca se poate aplica acum Algoritmul 3 prezentat în secțiunea precedentă pentru a afla
mulțimea $S = \{\sigma_1, \sigma_2, ..., \sigma_t\}$ de agregări posibile. Propunem combinarea 
soluțiilor din $S$ printr-un sistem de vot. Fiecare cuvânt va primi un scor bazat pe pozițiile pe 
care se află în clasamentele din $S$.

\begin{gather*}
  scor \colon U \to \mathbb{N} \\
  scor(x_i) = \smashoperator{\sum_{j=1}^{|S|}} \sigma_j(x_i), \forall x_i \in U
\end{gather*}

Putem în sfârșit să construim lista finală, introdusă vag la începutul capitolului, de cuvinte 
ordonate în funcție de probabilitatea de a fi etimonul latinesc al unor anumite cuvinte moderne 
din \textit{ro, it, fr, es, pt}. Pe prima poziție se va afla cuvântul produs cu scorul cel mai mic,
adică cel ce se află pe primele poziții în clasamentele $S$, pe a doua poziție, cel cu următorul cel
mai mic scor, și asa mai departe. În cazul în care există mai multe cuvinte cu același scor, le 
vom păstra pe toate pe aceeași poziție și le vom filtra manual folosind reguli lingvistice.

\begin{gather*}
  F = (x_1 > x_2 > x_3 > x_4 > x_5) \\
  x_1 = \min (scor(x)) \\
  x_2 = \min_{x, x \neq x_1} (scor(x)) \\
  x_3 = \min_{x, x \neq x_1, x_2} (scor(x)) \\
  x_4 = \min_{x, x \neq x_1, x_2, x_3} (scor(x)) \\
  x_5 = \min_{x, x \neq x_1, x_2, x_3, x_4} (scor(x)) \\
  \{x_1, x_2, x_3, x_4, x_5\} \subseteq U
\end{gather*}


\chapter{Experimente si rezultate}
In capitolul \ref{chap:two} am prezentat pasii metodei propuse pentru reconstructia cuvintelor
latinesti. Pe scurt, vom antrena mai multe sisteme CRF pe un set de date pentru a face mai apoi 
predictii de cuvinte. Din date vor fi extrase anumite atribute pe baza unor alinieri, conform 
sectiunii \ref{subs:one}. Predictiile sunt apoi agregate pe baza distantei rank ca in capitolul
\ref{chap:three}.

In experimentul rulat ne propunem sa reconstruim cuvinte romanesti neatestate pornind de la un posibil
etimon latinesc.  

\section{Antrenare}
Pentru antrenarea sistemelor CRF am folosit datasetul propus de Ciobanu si Dinu \cite{dataset}.
Acesta contine 3218 grupuri complete de cuvinte cognate pentru cinci limbi romanice (romana, italiana, 
franceza, spaniola, portugheza) si etimoane lor latinesti. Impartim acest set de date in trei parti,
pentru antrenare, pentru reglarea parametrilor si pentru testare. Dimensiunea proportilor este $3:1:1$.

Aliniem fiecare pereche de cuvinte cognate din care extragem 3-grame ca si atribute pentru sistemele CRF.
Folosim implementarea data de Mallet toolkit \cite{mallet}. Parametrii sunt gasiti printr-o cautare
de tip \textit{grid search} cu numarul de iteratii $\{1, 5, 10, 25, 50, 100\}$ si pentru dimensiunea
ferestrei $w$ de $\{1, 2, 3, 4, 5\}$.

Pentru cei mai buni parametri (50 de iteratii) obtinem acuratetea top-10:
\begin{itemize}
    \item Spaniola   $61\%$
    \item Franceza   $62\%$
    \item Italiana   $62\%$
    \item Portugheza $51\%$
    \item Latina     $65\%$
\end{itemize}

\section{Testare}
Aplicam modelele invatate pe o lista de 235 de cuvinte cognate ce exista in limbile romanice din 
Occident dar nu si in limba romana. Acestea au fost extrase dintr-o lista propusa de Ripeanu in \cite{ripeanubook}.
Agregam apoi rezultatele folosind metoda din capitolul \ref{chap:three}. Procesam productiile conform 
urmatoarelor doua reguli:

\begin{enumerate}
  \item diftongul \textit{iă} nu exista in limba romana si il inlocuim cu \textit{ie}
  \item consoanele duble dispar (spre exemplu \textit{ll} devinte \textit{l})
\end{enumerate}

\section{Evaluare}

\subsection{Producerea de cuvinte cognate}
Evaluarea a fost facuta manual intrucat natura problemei face dificila gasirea unei metode de evaluare
automata. Algoritmul prezentat produce cuvinte romanesti neatestate deci doar un expert in domeniu
le-ar putea valida. Am decis sa folosim doar italiana si latina pentru a reconstrui cuvinte romanesti
deoarece spaniola, franceza, portugheza sunt mai departe de romana din punct de vedere ortografic,
iar metoda prezentata se bazeaza pe analiza schimbarilor grafice.

Rezultatele obtinute prin evaluarea manuala a productiilor de cuvinte romanesti pornind de la limba
latina si italiana:
\begin{center}
  \begin{tabular}{|| l l l ||}
    \hline
    Tipul cognat romanesc & Latina & Italiana \\[0.5ex]
    \hline
    \hline
    Reali                                & 82 (34.8\%)  & 72 (30.6\%) \\
    \hline
    Reali cu interventie lingvistica     & 12 (5.1\%)   & 11 (4.6\%) \\
    \hline
    Virtuali                             & 69 (29.3\%)  & 32 (13.6\%) \\
    \hline
    Virtuali cu interventie lingvistica  & 28 (11.9\%)  & 11 (5.1\%) \\
    \hline
    Inexistenti                          & 51 (21.7\%)  & 111 (47.2\%) \\
    \hline
  \end{tabular}
\end{center}

\begin{itemize}
  \item cognat \textbf{real}: cuvant ce exista in limba romana datorita unor procese de dezvoltare
    interne ale limbii sau prin imprumutarea masiva a acestora din alte limbi
  \item cognat \textbf{real cu interventie lingvistica}: cuvant ce exista in limba romana in urma
    aparitiei unor noi criterii lingvistice
  \item cognat \textbf{virtual}: cuvant ce ar fi putut exista in limba romana dar care nu au fost
    atestate
  \item cognat \textbf{virtual cu interventie lingvistica}: cuvant ce ar fi putut exista in limba
    romana in urma introducerii unor noi criterii lingvistice dar care nu au fost atestate
  \item cognat \textbf{inexsitent}: cele ce nu apartin niciunei categorii de mai sus
\end{itemize}

\subsection{Reconstruirea cuvintelor latinesti neatestate}
In acest caz, setul de date contine cuvantul latinesc propus dar neatestatat. Este vorba de un subset
al setului de date propus de Ripeanu in \cite{ripeanubook}. Deci am putut rula o evaluare automata
a rezultatelor. In urma agregarii cu distanta rank am obtinut o acuratete de $23\%$ in top-10. 
Adica am verificat ca etimonul latinesc sa se afle in primele zece productii.

Dar, daca nu am folosi agregarea ci doar limba italiana, se obtine o acuratete top-10 de $38\%$.
Astfel, metoda confirma mai mult de un sfert din cuvintele neatestate latinesti, reconstruite artificial.

\chapter{Concluzii}
Aceasta lucrare prezinta o metoda computationala aplicata cu succes in reconstructia cuvintelor 
cognate in seturi de date incomplete si in producerea etimonului latinesc in cazul limbii romane.
Pentru acestea se folosesc cinci limbi romanice moderne: spanionla, italiana, franceza, portugheza
si romana. Algoritmul are la baza idei preluate din lingvistica comparativa incercand sa automatizeze
procesul de detectie al sabloanelor de modificare ortografica in evolutia cuvintelor. Aceasta nu ia
in considerare alti factori precum cei sociali, economici, tehnologici etc, factori ce ar fi dificili
de modelat matematic.

Astfel, productiile obtinute vin in ajutorul expertilor in domeniul lingvisticii istorice pentru
studiul evolutiei si recostructiei artificiale de limbaj.

\section{Imbunatatiri}
In primul rand, o idee ce merita a fi cercetata este adaugarea dimensiunii fonetice in analiza 
cuvintelor cognate din limbile romanice moderne si latina. Asadar, sistemul va fi adaptat pentru a 
combina deductiile bazate pe schimbarile ortografice cu cele bazate pe transcrierea fonetica.

In al doilea rand, am putae extinde algoritmul pentru a lua in considerare toate limbile romanice
moderne precun \textit{catalana}, \textit{siciliana} si dialectele lor. Cel mai probabil, ar trebui
grupate in functie de similaritate pentru a obtine cele mai bune rezultate, cum am procedat si in 
experimentele rulate (italiana, romana si latina). Spre exemplu, catalana este mai apropiata de 
spaniola deci schimbarile produse intr-una dintre ele au sanse mai mari sa fie similare decat
cele produse in romana.

In al treilea rand, merita cercetata adaptabilitatea metodei prezentate in adaugarea dimensiunii 
semantice a cuvintelor. Putem oare adauga sensul cuvintelor ca un alt set de atribute in invatarea
sistemelor CRF?




\appendix

\chapter{Dummy Appendix}

You can defer lengthy calculations that would otherwise only interrupt
the flow of your thesis to an appendix.



\backmatter

\bibliographystyle{plain}
\bibliography{refs}

\end{document}

