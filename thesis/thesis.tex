% Template version used: v1.4
%
% Largely adapted from Adrian Nievergelt's template for the ADPS
% (lecture notes) project.
% Adapted for a bachelor's thesis


%% We use the memoir class because it offers a many easy to use features.
\documentclass[11pt,a4paper,titlepage]{memoir}

%% Packages
%% ========

%% LaTeX Font encoding -- DO NOT CHANGE
\usepackage[OT1]{fontenc}

%% Babel provides support for languages.  'english' uses British
%% English hyphenation and text snippets like "Figure" and
%% "Theorem". Use the option 'ngerman' if your document is in German.
%% Use 'american' for American English.  Note that if you change this,
%% the next LaTeX run may show spurious errors.  Simply run it again.
%% If they persist, remove the .aux file and try again.
\usepackage[romanian]{babel}

%% Input encoding 'utf8'. In some cases you might need 'utf8x' for
%% extra symbols. Not all editors, especially on Windows, are UTF-8
%% capable, so you may want to use 'latin1' instead.
\usepackage[utf8]{inputenc}

%% This changes default fonts for both text and math mode to use Herman Zapfs
%% excellent Palatino font.  Do not change this.
\usepackage[sc]{mathpazo}

%% The AMS-LaTeX extensions for mathematical typesetting.  Do not
%% remove.
\usepackage{amsmath,amssymb,amsfonts,mathrsfs}

%% NTheorem is a reimplementation of the AMS Theorem package. This
%% will allow us to typeset theorems like examples, proofs and
%% similar.  Do not remove.
%% NOTE: Must be loaded AFTER amsmath, or the \qed placement will
%% break
\usepackage[amsmath,thmmarks]{ntheorem}

%% LaTeX' own graphics handling
\usepackage{graphicx}

%% We unfortunately need this for the Rules chapter.  Remove it
%% afterwards; or at least NEVER use its underlining features.
\usepackage{soul}

%% This allows you to add .pdf files. It is used to add the
%% declaration of originality.
\usepackage{pdfpages}

%% Some more packages that you may want to use.  Have a look at the
%% file, and consult the package docs for each.
%% See the TeXed file for more explanations

%% [OPT] Multi-rowed cells in tabulars
%\usepackage{multirow}

%% [REC] Intelligent cross reference package. This allows for nice
%% combined references that include the reference and a hint to where
%% to look for it.
\usepackage{varioref}

%% [OPT] Easily changeable quotes with \enquote{Text}
%\usepackage[german=swiss]{csquotes}

%% [REC] Format dates and time depending on locale
\usepackage{datetime}

%% [OPT] Provides a \cancel{} command to stroke through mathematics.
%\usepackage{cancel}

%% [NEED] This allows for additional typesetting tools in mathmode.
%% See its excellent documentation.
\usepackage{mathtools}

%% [ADV] Conditional commands
%\usepackage{ifthen}

%% [OPT] Manual large braces or other delimiters.
%\usepackage{bigdelim, bigstrut}

%% [REC] Alternate vector arrows. Use the command \vv{} to get scaled
%% vector arrows.
\usepackage[h]{esvect}

%% [NEED] Some extensions to tabulars and array environments.
\usepackage{array}

%% [OPT] Postscript support via pstricks graphics package. Very
%% diverse applications.
%\usepackage{pstricks,pst-all}

%% [?] This seems to allow us to define some additional counters.
%\usepackage{etex}

%% [ADV] XY-Pic to typeset some matrix-style graphics
%\usepackage[all]{xy}

%% [OPT] This is needed to generate an index at the end of the
%% document.
%\usepackage{makeidx}

%% [OPT] Fancy package for source code listings.  The template text
%% needs it for some LaTeX snippets; remove/adapt the \lstset when you
%% remove the template content.
\usepackage{listings}
\lstset{language=TeX,basicstyle={\normalfont\ttfamily}}

%% [REC] Fancy character protrusion.  Must be loaded after all fonts.
\usepackage[activate]{pdfcprot}

%% [REC] Nicer tables.  Read the excellent documentation.
\usepackage{booktabs}

%% Up greek letters
\usepackage{upgreek}


%% Our layout configuration.  DO NOT CHANGE.
%% Memoir layout setup

%% NOTE: You are strongly advised not to change any of them unless you
%% know what you are doing.  These settings strongly interact in the
%% final look of the document.

\usepackage{UNIBUClogo}

% Turn extra space before chapter headings off.
\setlength{\beforechapskip}{0pt}

\nonzeroparskip
\parindent=0pt
\defaultlists

% Chapter style redefinition
\makeatletter

\pagestyle{ruled}
\makeevenfoot{ruled}{}{}{\thepage}
\makeoddfoot{ruled}{}{}{\thepage}
\copypagestyle{chapter}{ruled}

\makeoddhead{chapter}{}{}{}
\makeevenhead{chapter}{}{}{}
\makeheadrule{chapter}{\textwidth}{0pt}
\copypagestyle{abstract}{empty}

\makechapterstyle{bianchimod}{%
  \chapterstyle{default}
  \renewcommand*{\chapnamefont}{\normalfont\Large}
  \renewcommand*{\chapnumfont}{\normalfont\Large}
  \renewcommand*{\printchaptername}{%
    \chapnamefont\centering\@chapapp}
  \renewcommand*{\printchapternum}{\chapnumfont {\thechapter}}
  \renewcommand*{\chaptitlefont}{\normalfont\huge}
  \renewcommand*{\printchaptertitle}[1]{%
    \hrule\vskip\onelineskip \centering \chaptitlefont\textbf{\vphantom{gyM}##1}\par}
  \renewcommand*{\afterchaptertitle}{\vskip\onelineskip \hrule\vskip
    \afterchapskip}
  \renewcommand*{\printchapternonum}{%
    \vphantom{\chapnumfont {9}}\afterchapternum}}

% Use the newly defined style
\chapterstyle{bianchimod}

\setsecheadstyle{\Large\bfseries}
\setsubsecheadstyle{\large\bfseries}
\setsubsubsecheadstyle{\bfseries}
\setparaheadstyle{\normalsize\bfseries}
\setsubparaheadstyle{\normalsize\itshape}
\setsubparaindent{0pt}

% Set captions to a more separated style for clearness
\captionnamefont{\bfseries\footnotesize}
\captiontitlefont{\footnotesize}
\setlength{\intextsep}{16pt}
\setlength{\belowcaptionskip}{1pt}

% Set section and TOC numbering depth to subsection
\setsecnumdepth{subsection}
\settocdepth{subsection}

%% Titlepage adjustments
\pretitle{\vspace{0pt plus 0.7fill}\begin{center}\HUGE\normalfont\bfseries}
\posttitle{\end{center}\par}
\preauthor{\par\begin{center}\let\and\\\Large\normalfont}
\postauthor{\end{center}}
\predate{\par\begin{center}\Large\normalfont}
\postdate{\end{center}}

\def\@advisors{}
\newcommand{\advisors}[1]{\def\@advisors{#1}}
\def\@department{}
\newcommand{\department}[1]{\def\@department{#1}}
\def\@thesistype{}
\newcommand{\thesistype}[1]{\def\@thesistype{#1}}

\renewcommand{\maketitlehooka}{\noindent\UNIBUClogo[2in]}

\renewcommand{\maketitlehookb}{\vspace{1in}%
  \par\begin{center}\Large\normalfont\@thesistype\end{center}}

\renewcommand{\maketitlehookd}{%
  \vfill\par
  \begin{flushright}
    \normalfont
    \@advisors\par
    \@department, UNIBUC 
  \end{flushright}
}

\checkandfixthelayout

\setlength{\droptitle}{-48pt}

\makeatother

% This defines how theorems should look. Best leave as is.
\theoremstyle{plain}
\setlength\theorempostskipamount{0pt}

%%% Local Variables:
%%% mode: latex
%%% TeX-master: "thesis"
%%% End:



%% Theorem environments.  You will have to adapt this for a German
%% thesis.
%% Theorem-like environments

%% This can be changed according to language. You can comment out the ones you
%% don't need.

\numberwithin{equation}{chapter}

%% German theorems
%\newtheorem{satz}{Satz}[chapter]
%\newtheorem{beispiel}[satz]{Beispiel}
%\newtheorem{bemerkung}[satz]{Bemerkung}
%\newtheorem{korrolar}[satz]{Korrolar}
%\newtheorem{definition}[satz]{Definition}
%\newtheorem{lemma}[satz]{Lemma}
%\newtheorem{proposition}[satz]{Proposition}

%% English variants
\newtheorem{theorem}{Teoremă}[chapter]
\newtheorem{example}[theorem]{Exemplu}
\newtheorem{remark}[theorem]{Propoziție}
\newtheorem{corollary}[theorem]{Corolar}
\newtheorem{definition}[theorem]{Definiție}
\newtheorem{lemma}[theorem]{Lemă}
\newtheorem{proposition}[theorem]{Propoziție}
\newtheorem{problem}[theorem]{Problemă}

%% Proof environment with a small square as a "qed" symbol
\theoremstyle{nonumberplain}
\theorembodyfont{\normalfont}
\theoremsymbol{\ensuremath{\square}}
\newtheorem{proof}{Proof}
%\newtheorem{beweis}{Beweis}



%% Helpful macros.
%% Custom commands
%% ===============

%% Special characters for number sets, e.g. real or complex numbers.
\newcommand{\C}{\mathbb{C}}
\newcommand{\K}{\mathbb{K}}
\newcommand{\N}{\mathbb{N}}
\newcommand{\Q}{\mathbb{Q}}
\newcommand{\R}{\mathbb{R}}
\newcommand{\Z}{\mathbb{Z}}
\newcommand{\X}{\mathbb{X}}

%% Fixed/scaling delimiter examples (see mathtools documentation)
\DeclarePairedDelimiter\abs{\lvert}{\rvert}
\DeclarePairedDelimiter\norm{\lVert}{\rVert}

%% Use the alternative epsilon per default and define the old one as \oldepsilon
\let\oldepsilon\epsilon
\renewcommand{\epsilon}{\ensuremath\varepsilon}

%% Also set the alternate phi as default.
\let\oldphi\phi
\renewcommand{\phi}{\ensuremath{\varphi}}



%% Make document internal hyperlinks wherever possible. (TOC, references)
%% This MUST be loaded after varioref, which is loaded in 'extrapackages'
%% above.  We just load it last to be safe.
\usepackage[linkcolor=black,colorlinks=true,citecolor=black,filecolor=black]{hyperref}


%% Document information
%% ====================

\title{Aspecte computaționale în producerea de cuvinte}
\author{Maria-Smaranda Pandele}
\thesistype{Teză de Licență}
\advisors{Coordonator: Prof.\ Dr.\ Liviu Dinu}
\department{Facultatea de Matematică și Informatică}
\date{22 Iunie, 2019}
\begin{document}

\frontmatter

%% Title page is autogenerated from document information above.  DO
%% NOT CHANGE.
\begin{titlingpage}
  \calccentering{\unitlength}
  \begin{adjustwidth*}{\unitlength-24pt}{-\unitlength-24pt}
    \maketitle
  \end{adjustwidth*}
\end{titlingpage}

%% The abstract of your thesis.  Edit the file as needed.
\begin{abstract}
  This example thesis briefly shows the main features of our thesis
  style, and how to use it for your purposes.
\end{abstract}



%% TOC with the proper setup, do not change.
\cleartorecto
\tableofcontents
\mainmatter

%% Your real content!
% Some commands used in this file
\newcommand{\package}{\emph}

\chapter{Introduction}

This is version \verb-v1.4- of the template.

We assume that you found this template on our institute's website, so
we do not repeat everything stated there.  Consult the website again
for pointers to further reading about \LaTeX{}.  This chapter only
gives a brief overview of the files you are looking at.

\section{Features}
\label{sec:features}

The rest of this document shows off a few features of the template
files.  Look at the source code to see which macros we used!

The template is divided into \TeX{} files as follows:
\begin{enumerate}
\item \texttt{thesis.tex} is the main file.
\item \texttt{extrapackages.tex} holds extra package includes.
\item \texttt{layoutsetup.tex} defines the style used in this document.
\item \texttt{theoremsetup.tex} declares the theorem-like environments.
\item \texttt{macrosetup.tex} defines extra macros that you may find
  useful.
\item \texttt{introduction.tex} contains this text.
\item \texttt{sections.tex} is a quick demo of each sectioning level
  available.
\item \texttt{refs.bib} is an example bibliography file.  You can use
  Bib\TeX{} to quote references.  For example, read
  \cite{bringhurst1996ets} if you can get a hold of it.
\end{enumerate}


\subsection{Extra package includes}

The file \texttt{extrapackages.tex} lists some packages that usually
come in handy.  Simply have a look at the source code.  We have
added the following comments based on our experiences:
\begin{description}
\item[REC] This package is recommended.
\item[OPT] This package is optional.  It usually solves a specific
  problem in a clever way.
\item[ADV] This package is for the advanced user, but solves a problem
  frequent enough that we mention it. Consult the package's
  documentation.
\end{description}

As a small example, here is a reference to the Section \emph{Features}
typeset with the recommended \package{varioref} package:
\begin{quote}
  See Section~\vref{sec:features}.
\end{quote}


\subsection{Layout setup}

This defines the overall look of the document -- for example, it
changes the chapter and section heading appearance.  We consider this
a `do not touch' area.  Take a look at the excellent \emph{Memoir}
documentation before changing it.

In fact, take a look at the excellent \emph{Memoir} documentation,
full stop.


\subsection{Theorem setup}

This file defines a bunch of theorem-like environments.

\begin{theorem}
  An example theorem.
\end{theorem}

\begin{proof}
  Proof text goes here.
\end{proof}

Note that the q.e.d.\ symbol moves to the correct place automatically
if you end the proof with an \texttt{enumerate} or
\texttt{displaymath}.  You do not need to use \verb-\qedhere- as with
\package{amsthm}.

\begin{theorem}[Some Famous Guy]
  Another example theorem.
\end{theorem}

\begin{proof}
  This proof
  \begin{enumerate}
  \item ends in an enumerate.
  \end{enumerate}
\end{proof}

\begin{proposition}
  Note that all theorem-like environments are by default numbered on
  the same counter.
\end{proposition}

\begin{proof}
  This proof ends in a display like so:
  \begin{displaymath}
    f(x) = x^2.
  \end{displaymath}
\end{proof}


\subsection{Macro setup}

For now the macro setup only shows how to define some basic macros,
and how to use a neat feature of the \package{mathtools} package:
\begin{displaymath}
  \abs{a}, \quad \abs*{\frac{a}{b}}, \quad \abs[\big]{\frac{a}{b}}.
\end{displaymath}


\chapter{Typography}

\theoremstyle{plain}
\theoremsymbol{}
\newtheorem{Rule}[theorem]{Rule}

\section{Punctuation}

\begin{Rule}
  Use opening (`) and closing (') quotation marks correctly.
\end{Rule}

In \LaTeX, the closing quotation mark is typed like a normal
apostrophe, while the opening quotation mark is typed using the French
\emph{accent grave} on your keyboard (the \emph{accent grave} is the
one going down, as in \emph{frère}).

Note that any punctuation that \emph{semantically} follows quoted
speech goes inside the quotes in American English, but outside in
Britain.  Also, Americans use double quotes first.  Oppose
\begin{quote}
  ``Using `lasers,' we punch a hole in \ldots\ the Ozone Layer,''
  Dr.\ Evil said.
\end{quote}
to
\begin{quote}
  `Using ``lasers'', we punch a hole in \ldots\ the Ozone Layer',
  Dr.\ Evil said.
\end{quote}

\begin{Rule}
  Use hyphens (-), en-dashes (--) and em-dashes (---) correctly.
\end{Rule}

A hyphen is only used in words like `well-known', `$3$-colorable'
etc., or to separate words that continue in the next line (which is
known as hyphenation).  It is entered as a single ASCII hyphen
character (\texttt{-}).

To denote ranges of numbers, chapters, etc., use an en-dash (entered
as two ASCII hyphens \texttt{--}) with no spaces on either side.  For
example, using Equations (1)--(3), we see\ldots

As the equivalent of the German \emph{Gedankenstrich}, use an en-dash
with spaces on both sides -- in the title of Section \ref{sec:list},
it would be wrong to use a hyphen instead of the dash. (Some English
authors use the even longer emdash (---) instead, which is typed as
three subsequent hyphens in \LaTeX. This emdash is used without spaces
around it---like so.)

\section{Spacing}

\begin{Rule}
  \label{rule:no-manual-spacing}
  Do not add spacing manually.
\end{Rule}

You should never use the commands \lstinline-\\- (except within
tabulars and arrays), \lstinline[showspaces=true]-\ - (except to
prevent a sentence-ending space after Dr.\ and such),
\lstinline-\vspace-, \lstinline-\hspace-, etc.  The choices programmed
into \LaTeX{} and this style should cover almost all cases.  Doing it
manually quickly leads to inconsistent spacing, which looks terrible.
Note that this list of commands is by no means conclusive.

\begin{Rule}
  Judiciously insert spacing in maths where it helps.
\end{Rule}

This directly contradicts Rule~\ref{rule:no-manual-spacing}, but in
some cases \TeX{} fails to correctly decide how much spacing is
required.  For example, consider
\begin{displaymath}
  f(a,b) = f(a+b, a-b).
\end{displaymath}
In such cases, inserting a thin math space \lstinline-\,- greatly
increases readability:
\begin{displaymath}
  f(a,b) = f(a+b,\, a-b).
\end{displaymath}

Along similar lines, there are variations of some symbols with
different spacing.  For example, Lagrange's Theorem states that
\(\abs{G}=[G:H]\abs{H}\), but the proof uses a bijection \(f\colon
aH\to bH\).  (Note how the first colon is symmetrically spaced, but
the second is not.)

\begin{Rule}
  Learn when to use \lstinline[showspaces=true]-\ - and
  \lstinline-\@-.
\end{Rule}

Unless you use `french spacing', the space at the end of a sentence is
slightly larger than the normal interword space.

The rule used by \TeX{} is that any space following a period,
exclamation mark or question mark is sentence-ending, except for
periods preceded by an upper-case letter.  Inserting \lstinline-\-
before a space turns it into an interword space, and inserting
\lstinline-\@- before a period makes it sentence-ending.  This means
you should write
\begin{lstlisting}
Prof.\ Dr.\ A. Steger is a member of CADMO\@.
If you want to write a thesis with her, you
should use this template.
\end{lstlisting}
which turns into
\begin{quote}
  Prof.\ Dr.\ A. Steger is a member of CADMO\@.  If you want to write
  a thesis with her, you should use this template.
\end{quote}
The effect becomes more dramatic in lines that are stretched slightly
during justification:
\begin{quote}
  \parbox{\linewidth}{\hbox to \linewidth{%
      Prof.\ Dr.\ A. Steger is a member of CADMO\@.  If you}}
\end{quote}

\begin{Rule}
  Place a non-breaking space (\lstinline-~-) right before references.
\end{Rule}

This is actually a slight simplification of the real rule, which
should invoke common sense.  Place non-breaking spaces where a line
break would look `funny' because it occurs right in the middle of a
construction, especially between a reference type (Chapter) and its
number.

\section{Choice of `fonts'}

Professional typography distinguishes many font attributes, such as
family, size, shape, and weight.  The choice for sectional divisions
and layout elements has been made, but you will still occasionally
want to switch to something else to get the reader's attention.  The
most important rule is very simple.

\begin{Rule}
  When emphasising a short bit of text, use \lstinline-\emph-.
\end{Rule}

In particular, \emph{never} use bold text (\lstinline-\textbf-).
Italics (or Roman type if used within italics) avoids distracting the
eye with the huge blobs of ink in the middle of the text that bold
text so quickly introduces.

Occasionally you will need more notation, for example, a consistent
typeface used to identify algorithms.

\begin{Rule}
  Vary one attribute at a time.
\end{Rule}

For example, for \textsc{WeirdSort} we only changed the shape to small
caps.  Changing two attributes, say, to bold small caps would be
excessive (\LaTeX{} does not even have this particular variation).
The same holds for mathematical notation: the reader can easily
distinguish \(g_n\), \(G(x)\), \(\mathcal{G}\) and \(\mathsf{G}\).

\begin{Rule}
  Never underline or uppercase.
\end{Rule}

No exceptions to this one, unless you are writing your thesis on a
typewriter.  Manually.  Uphill both ways.  In a blizzard.

\section{Displayed equations}

\begin{Rule}
  Insert paragraph breaks \emph{after} displays only where they
  belong.  Never insert paragraph breaks \emph{before} displays.
\end{Rule}

\LaTeX{} translates sequences of more than one linebreak (i.e., what
looks like an empty line in the source code) into a paragraph break in
almost all contexts.  This also happens before and after displays,
where extra spacing is inserted to give a visual indication of the
structure.  Adding a blank line in these places may look nice in the
sources, but compare the resulting display

\begin{displaymath}
  a = b
\end{displaymath}

to the following:
\begin{displaymath}
  a = b
\end{displaymath}
The first display is surrounded by blank lines, but the second is not.
It is bad style to start a paragraph with a display (you should always
tell the reader what the display means first), so the rule follows.

\begin{Rule}
  Never use \lstinline-eqnarray-.
\end{Rule}

It is at the root of most ill-spaced multiline displays.  The
\package{amsmath} package provides better alternatives, such as the
\lstinline-align- family
\begin{align*}
  f(x) &= \sin x, \\
  g(x) &= \cos x,
\end{align*}
and \lstinline-multline- which copes with excessively long equations:
\begin{multline*}
  \def\P{\mathrm P}
  \P\bigl[X_{t_0} \in (z_0, z_0+dz_0],\ldots, X_{t_n}\in(z_n,z_n+dz_n]\bigr]
  \\= \nu(dz_0) K_{t_1}(z_0,dz_1) K_{t_2-t_1}(z_1,dz_2)\cdots
  K_{t_n-t_{n-1}}(z_{n-1},dz_n).
\end{multline*}

\section{Floats}

By default this style provides floating environments for tables and
figures.  The general structure should be as follows:
\begin{lstlisting}
\begin{figure}
  \centering
  % content goes here
  \caption{A short caption}
  \label{some-short-label}
\end{figure}
\end{lstlisting}
Note that the label must follow the caption, otherwise the label will
refer to the surrounding section instead.  Also note that figures
should be captioned at the bottom, and tables at the top.

The whole point of floats is that they, well, \emph{float} to a place
where they fit without interrupting the text body.  This is a frequent
source of confusion and changes; please leave it as is.

\begin{Rule}
  Do not restrict float movement to only `here'
  \textnormal{(\lstinline-h-)}.
\end{Rule}

If you are still tempted, you should avoid the float altogether and
just show the figure or table inline, similar to a displayed equation.

%%% Local Variables:
%%% mode: latex
%%% TeX-master: "thesis"
%%% End:

\chapter{Example Chapter}

Dummy text.

\section{Example Section}

Dummy text.

\subsection{Example Subsection}

Dummy text.

\subsubsection{Example Subsubsection}

Dummy text.

\paragraph{Example Paragraph}

Dummy text.

\subparagraph{Example Subparagraph}

Dummy text.



\appendix

\chapter{Dummy Appendix}

You can defer lengthy calculations that would otherwise only interrupt
the flow of your thesis to an appendix.



\backmatter

\bibliographystyle{plain}
\bibliography{refs}

\end{document}

