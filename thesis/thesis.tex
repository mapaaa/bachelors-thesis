% Template version used: v1.4
%
% Largely adapted from Adrian Nievergelt's template for the ADPS
% (lecture notes) project.
% Adapted for a bachelor's thesis


%% We use the memoir class because it offers a many easy to use features.
\documentclass[11pt,a4paper,titlepage]{memoir}

%% Packages
%% ========

%% LaTeX Font encoding -- DO NOT CHANGE
\usepackage[OT1]{fontenc}

%% Babel provides support for languages.  'english' uses British
%% English hyphenation and text snippets like "Figure" and
%% "Theorem". Use the option 'ngerman' if your document is in German.
%% Use 'american' for American English.  Note that if you change this,
%% the next LaTeX run may show spurious errors.  Simply run it again.
%% If they persist, remove the .aux file and try again.
\usepackage[romanian]{babel}

%% Input encoding 'utf8'. In some cases you might need 'utf8x' for
%% extra symbols. Not all editors, especially on Windows, are UTF-8
%% capable, so you may want to use 'latin1' instead.
\usepackage[utf8]{inputenc}

%% This changes default fonts for both text and math mode to use Herman Zapfs
%% excellent Palatino font.  Do not change this.
\usepackage[sc]{mathpazo}

%% The AMS-LaTeX extensions for mathematical typesetting.  Do not
%% remove.
\usepackage{amsmath,amssymb,amsfonts,mathrsfs}

%% NTheorem is a reimplementation of the AMS Theorem package. This
%% will allow us to typeset theorems like examples, proofs and
%% similar.  Do not remove.
%% NOTE: Must be loaded AFTER amsmath, or the \qed placement will
%% break
\usepackage[amsmath,thmmarks]{ntheorem}

%% LaTeX' own graphics handling
\usepackage{graphicx}

%% We unfortunately need this for the Rules chapter.  Remove it
%% afterwards; or at least NEVER use its underlining features.
\usepackage{soul}

%% This allows you to add .pdf files. It is used to add the
%% declaration of originality.
\usepackage{pdfpages}

%% Some more packages that you may want to use.  Have a look at the
%% file, and consult the package docs for each.
%% See the TeXed file for more explanations

%% [OPT] Multi-rowed cells in tabulars
%\usepackage{multirow}

%% [REC] Intelligent cross reference package. This allows for nice
%% combined references that include the reference and a hint to where
%% to look for it.
\usepackage{varioref}

%% [OPT] Easily changeable quotes with \enquote{Text}
%\usepackage[german=swiss]{csquotes}

%% [REC] Format dates and time depending on locale
\usepackage{datetime}

%% [OPT] Provides a \cancel{} command to stroke through mathematics.
%\usepackage{cancel}

%% [NEED] This allows for additional typesetting tools in mathmode.
%% See its excellent documentation.
\usepackage{mathtools}

%% [ADV] Conditional commands
%\usepackage{ifthen}

%% [OPT] Manual large braces or other delimiters.
%\usepackage{bigdelim, bigstrut}

%% [REC] Alternate vector arrows. Use the command \vv{} to get scaled
%% vector arrows.
\usepackage[h]{esvect}

%% [NEED] Some extensions to tabulars and array environments.
\usepackage{array}

%% [OPT] Postscript support via pstricks graphics package. Very
%% diverse applications.
%\usepackage{pstricks,pst-all}

%% [?] This seems to allow us to define some additional counters.
%\usepackage{etex}

%% [ADV] XY-Pic to typeset some matrix-style graphics
%\usepackage[all]{xy}

%% [OPT] This is needed to generate an index at the end of the
%% document.
%\usepackage{makeidx}

%% [OPT] Fancy package for source code listings.  The template text
%% needs it for some LaTeX snippets; remove/adapt the \lstset when you
%% remove the template content.
\usepackage{listings}
\lstset{language=TeX,basicstyle={\normalfont\ttfamily}}

%% [REC] Fancy character protrusion.  Must be loaded after all fonts.
\usepackage[activate]{pdfcprot}

%% [REC] Nicer tables.  Read the excellent documentation.
\usepackage{booktabs}

%% Up greek letters
\usepackage{upgreek}


%% Our layout configuration.  DO NOT CHANGE.
%% Memoir layout setup

%% NOTE: You are strongly advised not to change any of them unless you
%% know what you are doing.  These settings strongly interact in the
%% final look of the document.

\usepackage{UNIBUClogo}

% Turn extra space before chapter headings off.
\setlength{\beforechapskip}{0pt}

\nonzeroparskip
\parindent=0pt
\defaultlists

% Chapter style redefinition
\makeatletter

\pagestyle{ruled}
\makeevenfoot{ruled}{}{}{\thepage}
\makeoddfoot{ruled}{}{}{\thepage}
\copypagestyle{chapter}{ruled}

\makeoddhead{chapter}{}{}{}
\makeevenhead{chapter}{}{}{}
\makeheadrule{chapter}{\textwidth}{0pt}
\copypagestyle{abstract}{empty}

\makechapterstyle{bianchimod}{%
  \chapterstyle{default}
  \renewcommand*{\chapnamefont}{\normalfont\Large}
  \renewcommand*{\chapnumfont}{\normalfont\Large}
  \renewcommand*{\printchaptername}{%
    \chapnamefont\centering\@chapapp}
  \renewcommand*{\printchapternum}{\chapnumfont {\thechapter}}
  \renewcommand*{\chaptitlefont}{\normalfont\huge}
  \renewcommand*{\printchaptertitle}[1]{%
    \hrule\vskip\onelineskip \centering \chaptitlefont\textbf{\vphantom{gyM}##1}\par}
  \renewcommand*{\afterchaptertitle}{\vskip\onelineskip \hrule\vskip
    \afterchapskip}
  \renewcommand*{\printchapternonum}{%
    \vphantom{\chapnumfont {9}}\afterchapternum}}

% Use the newly defined style
\chapterstyle{bianchimod}

\setsecheadstyle{\Large\bfseries}
\setsubsecheadstyle{\large\bfseries}
\setsubsubsecheadstyle{\bfseries}
\setparaheadstyle{\normalsize\bfseries}
\setsubparaheadstyle{\normalsize\itshape}
\setsubparaindent{0pt}

% Set captions to a more separated style for clearness
\captionnamefont{\bfseries\footnotesize}
\captiontitlefont{\footnotesize}
\setlength{\intextsep}{16pt}
\setlength{\belowcaptionskip}{1pt}

% Set section and TOC numbering depth to subsection
\setsecnumdepth{subsection}
\settocdepth{subsection}

%% Titlepage adjustments
\pretitle{\vspace{0pt plus 0.7fill}\begin{center}\HUGE\normalfont\bfseries}
\posttitle{\end{center}\par}
\preauthor{\par\begin{center}\let\and\\\Large\normalfont}
\postauthor{\end{center}}
\predate{\par\begin{center}\Large\normalfont}
\postdate{\end{center}}

\def\@advisors{}
\newcommand{\advisors}[1]{\def\@advisors{#1}}
\def\@department{}
\newcommand{\department}[1]{\def\@department{#1}}
\def\@thesistype{}
\newcommand{\thesistype}[1]{\def\@thesistype{#1}}

\renewcommand{\maketitlehooka}{\noindent\UNIBUClogo[2in]}

\renewcommand{\maketitlehookb}{\vspace{1in}%
  \par\begin{center}\Large\normalfont\@thesistype\end{center}}

\renewcommand{\maketitlehookd}{%
  \vfill\par
  \begin{flushright}
    \normalfont
    \@advisors\par
    \@department, UNIBUC 
  \end{flushright}
}

\checkandfixthelayout

\setlength{\droptitle}{-48pt}

\makeatother

% This defines how theorems should look. Best leave as is.
\theoremstyle{plain}
\setlength\theorempostskipamount{0pt}

%%% Local Variables:
%%% mode: latex
%%% TeX-master: "thesis"
%%% End:



%% Theorem environments.  You will have to adapt this for a German
%% thesis.
%% Theorem-like environments

%% This can be changed according to language. You can comment out the ones you
%% don't need.

\numberwithin{equation}{chapter}

%% German theorems
%\newtheorem{satz}{Satz}[chapter]
%\newtheorem{beispiel}[satz]{Beispiel}
%\newtheorem{bemerkung}[satz]{Bemerkung}
%\newtheorem{korrolar}[satz]{Korrolar}
%\newtheorem{definition}[satz]{Definition}
%\newtheorem{lemma}[satz]{Lemma}
%\newtheorem{proposition}[satz]{Proposition}

%% English variants
\newtheorem{theorem}{Teoremă}[chapter]
\newtheorem{example}[theorem]{Exemplu}
\newtheorem{remark}[theorem]{Propoziție}
\newtheorem{corollary}[theorem]{Corolar}
\newtheorem{definition}[theorem]{Definiție}
\newtheorem{lemma}[theorem]{Lemă}
\newtheorem{proposition}[theorem]{Propoziție}
\newtheorem{problem}[theorem]{Problemă}

%% Proof environment with a small square as a "qed" symbol
\theoremstyle{nonumberplain}
\theorembodyfont{\normalfont}
\theoremsymbol{\ensuremath{\square}}
\newtheorem{proof}{Proof}
%\newtheorem{beweis}{Beweis}



%% Helpful macros.
%% Custom commands
%% ===============

%% Special characters for number sets, e.g. real or complex numbers.
\newcommand{\C}{\mathbb{C}}
\newcommand{\K}{\mathbb{K}}
\newcommand{\N}{\mathbb{N}}
\newcommand{\Q}{\mathbb{Q}}
\newcommand{\R}{\mathbb{R}}
\newcommand{\Z}{\mathbb{Z}}
\newcommand{\X}{\mathbb{X}}

%% Fixed/scaling delimiter examples (see mathtools documentation)
\DeclarePairedDelimiter\abs{\lvert}{\rvert}
\DeclarePairedDelimiter\norm{\lVert}{\rVert}

%% Use the alternative epsilon per default and define the old one as \oldepsilon
\let\oldepsilon\epsilon
\renewcommand{\epsilon}{\ensuremath\varepsilon}

%% Also set the alternate phi as default.
\let\oldphi\phi
\renewcommand{\phi}{\ensuremath{\varphi}}



%% Make document internal hyperlinks wherever possible. (TOC, references)
%% This MUST be loaded after varioref, which is loaded in 'extrapackages'
%% above.  We just load it last to be safe.
\usepackage[linkcolor=black,colorlinks=true,citecolor=black,filecolor=black]{hyperref}


%% Document information
%% ====================

\title{Aspecte computaționale în producerea de cuvinte}
\author{Maria-Smaranda Pandele}
\thesistype{Teză de Licență}
\advisors{Coordonator: Prof.\ Dr.\ Liviu Dinu}
\department{Facultatea de Matematică și Informatică}
\date{22 Iunie, 2019}
\begin{document}

\frontmatter

%% Title page is autogenerated from document information above.  DO
%% NOT CHANGE.
\begin{titlingpage}
  \calccentering{\unitlength}
  \begin{adjustwidth*}{\unitlength-24pt}{-\unitlength-24pt}
    \maketitle
  \end{adjustwidth*}
\end{titlingpage}

%% The abstract of your thesis.  Edit the file as needed.
\begin{abstract}
  This example thesis briefly shows the main features of our thesis
  style, and how to use it for your purposes.
\end{abstract}



%% TOC with the proper setup, do not change.
\cleartorecto
\tableofcontents
\mainmatter

%% Your real content!
% Some commands used in this file
\newcommand{\package}{\emph}

\chapter{Introduction}

This is version \verb-v1.4- of the template.

We assume that you found this template on our institute's website, so
we do not repeat everything stated there.  Consult the website again
for pointers to further reading about \LaTeX{}.  This chapter only
gives a brief overview of the files you are looking at.

\section{Features}
\label{sec:features}

The rest of this document shows off a few features of the template
files.  Look at the source code to see which macros we used!

The template is divided into \TeX{} files as follows:
\begin{enumerate}
\item \texttt{thesis.tex} is the main file.
\item \texttt{extrapackages.tex} holds extra package includes.
\item \texttt{layoutsetup.tex} defines the style used in this document.
\item \texttt{theoremsetup.tex} declares the theorem-like environments.
\item \texttt{macrosetup.tex} defines extra macros that you may find
  useful.
\item \texttt{introduction.tex} contains this text.
\item \texttt{sections.tex} is a quick demo of each sectioning level
  available.
\item \texttt{refs.bib} is an example bibliography file.  You can use
  Bib\TeX{} to quote references.  For example, read
  \cite{bringhurst1996ets} if you can get a hold of it.
\end{enumerate}


\subsection{Extra package includes}

The file \texttt{extrapackages.tex} lists some packages that usually
come in handy.  Simply have a look at the source code.  We have
added the following comments based on our experiences:
\begin{description}
\item[REC] This package is recommended.
\item[OPT] This package is optional.  It usually solves a specific
  problem in a clever way.
\item[ADV] This package is for the advanced user, but solves a problem
  frequent enough that we mention it. Consult the package's
  documentation.
\end{description}

As a small example, here is a reference to the Section \emph{Features}
typeset with the recommended \package{varioref} package:
\begin{quote}
  See Section~\vref{sec:features}.
\end{quote}


\subsection{Layout setup}

This defines the overall look of the document -- for example, it
changes the chapter and section heading appearance.  We consider this
a `do not touch' area.  Take a look at the excellent \emph{Memoir}
documentation before changing it.

In fact, take a look at the excellent \emph{Memoir} documentation,
full stop.


\subsection{Theorem setup}

This file defines a bunch of theorem-like environments.

\begin{theorem}
  An example theorem.
\end{theorem}

\begin{proof}
  Proof text goes here.
\end{proof}

Note that the q.e.d.\ symbol moves to the correct place automatically
if you end the proof with an \texttt{enumerate} or
\texttt{displaymath}.  You do not need to use \verb-\qedhere- as with
\package{amsthm}.

\begin{theorem}[Some Famous Guy]
  Another example theorem.
\end{theorem}

\begin{proof}
  This proof
  \begin{enumerate}
  \item ends in an enumerate.
  \end{enumerate}
\end{proof}

\begin{proposition}
  Note that all theorem-like environments are by default numbered on
  the same counter.
\end{proposition}

\begin{proof}
  This proof ends in a display like so:
  \begin{displaymath}
    f(x) = x^2.
  \end{displaymath}
\end{proof}


\subsection{Macro setup}

For now the macro setup only shows how to define some basic macros,
and how to use a neat feature of the \package{mathtools} package:
\begin{displaymath}
  \abs{a}, \quad \abs*{\frac{a}{b}}, \quad \abs[\big]{\frac{a}{b}}.
\end{displaymath}


\chapter{Agregarea rezultatelor folosind distanța rank}
\label{chap:three}
Am văzut cum putem obține producții de cuvinte combinând câte o singură limbă romanică cu 
limba latină. Pentru a îmbunătății rezultatele vrem să folosim informația din mai multe 
limbi romanice moderne. Astfel, fiecare clasificator întoarce o listă ordonată de cuvinte latinești, 
pe prima poziție aflându-se etimonul latinesc cu cea mai mare probabilitate. Prin agregarea 
acestora cu o anumită metrică vom obține o lista sortata cu mai probabile cuvinte latinești. Metrica
folosita este distanta rank \cite{rankdistance} întrucât s-au obținut rezultate bune în alte 
probleme de natură lingvistică precum determinarea similitudinii silabice a limbilor romanice 
\cite{syllabicsim}, \cite{simnat}.

În primul rând vom defini ce înseamna o listă ordonată de elemente. În al doilea rând, vom explica
distanța rank între două clasamente și între un clasament și o mulțime de clasamente. Apoi vom
prezenta o metodă de aflare a tuturor agregărilor unei mulțimi de mai multe clasamente folosind
distanța rank. În final, vom prelucra mulțimea de agregări bazat pe un sistem de vot pentru a 
determina \textbf{o singură listă ordonată} de posibile etimoane latinești.

\section{Clasamente și distanța rank}
Un \textit{clasament} este o listă ordonată de obiecte după un anumit criteriu, pe prima poziție 
aflându-se cel cu cea mai mare importanță. În unele situații se pune problema găsirii unui clasament
cât mai apropiat de o mulțime de mai multe clasamente. Pentru a rezolva această problemă trebuie să
definim mai întâi ce înseamnă distanța dintre două clasamente sau dintre un singur clasament și o
mulțime de clasamente.

Există mai multe metrici folosite cu succes în diverse aplicații: distanța \textit{Kedall tau}, 
\textit{Spearman footrule}, \textit{Levenshtein}, dar noi vom folosi distanța \textit{rank}
introdusa in articolul \cite{rankdistance}. În întregul capitol vom folosi următoarele notații:
\begin{itemize}
    \item $U = \{1, 2, ..., n\}$ o mulțime finită de obiecte numită univers 
    \item $\tau = (x_1 > x_2 > ... > x_d)$ un clasament peste universul $U$ 
    \item $>$ o relație de ordine strictă reprezentând criteriul de ordonare 
    \item $\tau(x)$ = poziția elementului $x \in U$ în clasamentul $\tau$ dacă $x \in \tau$, 
      numerotând pozițiile de la 1 începând cu cel mai important obiect din clasament
\end{itemize}

Dacă un clasament conține toate elementele din univers, atunci el se va numi 
\textit{clasament total}. Asemănător, dacă conține doar o submulțime de obiecte din univers, atunci
îl vom numi \textit{clasament parțial}.

Notăm ordinea elementului $x$ în $\tau$ astfel:
\begin{gather}
\label{ord}
  ord(\tau, x)= \begin{dcases}
    |n + 1 - \tau(x)|    &, x \in \tau \\
    0                    &, x \in U \setminus \tau
  \end{dcases}
\end{gather}

\begin{definition}
Fie $\tau$ și $\sigma$ două clasamente parțiale peste același univers $U$. Atunci distanța rank va fi
\begin{gather}
  \Updelta(\tau, \sigma) = \smashoperator{\sum_{x \in \tau \cup \sigma}} |ord(\tau, x) - ord(\sigma, x)|
\end{gather}
\end{definition}

Se observă faptul că, în calculul distanței rank, se ia în considerare ordinea definită mai sus
și nu poziția. În primul rând, cum primele poziții sunt cele mai importante, distanța dintre două
clasamente trebuie sa fie cu atât mai mare cu cât diferă mai mult începutul lor.\cite{linguisticstructuresmarcus}
În al doilea rând, definiția funcției $ord$ pune accentul pe lungimea clasamentelor întrucăt putem
presupune că un clasament mai lung a fost obținut în urma comparării mai multor obiecte din univers.
Deci ordinea elementelor este mai solidă. Spre exemplu, dacă două clasamente de lungimi diferite au
același element pe prima poziție, există totuși o diferență a ordinii obiectului în cele două liste,
diferență ce contribuie la calculul distanței rank totale.\cite{rankaggregationproblem}

\subsection{Agregări cu distanța rank}
O \textit{agregare de clasamente} reprezintă un singur clasament $\sigma$ astfel încât o anumită 
metrică de la acesta la mulțimea de liste de agregat $T$ este minimă. Raportându-ne la distanța
rank avem\cite{rankdistance}:

\begin{definition}
Fie un set de clasamente $T = \{\tau_1, \tau_2, ..., \tau_m\}$ dintr-un univers $U$ și
$\sigma = (\sigma_1 > \sigma_2 > ... > \sigma_k)$ un clasament astfel încât $\sigma_i \in U, 
\forall 1 \leqslant i \leqslant k$. Definim distanța rank de la $\sigma$ la $T$ astfel:
\begin{gather}
  \Updelta(\sigma, T) = \smashoperator{\sum_{i = 1}^{m}} \Updelta(\sigma, \tau_i)
\end{gather}
\end{definition}

\begin{definition}
\label{def:Aset}
Se numește mulțime de agregari de lungime $k$ a mulțimii $T$ folosind distanța rank, setul
$
  A(T, k) = \{\sigma=(\sigma_1 > \sigma_2 > ... > \sigma_k) | \sigma_i \in U, 
  \forall 1 \leqslant i \leqslant k$, si 
  $\Updelta(\sigma, T)$ este minim posibila \}
\end{definition}


\begin{problem}
Fie $U$ un set de obiecte și $T = \{\tau_1, \tau_2, ..., \tau_m\}$ o mulțime de clasamente peste
universul $U$. Vrem să determinăm mulțimea de agregări $A(T, k)$ pentru un k fixat. 
\end{problem}

Construim următoarele matrici bidimensionale $W^k(i, j)$ cu $n$ linii și $n$ coloane. Fiecare celulă
din fiecare matrice reprezintă costul total din distanța rank de la un clasament $\sigma$, de 
lungime $l$, către o mulțime $T = \{\tau_1, \tau_2, ..., \tau_m\}$ fixată indus de plasarea 
elementului $x_i \in U$ pe poziția $j$ în $\sigma$ \cite{rankaggregationproblem}. Se observă faptul
că un clasament peste universul $U$ definit mai sus poate avea lungimea maxim $n$. Rezultă că 
numărul de coloane al matricilor $W^t$ este egal cu $n$.
\begin{gather}
  \label{eq:wmatrix}
  W^k(i, j) = \begin{dcases}
    \smashoperator{\sum_{p=1}^{m}} | ord(p, i) - k + j |    &, j \leqslant k \\
    \smashoperator{\sum_{p=1}^{m}} | ord(p, i) |            &, j > k
  \end{dcases}
\end{gather}

\begin{remark}
Distanța de la un clasament $\sigma=(\sigma_1 > \sigma_2 > ... \sigma_k)$ la mulțimea $T$ este
\[
  \Updelta(\sigma, T) = \smashoperator{\sum_{x_i \in U \cap \sigma}} W^k(i, \sigma(x_i)) +
      \smashoperator{\sum_{x_i \in U \setminus \sigma}} W^k(i, k + 1)
\]unde $n$ reprezintă numărul de obiecte din univers, iar $k < n$.
\end{remark}
Se observă faptul că, în cazul în care $\sigma$ conține toate elementele din $U$, deci cazul $k = n$
, formula se devine
\begin{gather}
  \label{eq:minimize}
  \Updelta(\sigma, T) = \smashoperator{\sum_{x_i \in U \cap \sigma}} W^k(i, \sigma(x_i))
\end{gather}

\subsection{Reducerea la o problemă de cuplaj perfect de cost minim}
Fiecare matrice $W^l$ din secțiunea precedentă este calculată în mod independent de celelalte
Deci putem determina doar o singură matrice pentru o anumită lungime fixată $l$. 
Astfel, problema se reduce la găsirea unui clasament $\sigma$ ce 
minimizează formula \eqref{eq:minimize}. Formal:

\begin{problem}
\label{def:problem}
Fiind dată o matrice pătratică $W$, $W = (w_{i, j})_{1 \leqslant i,j \leqslant n}$ vrem să
determinăm următoarea mulțime:
\[
  S = \{(i_1, i_2, ..., i_k) | (i_p \neq i_j, \forall p \neq j), (i_j \in U) \text{ și } \smashoperator{\sum_{j=1}^n} w_{i_j, j} \text{ este minim}\}
\]
\end{problem}

Problema de mai sus se aseamănă cu o problemă de cuplaj de dimensiune $k$ de cost minim întrucât 
vrem să formăm perechi între obiectele dintr-un univers și pozițiile unui clasament de tip agregare, 
iar fiecare combinație are un anumit cost. Practic $(i_1, i_2, ..., i_n)$ reprezintă o permutare a 
elementelor din $U$.

O soluție pentru a rezolva problema precedentă este aplicarea algoritmul Ungar prezentat de Khun \cite{hungarianmethod}.
Altfel, putem considera matricea $W$ ca fiind o matrice de costuri într-un graf bipartit $G$ pe care
aplicăm un algoritm clasic de găsire a cuplajului maxim de cost minim (din care luam doar $k$ perechi). 
Conform \cite{flowassignment} această problemă poate fi rezolvată în timp polinomial 
$\mathcal{O}(n^3)$ construind o rețea de flux cu capacități convenabile și prin găsirea unor drumuri 
de augmentare minime, din punct de vedere al costului, folosind algoritmul lui Dijkstra\cite{dijkstra}.

Toate aceste rezolvări determină o singură agregare dar nu și pe toate, adică mulțimea $A(T, k)$ din
definiția \ref{def:Aset}. În continuare prezentăm o metoda de determinare a tuturor agregărilor 
bazată pe găsirea tuturor cuplajelor \textit{perfecte} de cost minim dintr-un graf, metoda 
prezentată în \cite{allmatchings}. Algoritmul rulează într-un timp polinomial. Particularizăm 
problemele și algoritmii din articolul  \textit{A generalization of the assignment problem, and its
application to the rank aggregation  problem} \cite{allmatchings} pentru Problema \ref{def:problem}.


\subsection{Calcularea tuturor agregărilor optime}
Reamintim faptul că dorim să calculăm mulțimea de agregări $A(T, k)$, știind costul plasării 
fiecărui element pe fiecare poziție, memorat în matricea $W^k$ calculată la \eqref{eq:wmatrix}. 
Reformulăm problema în elemente de teoria grafurilor. Astfel, asociem Problemei \ref{def:problem} 
un graf $G = (V, E, c, w)$, unde $V$ reprezintă mulțimea de noduri, $E$ este mulțimea de muchii iar
$c \colon E \to \mathbb{N}$ și $w \colon E \to \mathbb{N}$ reprezintă capacitatea unei muchii
respectiv costul acesteia. Legăturile între Problema \ref{def:problem} și graful $G$ sunt:
\begin{itemize}
  \item $V = \{src, dst\} \cup U \cup \{1, 2,..., k, k+1\}$
  \item $E = \{(src, x_i) | x_i \in U\} \cup \{(x_i, j) | x_i \in U \text{ si } j = 1,...,k\} \cup 
    \{(j, dst) | j = 1,...,k\}$  
  \item $c(muchie) = 1, \forall muchie = (x, j) \in E, j \neq k+1$
  \item $c(muchie) = \infty, \forall muchie = (x, k+1) \in E$
  \item funcția $w$ astfel:
  \begin{itemize}
    \item $w((src, x_i)) = 0, \forall x_i \in U$
    \item $w((x_i, j)) = W^k(i, j), \forall x_i \in U, j = 1,...,k+1$
    \item $w((j, dst)) = 0, j = 1,...,k+1$
  \end{itemize}
\end{itemize}

Potrivirea unui element $x$ cu poziția $k+1$ va semnifica excluderea acestuia din agregare ce 
afectează distanța rank dintre un clasament ce nu conține elementul $x$ și mulțimea întreagă $T$.
Se poate calcula ușor în acest graf un flux maxim de cost minim folosind algoritmi clasici\cite{flowassignment}).
Notăm prin $solve(W)$ un asemenea algoritm.
Fie soluția $M = \{(x, j) | x \in U \text{ și } j = 1,...,k\}$. Următorul pas este aflarea unei
soluții $M'$ diferite de $M$.

\begin{proposition}
Doua cuplaje $M$ și $M'$ sunt diferite dacă există cel puțin o pereche $(x, y)$ care se află în $M$
și nu se află în $M'$.
\end{proposition}

Astfel, propunem următorul algoritm, adaptat din \cite{allmatchings}, prin care căutăm o a doua
soluție $M'$ fixând câte o muchie candidat $(x, y)$ prin care $M'$ sa difere de $M$. Setând costul
muchiei $(x, y)$ pe o valoare infinită, avem garanția ca aceasta nu va fi luată în considerare în 
construcția lui $M'$.
\begin{algorithm}
\label{P}
\caption{anotherSolution}
\begin{algorithmic}[1]
\REQUIRE $W, M$
\ENSURE $M'$
  \STATE $s \gets \sum_{(u, v) \in M} w_{uv}$
  \FORALL{$(x, y) \in M$}
    \STATE $temp \gets w_{xy}$
    \STATE $w_{xy} \gets \infty$
    \STATE $M' \gets solve(W)$
    \IF{$M' \neq \emptyset \text{ si } \sum_{(u, v) \in M'} w_{uv} = s$}
    \RETURN $M'$
    \ELSE
    \STATE $w_{xy} \gets temp$
    \ENDIF
  \ENDFOR
  \RETURN $\emptyset$
\end{algorithmic}
\end{algorithm}

Algoritmul returnează fie mulțimea vidă, fie o soluție $M'$ astfel încât exista o pereche $(x, y)
\in M \setminus M'$. Se poate împărți problema inițială în două subprobleme disjuncte $P_1$ și
$P_2$:
\begin{itemize}
  \item $P_1\colon$ mulțimea tuturor cuplajelor ce conțin muchia $(x, y)$ \\
    În acest caz forțăm păstrarea perechii în soluție prin setarea tuturor celorlalte valori de pe
    linia $x$, coloana $y$ pe o valoare infinita în matricea $W$:
    $w_{iy} = w_{xj}, \forall i = 1,..,n \text{, }i \neq x  \text{ și } j = 1,...,n \text{, }j \neq y$
  \item $P_2\colon$ mulțimea tuturor cuplajelor ce \textbf{nu} conțin muchia $(x, y)$ \\
    În acest caz perechea $(x, y)$ nu va mai fi niciodată aleasă într-o soluție dacă costul acesteia
    este infinit:
    $w_{xy} = \infty$
\end{itemize}

Evident, există deja cate o soluție calculată pentru cele 2 subprobleme și anume $M \in P_1$ și 
$M' \in P_2$. Prim urmare, se poate aplica Algoritmul 1 în mod recursiv pentru fiecare dintre 
aceste subprobleme pentru a determina întreaga mulțime de soluții. Această abordare conduce la
construirea unei structuri de căutare arborescente în care rădăcina reprezintă problema inițială
\ref{def:problem}, iar fiecare nod intern constituie o împărțire pe subprobleme după o pereche 
$(x, y)$. Soluția finală se construiește traversând arborele în adâncime și păstrând toate soluțiile
parțiale calculate la fiecare pas. Nu se va genera aceeași soluție de mai multe ori prin faptul că
problemele $P_1$ și $P_2$ sunt complet disjuncte.

\begin{algorithm}
\caption{dfsAgregare}
\begin{algorithmic}[1]
\REQUIRE $S, M, W$
  \STATE $s \gets \sum_{(u, v) \in M} w_{uv}$
  \STATE $M' \gets another_solution(W, M)$
  \IF{$M' \neq \emptyset$}
    \RETURN
  \ELSE
    \STATE $S \gets M'$
    \STATE $(x, y) \in M \setminus M'$
    \STATE $w_{xy} \gets \infty$
    \STATE $dfsAgregare(S, M', W)$
    \STATE $w_{iy} \gets w_{xj}, \forall i = 1,..,n \text{, }i \neq x  \text{ si } j = 1,...,n \text{, }j \neq y$
    \STATE $dfsAgregare(S, M' \setminus (x, y), W)$
  \ENDIF
\end{algorithmic}
\end{algorithm}

\begin{algorithm}
\caption{Calculează toate cuplajele perfecte de cost minim}
\begin{algorithmic}[1]
\REQUIRE $W$
\ENSURE $S$
  \STATE $S \gets \emptyset$
  \STATE $M \gets solve(W)$
  \STATE $S \gets S \cup M$
  \STATE $dfsAgregare(S, M, W)$
  \RETURN $S$
\end{algorithmic}
\end{algorithm}

Algoritmul 3 determină toate cuplajele perfecte de cost minim pornind de la o matrice de costuri
$W$. Calculând matricea $W$ după formula \eqref{eq:wmatrix}, atunci mulțimea $S$ determinată de
algoritm este chiar soluția căutată în Problema \ref{def:problem}. Câteva aspecte legate de
complexitatea metodei prezentate: \\
Fie $x = |S|$, numărul total de soluții pentru o anumita problema.
\begin{itemize}
  \item Pentru fiecare soluție nou calculată, se construiesc două noi alte probleme. În total se vor
    rezolva maxim $2*x+1$ subprobleme.
  \item O subproblemă necesită găsirea unui cuplaj de cost minim ce se poate calcula într-un timp
    polinomial folosind metoda Ungară\cite{hungarianmethod} ori un algoritm clasic de determinare
    a fluxului maxim de cost minim într-un graf bipartit\cite{flowassignment}.
\end{itemize}
Intuitiv, putem afirma ca Algoritmul 3 rulează într-un timp polinomial. Complexitatea acestuia a
fost demonstrată în \cite{allmatchings} ca fiind $\mathcal{O}((2x+1)k^3n\log(n+k))$.

Subproblemele sunt complet independente de complementarele lor. Prin urmare, rezolvările acestora se
pot rula în paralel pe mai multe fire de execuție. În experimentele rulate am ales să pornim
un nou fir de execuție pentru fiecare subproblemă de tipul $P_2$ până la un anumit nivel din
arborele de căutare determinat în funcție de procesorul folosit. Subproblema de tipul $P_1$
a rămas să ruleze pe firul de execuție principal.


\section{Determinarea tuturor agregărilor producțiilor de cuvinte}
În capitolul precedent am prezentat o metoda de a combina o limbă romanică modernă și limba latină
pentru a automatiza procesul de determinare a etimonului latinesc. Metoda returna primele $n$
cuvinte posibile ordonate de la cel cu probabilitatea cea mai mare la cel cu probabilitatea cea mai
mică. Vom considera aceste liste de cuvinte ca fiind clasamente. Pentru fiecare cuvânt latinesc vom
agrega clasamentele produse din fiecare limbă romanică modernă \textit{(ro, it, fr, es, pt)}.
Se observă faptul că pot exista mai multe astfel de agregări așa că ne propunem să le aflăm pe toate
într-un mod eficient din punct de vedere al complexității timp. Alegem să luăm în considerare
doar primele 5 cele mai bune cuvinte din fiecare set. Astfel, pentru un singur cuvânt latinesc, vom 
avea:
\begin{gather*}
  R = \{r_1, r_2, ..., r_k\}, \text{clasamentele produse din ro, it, fr, es, pt} \\
  k = |R|, 1 \leqslant k \leqslant 5 \\
  U = \smashoperator{\bigcup_{i = 1}^{k}} r_i \text{ universul de cuvinte} \\
  n = |U|
\end{gather*}
Definim o matrice bidimensională de $k$ linii și $n$ coloane în care calculăm ordinea fiecărui 
cuvânt din universul unui cuvânt în fiecare clasament dat:
\[
  ord[i][j] = \begin{dcases}
    |6 - r_i(x_j)|    &, x_j \in r_i \\
    0                 &, x_j \in U \setminus r_i 
  \end{dcases}
\]

Calculăm apoi matricea de costuri $W^5$ conform formulei de la \ref{eq:wmatrix} pentru a afla cum
este afectat costul final dacă un anumit cuvânt $x_i \in U$ este plasat pe poziția $j$. Aceasta 
poziție poate sa fie mai mare decât $5$ dar conform formulei construite, nu se va face nicio 
distincție între a plasa un cuvânt pe poziția $6$ spre exemplu, sau poziția $7$, considerându-se
că respectivul element nu va aparține în agregarea finala de lungime $5$.

Este limpede ca se poate aplica acum Algoritmul 3 prezentat în secțiunea precedentă pentru a afla
mulțimea $S = \{\sigma_1, \sigma_2, ..., \sigma_t\}$ de agregări posibile. Propunem combinarea 
soluțiilor din $S$ printr-un sistem de vot. Fiecare cuvânt va primi un scor bazat pe pozițiile pe 
care se află în clasamentele din $S$.

\begin{gather*}
  scor \colon U \to \mathbb{N} \\
  scor(x_i) = \smashoperator{\sum_{j=1}^{|S|}} \sigma_j(x_i), \forall x_i \in U
\end{gather*}

Putem în sfârșit să construim lista finală, introdusă vag la începutul capitolului, de cuvinte 
ordonate în funcție de probabilitatea de a fi etimonul latinesc al unor anumite cuvinte moderne 
din \textit{ro, it, fr, es, pt}. Pe prima poziție se va afla cuvântul produs cu scorul cel mai mic,
adică cel ce se află pe primele poziții în clasamentele $S$, pe a doua poziție, cel cu următorul cel
mai mic scor, și asa mai departe. În cazul în care există mai multe cuvinte cu același scor, le 
vom păstra pe toate pe aceeași poziție și le vom filtra manual folosind reguli lingvistice.

\begin{gather*}
  F = (x_1 > x_2 > x_3 > x_4 > x_5) \\
  x_1 = \min (scor(x)) \\
  x_2 = \min_{x, x \neq x_1} (scor(x)) \\
  x_3 = \min_{x, x \neq x_1, x_2} (scor(x)) \\
  x_4 = \min_{x, x \neq x_1, x_2, x_3} (scor(x)) \\
  x_5 = \min_{x, x \neq x_1, x_2, x_3, x_4} (scor(x)) \\
  \{x_1, x_2, x_3, x_4, x_5\} \subseteq U
\end{gather*}



\appendix

\chapter{Dummy Appendix}

You can defer lengthy calculations that would otherwise only interrupt
the flow of your thesis to an appendix.



\backmatter

\bibliographystyle{plain}
\bibliography{refs}

\end{document}

