\chapter{Introducere}
Limbajul unei tari se afla in continua schimbare datorita mai multor cauze precum cauze sociale, 
economice, migratia populatiei, progresele in stiinta, technologie si medicina. De-a lungul timpului,
modificarile s-au petrecut inevitabil si nu au fost neaparat reglementate de experti in domeniu.
Astfel a aparut o noua ramura a lingvisticii ce studiaza sistematic aceste transformari incercand
sa gaseasca sabloane, reguli si sa puna o ordine asupra schimbarilor lexicale, fonetice, semantice
si sintactice. Cateva exemple clasice ar fi determinarea etimologiei unui cuvant (romanescul
\textit{genunchi} provine din latinescul \textit{genuclum}) sau determinarea similaritatii
limbajelor.

De cele mai multe ori, popoare ce vorbeau aceeasi limba s-au despartit si au evoluat diferit, aparand
limbaje derivate. Ne putem da seama de gradul de rudenie a acestora prin identificarea
formelor \textbf{cognate} (grupuri de cuvinte ce au derivat din acelasi etimon). In alte cazuri,
limbiile au imprumutat cuvinte intre ele fie luandu-le ca atare (cuvantul japonez \textit{sushi}),
fie modificand forma lor (romanescul \textit{cafea} din turcescul \textit{kahve}). Astfel, se pot 
descoperi chiar relatii de natura istorica intre limbi si popoarele ce le folosesc. Spre exemplu,
exista peste 200 de cuvinte ce se regasesc in toate limbile romanice moderne mai putin limba
romana. Este greu de crezut ca o parte din aceste cuvinte nu au existat in lexicul limbii romane
la un moment dat. Fischer\cite{fischer} a identificat cateva cauze ce au condus la disparitia acestor cuvinte
si inlocuirea acesora prin cuvinte cu alte etimologi: cauze externe, social-econimice, schimbarea
ocupatiilor romanilor, intreruperea Romaniei cu lumea Occidentala, dezvoltarea limbii romane departe
de nucelul romanic si asa mai departe.

\section{Reconstructia limbajelor folosind metode comparative}
Lingvisticii istorici se ocupa cu cercetarea schimbarilor limbilor de-a lungul timpului. 
O preocupare majora a acestora o reprezinta producerea de cuvinte inrudite. Ei folosesc
metode comparative prin care analizeaza modificarile limbajelor efectuand comparatii intre limibi
inrudite pentru a deduce, printr-o inginerie inversa, propietati ale limbii stramos comune.\cite{weissbook}
Practic, se incearca determinarea grupurilor de cuvinte cognate si gasirea unor reguli prin care
au fost obtinute din limba stramos. Acestea nu sunt neaparat atestate, cele mai multe dintre ele 
denumindu-se \textbf{proto-limbaje}. In consecinta, rezultatele obtinute sunt foarte greu de demonstrat
pentru ca nu exista dovezi arheologice concrete.

Asemenea metode au dat rezolvate satisfacatoare, reusind sa determine structura famililor de limbi
europene. Mai mult, reconstructia proto-limbajul \textbf{Indo-European} considerat cel mai vechi
limbaj cunoscut, a fost posibila prin punerea in corelatie cu limbajele derivate din acesta 
(proto-limbajul German, proto-limbajul Indo-Iranian).\cite{protostuff} In plus, metodele comparative
au avut avut succes si in analiza altor familii de limbaje de pe alte continente.

\section{Metode computationale}
In principal, tehnicile comparative au mai multi pasi efectuati manual de lingvistici si presupun 
multe ore de munca. In perioada in care calculatoarele nu erau inca inventate, cautarea cuvintelor 
in carti, dictionare si prelucrarea acestora era o munca nu numai obositoare si repetitiva dar si 
dificila, fiind nevoie de atentie continua. Insa, odata cu aparitia metodelor computationale, se pune accentul
pe determinarea automata a cuvintelor inrudite precum in \cite{kondrak} sau \cite{list}. Aceste
solutii sunt departe de a inlocui un expert in domeniul lingvisticii si isi propun mai mult sa vina
in ajutorul acestuia pentru a facilita dezvoltarea in profunzime a domeniului.

Metodele computationale doar automatizeaza procesul pastrand ideea pornirii de la formele moderne
ale cuvintelor pentru a le reconstrui pe cele vechi din care provin. Exista numeroase date intretinute
activ de catre specialisti ce determina conexiuni intre cuvinte din mai multe limbi moderne. Spre
exemplu, corpusul Europarl pentru limbile oficiale vorbite in Uniunea Europeana sau WordNet (Fellbaum, 1998)
pentru limba engleza. Asemenea resurse vaste sunt esentiale in aplicarea unor tehnici comparative 
automate. Pe baza lor se pot determina si aplica reguli de productie a cuvintelor intr-un mod formal,
fara prea mari interventii din partea lingvisticilor pentru asa zisele \textit{exceptii}.

Este dificil de prezis cum anume a fost modificat un cuvant pentru a ajunge in forma lui actuala.
Desi se presupune ca ar exista sabloane si reguli in evolutia unui cuvant din etimonul sau, sunt
si exemple care s-au detasat semnificativ de stramosul lor: latinescul \textit{umbilicu(lu)s} a ajuns
la foma de \textit{buric} in romana, \textit{nombril} in Franceza si \textit{umbigo} in Portugheza.

Detectia automata a cuvintelor cognate poate fi formulata ca o problema de clasificare prin invatare
autoamata. O serie de atribute s-au dovedid a fi de succes de-a lungul timpului precum $n$-gramele,
distante de editare, cel mai lung subsir comun etc. Jager si Sofroniev \cite{svmclass} determina
perechi cognate folosind un clasificator SVM iar apoi probabilitati si distante pentru a grupa
cuvinte in grupuri cognate. Rama \cite{cnn} aplica o retea neurala convolutionala. Ciobanu si Dinu
\cite{theone} folosesc campuri aleatoare conditionate reusind sa reconstruiasca proto-cuvinte din
seturi cognate incomplete.

\section{Latina si limbi romanice moderne}
Romana(ro), Italiana(it), Franceza(fr), Spaniola(es), Portugheza(pt) sunt doar cateva dintre limbile
moderne ce au evoluat din latina, in special din dialectul vulgar (latina vorbita de oamenii de rand 
in Imperiul Roman). Bineinteles, provenind din aceeasi limba, se pot pune foarte multe probleme de 
natura lingvistica precum gradul de similaritate dintre acestea, gasirea grupurilor de cuvinte cognate,
reconstructia de cuvinte latinesti neatestate si asa mai departe.

Tranzitia de la latina la o limba romanica moderna s-a efectuat prin schimbari majore de vocabular, 
sintaxa si fonologie. Aceste schimbari sunt mai mult sau mai putin similare pentru fiecare limba
romanica moderna. In principal, s-a urmarit simplificarea vocabulurului prin eliminarea arhaismelor
si a exceptilor in favoarea regulilor clare, reducerea sinonimelor, stabilirea clara a sensurilor
cuvintelor (conform \cite{sala}).

\section{Obiective si abordare}
Pentru limbile romanice \textit{ro, it, fr, es, pt} exita un dictionar etimologic in \cite{ripeanubook}
pornind de a limba latina. Acest set este incomplet si ne propunem sa-l completam folosind o metoda 
computationala bazata pe lingvistica comparativa. Evaluarea rezultatelor va fi facuta in mare parte
manual, aceste cuvinte nefiind atestate. Metoda computationala are la baza ideea prezentata in 
\cite{theone}.

Astfel, pornind de la mai multe perechi cognate din limbi romanice moderne, se va aplica o metoda 
de reconstructie a cuvintelor bazata pe etichetarea secventelor si campuri aleatoare conditionate. 
Fiecare limba moderna va fi analizata separat in raport cu limba latina. Apoi, rezultatele din toate 
limbile romanice vor fi combinate folosind agregari pe baza distantei rank \cite{rankdistance}.
Toate acestea vor fi detaliate pe larg in urmatoarele capitole.
