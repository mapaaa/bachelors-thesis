\chapter{Introducere}
Limbajul unei țări se află în continuă schimbare datorită mai multor cauze precum cauze sociale, 
economice, migrația populației, progresele în știință, tehnologie și medicină. De-a lungul timpului,
modificările s-au petrecut inevitabil și nu au fost neapărat reglementate de experți în domeniu.
Astfel a apărut o nouă ramură a lingvisticii ce studiază sistematic aceste transformări încercând
să găsească șabloane, reguli și să pună o ordine asupra schimbărilor lexicale, fonetice, semantice
și sintactice. Câteva exemple clasice ar fi determinarea etimologiei unui cuvânt (românescul
\textit{genunchi} provine din latinescul \textit{genuclum}) sau determinarea similarității
limbajelor.

De cele mai multe ori, popoare ce vorbeau aceeași limbă s-au despărțit și au evoluat diferit, apărând
limbaje derivate. Ne putem da seama de gradul de rudenie a acestora prin identificarea
formelor \textbf{cognate} (grupuri de cuvinte ce au derivat din același etimon). În alte cazuri,
limbile au împrumutat cuvinte între ele fie luându-le ca atare (cuvântul japonez \textit{sushi}),
fie modificând forma lor (romanescul \textit{cafea} din turcescul \textit{kahve}). Astfel, se pot 
descoperi chiar relații de natură istorică între limbi și popoarele ce le folosesc. Spre exemplu,
există peste 200 de cuvinte ce se regăsesc în toate limbile romanice moderne mai puțin limba
română. Este greu de crezut ca o parte din aceste cuvinte nu au existat în lexicul limbii române
la un moment dat. Fischer\cite{fischer} a identificat câteva cauze ce au condus la dispariția acestor cuvinte
și înlocuirea acestora prin cuvinte cu alte etimologi: cauze externe, social-economice, schimbarea
ocupațiilor romanilor, întreruperea României cu lumea Occidentala, dezvoltarea limbii române departe
de nucleul romanic și asa mai departe.

\section{Reconstrucția limbajelor folosind metode comparative}
Lingvisticii istorici se ocupă cu cercetarea schimbărilor limbilor de-a lungul timpului. 
O preocupare majora a acestora o reprezintă producerea de cuvinte înrudite. Ei folosesc
metode comparative prin care analizează modificările limbajelor efectuând comparații între limbaje
înrudite pentru a deduce, printr-o inginerie inversă, proprietăți ale limbii strămoș comune.\cite{weissbook}
Practic, se încearcă determinarea grupurilor de cuvinte cognate și găsirea unor reguli prin care
au fost obținute din limba strămoș. Acestea nu sunt neapărat atestate, cele mai multe dintre ele 
denumindu-se \textbf{proto-limbaje}. În consecința, rezultatele obținute sunt foarte greu de demonstrat
pentru ca nu exista dovezi arheologice concrete.

Asemenea metode au dat rezultate satisfăcătoare, reușind să determine structura familiilor de limbi
europene. Mai mult, reconstrucția proto-limbajul \textbf{Indo-European} considerat cel mai vechi
limbaj cunoscut, a fost posibilă prin punerea în corelație cu limbajele derivate din acesta 
(proto-limbajul German, proto-limbajul Indo-Iranian).\cite{protostuff} În plus, metodele comparative
au avut avut succes și în analiza altor familii de limbaje de pe alte continente.

\section{Metode computaționale}
În principal, tehnicile comparative au mai multi pași efectuați manual de lingviști și presupun 
multe ore de muncă. În perioada în care calculatoarele nu erau încă inventate, căutarea cuvintelor 
în cărți, dicționare și prelucrarea acestora era o muncă nu numai obositoare și repetitivă dar și 
dificilă, fiind nevoie de atenție continuă. Însă, odată cu apariția metodelor computaționale, se pune accentul
pe determinarea automată cuvintelor înrudite precum în \cite{kondrak} sau \cite{list}. Aceste
soluții sunt departe de a înlocui un expert în domeniul lingvisticii și își propun mai mult să vină
în ajutorul acestuia pentru a facilita dezvoltarea în profunzime a domeniului.

Metodele computaționale doar automatizează procesul păstrând ideea pornirii de la formele moderne
ale cuvintelor pentru a le reconstrui pe cele vechi din care provin. Există numeroase date întreținute
activ de către specialiști ce determină conexiuni între cuvinte din mai multe limbi moderne. Spre
exemplu, corpusul Europarl pentru limbile oficiale vorbite în Uniunea Europeana sau WordNet (Fellbaum, 1998)
pentru limba engleza. Asemenea resurse vaste sunt esențiale în aplicarea unor tehnici comparative 
automate. Pe baza lor se pot determina și aplica reguli de producție a cuvintelor într-un mod formal,
fără prea mari intervenții din partea lingvisticilor pentru asa zisele \textit{excepții}.

Este dificil de prezis cum anume a fost modificat un cuvânt pentru a ajunge în forma lui actuală.
Deși se presupune că ar exista șabloane și reguli în evoluția unui cuvânt din etimonul sau, sunt
și exemple care s-au detașat semnificativ de strămoșul lor: latinescul \textit{umbilicu(lu)s} a ajuns
la forma de \textit{buric} în română, \textit{nombril} în franceză și \textit{umbigo} în portugheză.

Detecția automată a cuvintelor cognate poate fi formulată ca o problemă de clasificare prin învățare
automată. O serie de atribute s-au dovedit a fi de succes de-a lungul timpului precum $n$-gramele,
distanțe de editare, cel mai lung subșir comun etc. Jager si Sofroniev \cite{svmclass} determină
perechi cognate folosind un clasificator SVM iar apoi probabilități și distanțe pentru a grupa
cuvinte în grupuri cognate. Rama \cite{cnn} aplica o rețea neurală convoluțională. Ciobanu și Dinu
\cite{theone} folosesc câmpuri aleatoare condiționate reușind să reconstruiască proto-cuvinte din
seturi cognate incomplete.

\section{Latina și limbi romanice moderne}
Română(ro), Italiană(it), Franceză(fr), Spaniolă(es), Portugheză(pt) sunt doar câteva dintre limbile
moderne ce au evoluat din latina, în special din dialectul vulgar (latina vorbita de oamenii de rand 
în Imperiul Roman). Bineînțeles, provenind din aceeași limbă, se pot pune foarte multe probleme de 
natură lingvistică precum gradul de similaritate dintre acestea, găsirea grupurilor de cuvinte cognate,
reconstrucția de cuvinte latinești neatestate și asa mai departe.

Tranziția de la latină la o limbă romanică moderna s-a efectuat prin schimbări majore de vocabular, 
sintaxă și fonologie. Aceste schimbări sunt mai mult sau mai puțin similare pentru fiecare limbă
romanică modernă. În principal, s-a urmărit simplificarea vocabularului prin eliminarea arhaismelor
și a excepțiilor în favoarea regulilor clare, reducerea sinonimelor, stabilirea clară a sensurilor
cuvintelor (conform \cite{sala}).

\section{Obiective si abordare}
Pentru limbile romanice \textit{ro, it, fr, es, pt} există un dicționar etimologic în \cite{ripeanubook}
pornind de a limba latină. Acest set este incomplet și ne propunem sa-l completam folosind o metoda 
computațională bazata pe lingvistica comparativă\cite{sub}. Evaluarea rezultatelor va fi făcută în 
mare parte manual, aceste cuvinte nefiind atestate. Metoda computațională are la baza ideea prezentată în 
\cite{theone}.

Astfel, pornind de la mai multe perechi cognate din limbi romanice moderne, se va aplica o metodă
de reconstrucție a cuvintelor bazată pe etichetarea secvențelor și câmpuri aleatoare condiționate. 
Fiecare limba modernă va fi analizată separat în raport cu limba latină. Apoi, rezultatele din toate 
limbile romanice vor fi combinate folosind agregări pe baza distanței rank \cite{rankdistance}.
Toate acestea vor fi detaliate pe larg în următoarele capitole.
