\chapter{Agregarea rezultatelor folosind distanța rank}
Am văzut cum putem obține producții de cuvinte combinând câte o singură limbă romanică cu 
limba latină. Pentru a îmbunătății rezultatele vrem să folosim informația din mai multe 
limbi romanice moderne. Astfel, fiecare clasificator întoarce o listă ordonată de cuvinte latinești, 
pe prima poziție aflându-se perechea cognate cu cea mai mare probabilitate. Prin agregarea 
acestora cu o anumită metrică vom obține cele mai probabile perechi cognate.

\section{Clasamente și distanța rank}
Un \textit{clasament} este o listă ordonată de obiecte, pe prima poziție aflându-se cel cu cea mai
mare importanță. În unele situații se pune problema găsirii unui clasament cât mai apropiat de o
mulțime de mai multe clasamente. Pentru a rezolva această problemă trebuie sa definim mai întâi ce
înseamnă distanța dintre două clasemente sau dintre un singur clasament și o mulțime de clasamente.

Există mai multe distanțe folosite cu succes în diverse aplicații: distanța \textit{Kedall tau}, 
\textit{Spearman footrule}, \textit{Levenshtein}, dar noi vom folosi distanța \textit{rank}
\cite{rankdistance}.

\section{Determinarea agregărilor optime}
\subsection{Reducerea la o problemă de cuplaj perfect de cost minim}
\subsection{Calcularea tuturor agregărilor optime}
