\chapter{Agregarea rezultatelor folosind distanța rank}
Am văzut cum putem obține producții de cuvinte combinând câte o singură limbă romanică cu 
limba latină. Pentru a îmbunătății rezultatele vrem să folosim informația din mai multe 
limbi romanice moderne. Astfel, fiecare clasificator întoarce o listă ordonată de cuvinte latinești, 
pe prima poziție aflându-se perechea cognate cu cea mai mare probabilitate. Prin agregarea 
acestora cu o anumită metrică vom obține cele mai probabile perechi cognate.

\section{Clasamente și distanța rank}
Un \textit{clasament} este o listă ordonată de obiecte după un anumit criteriu, pe prima poziție 
aflându-se cel cu cea mai mare importanță. În unele situații se pune problema găsirii unui clasament
cât mai apropiat de o mulțime de mai multe clasamente. Pentru a rezolva această problemă trebuie sa
definim mai întâi ce înseamnă distanța dintre două clasemente sau dintre un singur clasament și o
mulțime de clasamente.

Există mai multe distanțe folosite cu succes în diverse aplicații: distanța \textit{Kedall tau}, 
\textit{Spearman footrule}, \textit{Levenshtein}, dar noi vom folosi distanța \textit{rank}
\cite{rankdistance}, pe care o vom defini folosindu-ne de următoarele noțiuni. Fie
$U = \{1, 2, ..., n\}$ 
o mulțime finită de obiecte numită univers. Un clasament peste universul $U$
este lista ordonată 
$\tau = (x_1 > x_2 > ... > x_d)$, unde 
$\{x_1, x_2, ..., x_d\} \subseteq U$.
$>$ este o relație de ordine strictă reprezentând criteriul de ordonare. Notăm cu $\tau(x)$, poziția
elementului $x \in U$ în clasamentul $\tau$ dacă $x \in \tau$, numerotând pozițiile de la 1 
începând cu cel mai important obiect din clasament. Dacă un clasament conține toate elementele din
univers, atunci el se va numi \textit{clasament total}. Asemănător, dacă conține doar o submulțime
de obiecte din univers, atunci îl vom numi \textit{clasament parțial}.

Notăm ordinea elementului $x$ în $\tau$ astfel:
\[
  ord(\tau, x)= \begin{dcases}
    |n + 1 - \tau(x)|    &, x \in \tau \\
    0                    &, x \in U \setminus \tau
  \end{dcases}
\]

\begin{definition}
Fie $\tau$ si $\sigma$ două clasamente parțiale peste același univers $U$. Atunci distanța rank va fi
\[
  \Updelta(\tau, \sigma) = \smashoperator{\sum_{x \in \tau \cup \sigma}} |ord(\tau, x) - ord(\sigma, x)|
\]
\end{definition}

Se observă faptul că, în calculul distanței rank, se ia în considerare ordinea definită mai sus
și nu poziția. În primul rând, cum primele poziții sunt cele mai importante, distanța dintre două
clasamente trebuie sa fie cu atât mai mare cu cât diferă mai mult începutul lor\cite{linguisticstructuresmarcus}.
În al doilea rând, definiția funcției $ord$ pune accentul pe lungimea clasamentelor întrucăt putem
presupune că un clasament mai lung a fost obținut în urma comparării mai multor obiecte din univers.
Deci ordinea elementelor este mai solidă. Spre exemplu, dacă două clasamente de lungimi diferite au
același element pe prima poziție, există totuși o diferență a ordinii obiectului în cele două liste,
diferență ce contribuie la calculul distanței rank totale.\cite{rankaggregationproblem}

\subsection{Agregări cu distanța rank}
O \textit{agregare de clasamente} reprezintă un singur clasament $\sigma$ astfel încât o anumită 
metrică de la acesta la mulțimea de liste de agregat $T$ este minimă. Raportându-ne la distanța rank,
putem defini formal o agregare astfel:

\begin{definition}
Fie un set de clasamente $T = \{\tau_1, \tau_2, ..., \tau_m\}$ dintr-un univers $U$ și
$\sigma = (\sigma_1 > \sigma_2 > ... > \sigma_k)$ un clasament astfel încât $\sigma_i \in U, 
\forall 1 \leqslant i \leqslant k$. Definim distanța rank de la $\sigma$ la $T$ astfel:
\[
  \Updelta(\sigma, T) = \smashoperator{\sum_{i = 1}^{m}} \Updelta(\sigma, \tau_i)
\]
\end{definition}

\begin{definition}
Fie un set de clasamente $T = \{\tau_1, \tau_2, ..., \tau_m\}$ dintr-un univers 
$U = \{x_1, x_2, ..., x_n \}$. Se numește agregare a mulțimii $T$ folosind distanța rank clasamentul
$\sigma$, $\sigma_i \in U, \forall 1 \leqslant i \leqslant k$, cu propietatea că 
$\Updelta(\sigma, T)$ este minim posibilă.
\end{definition}

Definim următoarele matrici bidimensionale $W^l(i, j)$ cu $n$ linii și $n$ coloane. Fiecare celulă
din fiecare matrice va semnifica costul total din distanța rank de la un clasament $\sigma$, de 
lungime $l$, către o mulțime $T = \{\tau_1, \tau_2, ..., \tau_m\}$ fixată indus de plasarea 
elementului $x_i \in U$ pe poziția $j$ în $\sigma$ \cite{rankaggregationproblem}. Se observă faptul
că un clasament peste universul $U$ definit mai sus poate avea lungimea maxim $n$. Rezultă că 
numărul de coloane al matricilor $W^t$ este egal cu $n$.
\[
  W^l(i, j) = \begin{dcases}
    \smashoperator{\sum_{p=1}^{m}} | ord(p, i) - l + j |    &, j \leqslant l \\
    \smashoperator{\sum_{p=1}^{m}} | ord(p, i) |            &, j > l
  \end{dcases}
\]

\begin{remark}
Distanța de la un clasament $\sigma=(\sigma_1 > \sigma_2 > ... \sigma_k)$ la mulțimea $T$ este
\[
  \Updelta(\sigma, T) = \smashoperator{\sum_{x_i \in U \cap \sigma}} W^k(i, \sigma(x_i)) +
      \smashoperator{\sum_{x_i \in U \setminus \sigma}} W^k(i, k + 1)
\]unde $n$ este reprezintă numărul de obiecte din univers, iar $k < n$.
\end{remark}
Se observă faptul că, în cazul în care $\sigma$ conține toate elementele din $U$, deci cazul $k = n$
, formula se reduce la 
\begin{gather}
  \label{eq:minimize}
  \Updelta(\sigma, T) = \smashoperator{\sum_{x_i \in U \cap \sigma}} W^k(i, \sigma(x_i))
\end{gather}


\subsection{Reducerea la o problemă de cuplaj perfect de cost minim}
Fiecare matrice $W^l$ din secțiunea precedentă este calculată în mod independent de celelalte.
Deci putem determina doar o singură matrice pentru o anumită lungime fixată $l$. Vom presupune
în continuare ca vrem să calculăm agregările unui set de clasamente având lungimea egală cu numărul
total de elemente din univers. Astfel, problema se reduce la găsirea unu clasament $\sigma$ ce 
minimizează formula \eqref{eq:minimize}. Formal:

\begin{problem}
Fiind dată o matrice pătratică $W$, $W = (w_{i, j})_{1 \leqslant i,j \leqslant n}$ vrem să
determinăm următoarea mulțime:
\[
  S = \{(i_1, i_2, ..., i_n) | (i_k \neq i_j, \forall k \neq j), (i_j \in U) \text{ și } \smashoperator{\sum_{j=1}^n} w_{i_j, j} \text{ este minim}\}
\]
\end{problem}

Problema de mai sus se aseamănă cu o problemă de cuplaj perfect de cost minim întrucăt vrem să formăm
perechi între obiectele dintr-un univers și pozițiile unui clasament de tip agregare, iar fiecare
combinație are un anumit cost. Practic $(i_1, i_2, ..., i_n)$ reprezintă o permutare a elementelor
din $U$.

O solutie pentru a rezolva problema precedenta este aplicarea algoritmul Ungar prezentat de Khun \cite{hungarianmethod}.
Altfel, putem considera matricea $W$ ca fiind o matrice de costuri intr-un graf bipartit $G$ pe care
aplicam un algoritm clasic de gasire a cuplajului maxim de cost minim. Conform \cite{flowassignment}
aceasta problema poate fi rezolvata in timp polinomial $\mathcal{O}(n^3)$ construind o retea de flux
cu capacitati convenabile si prin gasirea unor drumuri de augmentare minime din punct de vedere
al costului folosind algoritmul lui Dijsktra\cite{dijkstra}.


\subsection{Calcularea tuturor agregărilor optime}



\section{Determinarea tuturor agregărilor producțiilor de cuvinte}
În capitolul precedent am prezentat o metoda de a combina o limbă romanică modernă si limba latină
pentru a automatiza procesul de determinare a etimonului latinesc. Metoda returna primele $n$
cuvinte posibile ordonate de la cel cu probabilitatea cea mai mare la cel cu probabilitatea cea mai
mică. Vom considera aceste liste de cuvinte ca fiind clasamente. Pentru fiecare cuvânt latinesc vom
agrega clasamentele produse din fiecare limbă romanică modernă \textit{(ro, it, fr, es, pt)}.
Se observă faptul că pot exista mai multe astfel de agregări așa că ne propunem să le aflăm pe toate
într-un mod eficient din punct de vedere al complexității timp. Alegem să luăm în considerare
doar primele 5 cele mai bune cuvinte din fiecare set. Astfel, pentru un singur cuvânt latinesc, vom 
avea:
\begin{gather*}
  R = \{r_1, r_2, ..., r_k\}, \text{clasamentele produse din ro, it, fr, es, pt} \\
  k = |R|, 1 \leqslant k \leqslant 5 \\
  U = \smashoperator{\bigcup_{i = 1}^{k}} r_i \text{ universul de cuvinte} \\
  n = |U|
\end{gather*}
Definim o matrice bidimensională de $k$ linii și $n$ coloane în care calculăm ordinea fiecărui 
cuvânt din univers în fiecare clasament dat:
\[
  ord[i][j] = \begin{dcases}
    |6 - r_i(x_j)|    &, x_j \in r_i \\
    0                 &, x_j \in U \setminus r_i 
  \end{dcases}
\]
