\chapter{Agregarea rezultatelor folosind distanța rank}
\label{chap:three}
Am văzut cum putem obține producții de cuvinte combinând câte o singură limbă romanică cu 
limba latină. Pentru a îmbunătății rezultatele vrem să folosim informația din mai multe 
limbi romanice moderne. Astfel, fiecare clasificator întoarce o listă ordonată de cuvinte latinești, 
pe prima poziție aflându-se etimonul latinesc cu cea mai mare probabilitate. Prin agregarea 
acestora cu o anumită metrică vom obține o lista sortata cu mai probabile cuvinte latinești. Metrica
folosita este distanta rank \cite{rankdistance} întrucât s-au obținut rezultate bune în alte 
probleme de natură lingvistică precum determinarea similitudinii silabice a limbilor romanice 
\cite{syllabicsim}, \cite{simnat}.

În primul rând vom defini ce înseamna o listă ordonată de elemente. În al doilea rând, vom explica
distanța rank între două clasamente și între un clasament și o mulțime de clasamente. Apoi vom
prezenta o metodă de aflare a tuturor agregărilor unei mulțimi de mai multe clasamente folosind
distanța rank. În final, vom prelucra mulțimea de agregări bazat pe un sistem de vot pentru a 
determina \textbf{o singură listă ordonată} de posibile etimoane latinești.

\section{Clasamente și distanța rank}
Un \textit{clasament} este o listă ordonată de obiecte după un anumit criteriu, pe prima poziție 
aflându-se cel cu cea mai mare importanță. În unele situații se pune problema găsirii unui clasament
cât mai apropiat de o mulțime de mai multe clasamente. Pentru a rezolva această problemă trebuie să
definim mai întâi ce înseamnă distanța dintre două clasamente sau dintre un singur clasament și o
mulțime de clasamente.

Există mai multe metrici folosite cu succes în diverse aplicații: distanța \textit{Kedall tau}, 
\textit{Spearman footrule}, \textit{Levenshtein}, dar noi vom folosi distanța \textit{rank}
introdusa in articolul \cite{rankdistance}. În întregul capitol vom folosi următoarele notații:
\begin{itemize}
    \item $U = \{1, 2, ..., n\}$ o mulțime finită de obiecte numită univers 
    \item $\tau = (x_1 > x_2 > ... > x_d)$ un clasament peste universul $U$ 
    \item $>$ o relație de ordine strictă reprezentând criteriul de ordonare 
    \item $\tau(x)$ = poziția elementului $x \in U$ în clasamentul $\tau$ dacă $x \in \tau$, 
      numerotând pozițiile de la 1 începând cu cel mai important obiect din clasament
\end{itemize}

Dacă un clasament conține toate elementele din univers, atunci el se va numi 
\textit{clasament total}. Asemănător, dacă conține doar o submulțime de obiecte din univers, atunci
îl vom numi \textit{clasament parțial}.

Notăm ordinea elementului $x$ în $\tau$ astfel:
\begin{gather}
\label{ord}
  ord(\tau, x)= \begin{dcases}
    |n + 1 - \tau(x)|    &, x \in \tau \\
    0                    &, x \in U \setminus \tau
  \end{dcases}
\end{gather}

\begin{definition}
Fie $\tau$ și $\sigma$ două clasamente parțiale peste același univers $U$. Atunci distanța rank va fi
\begin{gather}
  \Updelta(\tau, \sigma) = \smashoperator{\sum_{x \in \tau \cup \sigma}} |ord(\tau, x) - ord(\sigma, x)|
\end{gather}
\end{definition}

Se observă faptul că, în calculul distanței rank, se ia în considerare ordinea definită mai sus
și nu poziția. În primul rând, cum primele poziții sunt cele mai importante, distanța dintre două
clasamente trebuie sa fie cu atât mai mare cu cât diferă mai mult începutul lor.\cite{linguisticstructuresmarcus}
În al doilea rând, definiția funcției $ord$ pune accentul pe lungimea clasamentelor întrucăt putem
presupune că un clasament mai lung a fost obținut în urma comparării mai multor obiecte din univers.
Deci ordinea elementelor este mai solidă. Spre exemplu, dacă două clasamente de lungimi diferite au
același element pe prima poziție, există totuși o diferență a ordinii obiectului în cele două liste,
diferență ce contribuie la calculul distanței rank totale.\cite{rankaggregationproblem}

\subsection{Agregări cu distanța rank}
O \textit{agregare de clasamente} reprezintă un singur clasament $\sigma$ astfel încât o anumită 
metrică de la acesta la mulțimea de liste de agregat $T$ este minimă. Raportându-ne la distanța
rank avem\cite{rankdistance}:

\begin{definition}
Fie un set de clasamente $T = \{\tau_1, \tau_2, ..., \tau_m\}$ dintr-un univers $U$ și
$\sigma = (\sigma_1 > \sigma_2 > ... > \sigma_k)$ un clasament astfel încât $\sigma_i \in U, 
\forall 1 \leqslant i \leqslant k$. Definim distanța rank de la $\sigma$ la $T$ astfel:
\begin{gather}
  \Updelta(\sigma, T) = \smashoperator{\sum_{i = 1}^{m}} \Updelta(\sigma, \tau_i)
\end{gather}
\end{definition}

\begin{definition}
\label{def:Aset}
Se numește mulțime de agregari de lungime $k$ a mulțimii $T$ folosind distanța rank, setul
$
  A(T, k) = \{\sigma=(\sigma_1 > \sigma_2 > ... > \sigma_k) | \sigma_i \in U, 
  \forall 1 \leqslant i \leqslant k$, si 
  $\Updelta(\sigma, T)$ este minim posibila \}
\end{definition}


\begin{problem}
Fie $U$ un set de obiecte și $T = \{\tau_1, \tau_2, ..., \tau_m\}$ o mulțime de clasamente peste
universul $U$. Vrem să determinăm mulțimea de agregări $A(T, k)$ pentru un k fixat. 
\end{problem}

Construim următoarele matrici bidimensionale $W^k(i, j)$ cu $n$ linii și $n$ coloane. Fiecare celulă
din fiecare matrice reprezintă costul total din distanța rank de la un clasament $\sigma$, de 
lungime $l$, către o mulțime $T = \{\tau_1, \tau_2, ..., \tau_m\}$ fixată indus de plasarea 
elementului $x_i \in U$ pe poziția $j$ în $\sigma$ \cite{rankaggregationproblem}. Se observă faptul
că un clasament peste universul $U$ definit mai sus poate avea lungimea maxim $n$. Rezultă că 
numărul de coloane al matricilor $W^t$ este egal cu $n$.
\begin{gather}
  \label{eq:wmatrix}
  W^k(i, j) = \begin{dcases}
    \smashoperator{\sum_{p=1}^{m}} | ord(p, i) - k + j |    &, j \leqslant k \\
    \smashoperator{\sum_{p=1}^{m}} | ord(p, i) |            &, j > k
  \end{dcases}
\end{gather}

\begin{remark}
Distanța de la un clasament $\sigma=(\sigma_1 > \sigma_2 > ... \sigma_k)$ la mulțimea $T$ este
\[
  \Updelta(\sigma, T) = \smashoperator{\sum_{x_i \in U \cap \sigma}} W^k(i, \sigma(x_i)) +
      \smashoperator{\sum_{x_i \in U \setminus \sigma}} W^k(i, k + 1)
\]unde $n$ reprezintă numărul de obiecte din univers, iar $k < n$.
\end{remark}
Se observă faptul că, în cazul în care $\sigma$ conține toate elementele din $U$, deci cazul $k = n$
, formula se devine
\begin{gather}
  \label{eq:minimize}
  \Updelta(\sigma, T) = \smashoperator{\sum_{x_i \in U \cap \sigma}} W^k(i, \sigma(x_i))
\end{gather}

\subsection{Reducerea la o problemă de cuplaj perfect de cost minim}
Fiecare matrice $W^l$ din secțiunea precedentă este calculată în mod independent de celelalte
Deci putem determina doar o singură matrice pentru o anumită lungime fixată $l$. 
Astfel, problema se reduce la găsirea unui clasament $\sigma$ ce 
minimizează formula \eqref{eq:minimize}. Formal:

\begin{problem}
\label{def:problem}
Fiind dată o matrice pătratică $W$, $W = (w_{i, j})_{1 \leqslant i,j \leqslant n}$ vrem să
determinăm următoarea mulțime:
\[
  S = \{(i_1, i_2, ..., i_k) | (i_p \neq i_j, \forall p \neq j), (i_j \in U) \text{ și } \smashoperator{\sum_{j=1}^n} w_{i_j, j} \text{ este minim}\}
\]
\end{problem}

Problema de mai sus se aseamănă cu o problemă de cuplaj de dimensiune $k$ de cost minim întrucât 
vrem să formăm perechi între obiectele dintr-un univers și pozițiile unui clasament de tip agregare, 
iar fiecare combinație are un anumit cost. Practic $(i_1, i_2, ..., i_n)$ reprezintă o permutare a 
elementelor din $U$.

O soluție pentru a rezolva problema precedentă este aplicarea algoritmul Ungar prezentat de Khun \cite{hungarianmethod}.
Altfel, putem considera matricea $W$ ca fiind o matrice de costuri într-un graf bipartit $G$ pe care
aplicăm un algoritm clasic de găsire a cuplajului maxim de cost minim (din care luam doar $k$ perechi). 
Conform \cite{flowassignment} această problemă poate fi rezolvată în timp polinomial 
$\mathcal{O}(n^3)$ construind o rețea de flux cu capacități convenabile și prin găsirea unor drumuri 
de augmentare minime, din punct de vedere al costului, folosind algoritmul lui Dijkstra\cite{dijkstra}.

Toate aceste rezolvări determină o singură agregare dar nu și pe toate, adică mulțimea $A(T, k)$ din
definiția \ref{def:Aset}. În continuare prezentăm o metoda de determinare a tuturor agregărilor 
bazată pe găsirea tuturor cuplajelor \textit{perfecte} de cost minim dintr-un graf, metoda 
prezentată în \cite{allmatchings}. Algoritmul rulează într-un timp polinomial. Particularizăm 
problemele și algoritmii din articolul  \textit{A generalization of the assignment problem, and its
application to the rank aggregation  problem} \cite{allmatchings} pentru Problema \ref{def:problem}.


\subsection{Calcularea tuturor agregărilor optime}
Reamintim faptul că dorim să calculăm mulțimea de agregări $A(T, k)$, știind costul plasării 
fiecărui element pe fiecare poziție, memorat în matricea $W^k$ calculată la \eqref{eq:wmatrix}. 
Reformulăm problema în elemente de teoria grafurilor. Astfel, asociem Problemei \ref{def:problem} 
un graf $G = (V, E, c, w)$, unde $V$ reprezintă mulțimea de noduri, $E$ este mulțimea de muchii iar
$c \colon E \to \mathbb{N}$ și $w \colon E \to \mathbb{N}$ reprezintă capacitatea unei muchii
respectiv costul acesteia. Legăturile între Problema \ref{def:problem} și graful $G$ sunt:
\begin{itemize}
  \item $V = \{src, dst\} \cup U \cup \{1, 2,..., k, k+1\}$
  \item $E = \{(src, x_i) | x_i \in U\} \cup \{(x_i, j) | x_i \in U \text{ si } j = 1,...,k\} \cup 
    \{(j, dst) | j = 1,...,k\}$  
  \item $c(muchie) = 1, \forall muchie = (x, j) \in E, j \neq k+1$
  \item $c(muchie) = \infty, \forall muchie = (x, k+1) \in E$
  \item funcția $w$ astfel:
  \begin{itemize}
    \item $w((src, x_i)) = 0, \forall x_i \in U$
    \item $w((x_i, j)) = W^k(i, j), \forall x_i \in U, j = 1,...,k+1$
    \item $w((j, dst)) = 0, j = 1,...,k+1$
  \end{itemize}
\end{itemize}

Potrivirea unui element $x$ cu poziția $k+1$ va semnifica excluderea acestuia din agregare ce 
afectează distanța rank dintre un clasament ce nu conține elementul $x$ și mulțimea întreagă $T$.
Se poate calcula ușor în acest graf un flux maxim de cost minim folosind algoritmi clasici\cite{flowassignment}).
Notăm prin $solve(W)$ un asemenea algoritm.
Fie soluția $M = \{(x, j) | x \in U \text{ și } j = 1,...,k\}$. Următorul pas este aflarea unei
soluții $M'$ diferite de $M$.

\begin{proposition}
Doua cuplaje $M$ și $M'$ sunt diferite dacă există cel puțin o pereche $(x, y)$ care se află în $M$
și nu se află în $M'$.
\end{proposition}

Astfel, propunem următorul algoritm, adaptat din \cite{allmatchings}, prin care căutăm o a doua
soluție $M'$ fixând câte o muchie candidat $(x, y)$ prin care $M'$ sa difere de $M$. Setând costul
muchiei $(x, y)$ pe o valoare infinită, avem garanția ca aceasta nu va fi luată în considerare în 
construcția lui $M'$.
\begin{algorithm}
\label{P}
\caption{anotherSolution}
\begin{algorithmic}[1]
\REQUIRE $W, M$
\ENSURE $M'$
  \STATE $s \gets \sum_{(u, v) \in M} w_{uv}$
  \FORALL{$(x, y) \in M$}
    \STATE $temp \gets w_{xy}$
    \STATE $w_{xy} \gets \infty$
    \STATE $M' \gets solve(W)$
    \IF{$M' \neq \emptyset \text{ si } \sum_{(u, v) \in M'} w_{uv} = s$}
    \RETURN $M'$
    \ELSE
    \STATE $w_{xy} \gets temp$
    \ENDIF
  \ENDFOR
  \RETURN $\emptyset$
\end{algorithmic}
\end{algorithm}

Algoritmul returnează fie mulțimea vidă, fie o soluție $M'$ astfel încât exista o pereche $(x, y)
\in M \setminus M'$. Se poate împărți problema inițială în două subprobleme disjuncte $P_1$ și
$P_2$:
\begin{itemize}
  \item $P_1\colon$ mulțimea tuturor cuplajelor ce conțin muchia $(x, y)$ \\
    În acest caz forțăm păstrarea perechii în soluție prin setarea tuturor celorlalte valori de pe
    linia $x$, coloana $y$ pe o valoare infinita în matricea $W$:
    $w_{iy} = w_{xj}, \forall i = 1,..,n \text{, }i \neq x  \text{ și } j = 1,...,n \text{, }j \neq y$
  \item $P_2\colon$ mulțimea tuturor cuplajelor ce \textbf{nu} conțin muchia $(x, y)$ \\
    În acest caz perechea $(x, y)$ nu va mai fi niciodată aleasă într-o soluție dacă costul acesteia
    este infinit:
    $w_{xy} = \infty$
\end{itemize}

Evident, există deja cate o soluție calculată pentru cele 2 subprobleme și anume $M \in P_1$ și 
$M' \in P_2$. Prim urmare, se poate aplica Algoritmul 1 în mod recursiv pentru fiecare dintre 
aceste subprobleme pentru a determina întreaga mulțime de soluții. Această abordare conduce la
construirea unei structuri de căutare arborescente în care rădăcina reprezintă problema inițială
\ref{def:problem}, iar fiecare nod intern constituie o împărțire pe subprobleme după o pereche 
$(x, y)$. Soluția finală se construiește traversând arborele în adâncime și păstrând toate soluțiile
parțiale calculate la fiecare pas. Nu se va genera aceeași soluție de mai multe ori prin faptul că
problemele $P_1$ și $P_2$ sunt complet disjuncte.

\begin{algorithm}
\caption{dfsAgregare}
\begin{algorithmic}[1]
\REQUIRE $S, M, W$
  \STATE $s \gets \sum_{(u, v) \in M} w_{uv}$
  \STATE $M' \gets another_solution(W, M)$
  \IF{$M' \neq \emptyset$}
    \RETURN
  \ELSE
    \STATE $S \gets M'$
    \STATE $(x, y) \in M \setminus M'$
    \STATE $w_{xy} \gets \infty$
    \STATE $dfsAgregare(S, M', W)$
    \STATE $w_{iy} \gets w_{xj}, \forall i = 1,..,n \text{, }i \neq x  \text{ si } j = 1,...,n \text{, }j \neq y$
    \STATE $dfsAgregare(S, M' \setminus (x, y), W)$
  \ENDIF
\end{algorithmic}
\end{algorithm}

\begin{algorithm}
\caption{Calculează toate cuplajele perfecte de cost minim}
\begin{algorithmic}[1]
\REQUIRE $W$
\ENSURE $S$
  \STATE $S \gets \emptyset$
  \STATE $M \gets solve(W)$
  \STATE $S \gets S \cup M$
  \STATE $dfsAgregare(S, M, W)$
  \RETURN $S$
\end{algorithmic}
\end{algorithm}

Algoritmul 3 determină toate cuplajele perfecte de cost minim pornind de la o matrice de costuri
$W$. Calculând matricea $W$ după formula \eqref{eq:wmatrix}, atunci mulțimea $S$ determinată de
algoritm este chiar soluția căutată în Problema \ref{def:problem}. Câteva aspecte legate de
complexitatea metodei prezentate: \\
Fie $x = |S|$, numărul total de soluții pentru o anumita problema.
\begin{itemize}
  \item Pentru fiecare soluție nou calculată, se construiesc două noi alte probleme. În total se vor
    rezolva maxim $2*x+1$ subprobleme.
  \item O subproblemă necesită găsirea unui cuplaj de cost minim ce se poate calcula într-un timp
    polinomial folosind metoda Ungară\cite{hungarianmethod} ori un algoritm clasic de determinare
    a fluxului maxim de cost minim într-un graf bipartit\cite{flowassignment}.
\end{itemize}
Intuitiv, putem afirma ca Algoritmul 3 rulează într-un timp polinomial. Complexitatea acestuia a
fost demonstrată în \cite{allmatchings} ca fiind $\mathcal{O}((2x+1)k^3n\log(n+k))$.

Subproblemele sunt complet independente de complementarele lor. Prin urmare, rezolvările acestora se
pot rula în paralel pe mai multe fire de execuție. În experimentele rulate am ales să pornim
un nou fir de execuție pentru fiecare subproblemă de tipul $P_2$ până la un anumit nivel din
arborele de căutare determinat în funcție de procesorul folosit. Subproblema de tipul $P_1$
a rămas să ruleze pe firul de execuție principal.


\section{Determinarea tuturor agregărilor producțiilor de cuvinte}
În capitolul precedent am prezentat o metoda de a combina o limbă romanică modernă și limba latină
pentru a automatiza procesul de determinare a etimonului latinesc. Metoda returna primele $n$
cuvinte posibile ordonate de la cel cu probabilitatea cea mai mare la cel cu probabilitatea cea mai
mică. Vom considera aceste liste de cuvinte ca fiind clasamente. Pentru fiecare cuvânt latinesc vom
agrega clasamentele produse din fiecare limbă romanică modernă \textit{(ro, it, fr, es, pt)}.
Se observă faptul că pot exista mai multe astfel de agregări așa că ne propunem să le aflăm pe toate
într-un mod eficient din punct de vedere al complexității timp. Alegem să luăm în considerare
doar primele 5 cele mai bune cuvinte din fiecare set. Astfel, pentru un singur cuvânt latinesc, vom 
avea:
\begin{gather*}
  R = \{r_1, r_2, ..., r_k\}, \text{clasamentele produse din ro, it, fr, es, pt} \\
  k = |R|, 1 \leqslant k \leqslant 5 \\
  U = \smashoperator{\bigcup_{i = 1}^{k}} r_i \text{ universul de cuvinte} \\
  n = |U|
\end{gather*}
Definim o matrice bidimensională de $k$ linii și $n$ coloane în care calculăm ordinea fiecărui 
cuvânt din universul unui cuvânt în fiecare clasament dat:
\[
  ord[i][j] = \begin{dcases}
    |6 - r_i(x_j)|    &, x_j \in r_i \\
    0                 &, x_j \in U \setminus r_i 
  \end{dcases}
\]

Calculăm apoi matricea de costuri $W^5$ conform formulei de la \ref{eq:wmatrix} pentru a afla cum
este afectat costul final dacă un anumit cuvânt $x_i \in U$ este plasat pe poziția $j$. Aceasta 
poziție poate sa fie mai mare decât $5$ dar conform formulei construite, nu se va face nicio 
distincție între a plasa un cuvânt pe poziția $6$ spre exemplu, sau poziția $7$, considerându-se
că respectivul element nu va aparține în agregarea finala de lungime $5$.

Este limpede ca se poate aplica acum Algoritmul 3 prezentat în secțiunea precedentă pentru a afla
mulțimea $S = \{\sigma_1, \sigma_2, ..., \sigma_t\}$ de agregări posibile. Propunem combinarea 
soluțiilor din $S$ printr-un sistem de vot. Fiecare cuvânt va primi un scor bazat pe pozițiile pe 
care se află în clasamentele din $S$.

\begin{gather*}
  scor \colon U \to \mathbb{N} \\
  scor(x_i) = \smashoperator{\sum_{j=1}^{|S|}} \sigma_j(x_i), \forall x_i \in U
\end{gather*}

Putem în sfârșit să construim lista finală, introdusă vag la începutul capitolului, de cuvinte 
ordonate în funcție de probabilitatea de a fi etimonul latinesc al unor anumite cuvinte moderne 
din \textit{ro, it, fr, es, pt}. Pe prima poziție se va afla cuvântul produs cu scorul cel mai mic,
adică cel ce se află pe primele poziții în clasamentele $S$, pe a doua poziție, cel cu următorul cel
mai mic scor, și asa mai departe. În cazul în care există mai multe cuvinte cu același scor, le 
vom păstra pe toate pe aceeași poziție și le vom filtra manual folosind reguli lingvistice.

\begin{gather*}
  F = (x_1 > x_2 > x_3 > x_4 > x_5) \\
  x_1 = \min (scor(x)) \\
  x_2 = \min_{x, x \neq x_1} (scor(x)) \\
  x_3 = \min_{x, x \neq x_1, x_2} (scor(x)) \\
  x_4 = \min_{x, x \neq x_1, x_2, x_3} (scor(x)) \\
  x_5 = \min_{x, x \neq x_1, x_2, x_3, x_4} (scor(x)) \\
  \{x_1, x_2, x_3, x_4, x_5\} \subseteq U
\end{gather*}

