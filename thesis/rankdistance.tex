\chapter{Agregarea rezultatelor folosind distanța rank}
Am văzut cum putem obține producții de cuvinte combinând câte o singură limbă romanică cu 
limba latină. Pentru a îmbunătății rezultatele vrem să folosim informația din mai multe 
limbi romanice moderne. Astfel, fiecare clasificator întoarce o listă ordonată de cuvinte latinești, 
pe prima poziție aflându-se perechea cognate cu cea mai mare probabilitate. Prin agregarea 
acestora cu o anumită metrică vom obține cele mai probabile perechi cognate.

\section{Clasamente și distanța rank}
Un \textit{clasament} este o listă ordonată de obiecte după un anumit criteriu, pe prima poziție 
aflându-se cel cu cea mai mare importanță. În unele situații se pune problema găsirii unui clasament
cât mai apropiat de o mulțime de mai multe clasamente. Pentru a rezolva această problemă trebuie sa
definim mai întâi ce înseamnă distanța dintre două clasemente sau dintre un singur clasament și o
mulțime de clasamente.

Există mai multe distanțe folosite cu succes în diverse aplicații: distanța \textit{Kedall tau}, 
\textit{Spearman footrule}, \textit{Levenshtein}, dar noi vom folosi distanța \textit{rank}
\cite{rankdistance}, pe care o vom defini folosindu-ne de următoarele noțiuni. Fie
$U = \{1, 2, ..., n\}$ 
o mulțime finită de obiecte numită univers. Un clasament peste universul $U$
este lista ordonată 
$\tau = (x_1 > x_2 > ... > x_d)$, unde 
$\{x_1, x_2, ..., x_d\} \subseteq U$.
$>$ este o relație de ordine strictă reprezentând criteriul de ordonare. Notăm cu $\tau(x)$, poziția
elementului $x \in U$ în clasamentul $\tau$ dacă $x \in \tau$, numerotând pozițiile de la 1 
începând cu cel mai important obiect din clasament. Dacă un clasament conține toate elementele din
univers, atunci el se va numi \textit{clasament total}. Asemănător, dacă conține doar o submulțime
de obiecte din univers, atunci îl vom numi \textit{clasament parțial}.

Notăm ordinea elementului $x$ în $\tau$ astfel:
\[
  ord(\tau, x)= \begin{dcases}
    |n + 1 - \tau(x)|    &, x \in \tau \\
    0                    &, x \in U \setminus \tau
  \end{dcases}
\]

\begin{definition}
  Fie $\tau$ si $\sigma$ două clasamente parțiale peste același univers $U$. Atunci distanța rank va
  fi $\Updelta(\tau, \sigma) = \sum_{x \in \tau \cup \sigma} |ord(\tau, x) - ord(\sigma, x)|$.
\end{definition}

Se observă faptul că, în calculul distanței rank, se ia în considerare ordinea definită mai sus
și nu poziția. În primul rând, cum primele poziții sunt cele mai importante, distanța dintre două
clasamente trebuie sa fie cu atât mai mare cu cât diferă mai mult începutul lor\cite{linguisticstructuresmarcus}.
În al doilea rând, definiția funcției $ord$ pune accentul pe lungimea clasamentelor întrucăt putem
presupune că un clasament mai lung a fost obținut în urma comparării mai multor obiecte din univers.
Deci ordinea elementelor este mai solidă. Spre exemplu, dacă două clasamente de lungimi diferite au
același element pe prima poziție, există totuși o diferență a ordinii obiectului în cele două liste,
diferență ce contribuie la calculul distanței rank totale.\cite{rankaggregationproblem}

\subsection{Agregări cu distanța rank}
O \textit{agregare de clasamente} reprezintă un singur clasament $\sigma$ astfel încât o anumită 
metrică de la acesta la mulțimea de liste de agregat $T$ este minimă. Raportându-ne la distanța rank,
putem defini formal o agregare astfel:

\begin{definition}
$\sigma$ este o agregare a unui set $T = \{\tau_1, \tau_2, ..., \tau_m\} (m = |T|)$ de clasamente 
dacă și numai dacă $max_{1 \leqslant i \leqslant m}(\Updelta(\tau_i, \sigma))$ este minim posibil,
unde $\Updelta(\alpha, \beta)$ reprezintă distanța rank dintre clasamentele $\alpha$ și $\beta$.
\end{definition}

\section{Determinarea agregărilor optime}
\subsection{Reducerea la o problemă de cuplaj perfect de cost minim}
\subsection{Calcularea tuturor agregărilor optime}
