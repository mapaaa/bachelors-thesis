\chapter{Experimente si rezultate}
In capitolul \ref{chap:two} am prezentat pasii metodei propuse pentru reconstructia cuvintelor
latinesti. Pe scurt, vom antrena mai multe sisteme CRF pe un set de date pentru a face mai apoi 
predictii de cuvinte. Din date vor fi extrase anumite atribute pe baza unor alinieri, conform 
sectiunii \ref{subs:one}. Predictiile sunt apoi agregate pe baza distantei rank ca in capitolul
\ref{chap:three}.

In experimentul rulat ne propunem sa reconstruim cuvinte romanesti neatestate pornind de la un posibil
etimon latinesc.  

\section{Antrenare}
Pentru antrenarea sistemelor CRF am folosit datasetul propus de Ciobanu si Dinu \cite{dataset}.
Acesta contine 3218 grupuri complete de cuvinte cognate pentru cinci limbi romanice (romana, italiana, 
franceza, spaniola, portugheza) si etimoane lor latinesti. Impartim acest set de date in trei parti,
pentru antrenare, pentru reglarea parametrilor si pentru testare. Dimensiunea proportilor este $3:1:1$.

Aliniem fiecare pereche de cuvinte cognate din care extragem 3-grame ca si atribute pentru sistemele CRF.
Folosim implementarea data de Mallet toolkit \cite{mallet}. Parametrii sunt gasiti printr-o cautare
de tip \textit{grid search} cu numarul de iteratii $\{1, 5, 10, 25, 50, 100\}$ si pentru dimensiunea
ferestrei $w$ de $\{1, 2, 3, 4, 5\}$.

Pentru cei mai buni parametri (50 de iteratii) obtinem acuratetea top-10:
\begin{itemize}
    \item Spaniola   $61\%$
    \item Franceza   $62\%$
    \item Italiana   $62\%$
    \item Portugheza $51\%$
    \item Latina     $65\%$
\end{itemize}

\section{Testare}
Aplicam modelele invatate pe o lista de 235 de cuvinte cognate ce exista in limbile romanice din 
Occident dar nu si in limba romana. Acestea au fost extrase dintr-o lista propusa de Ripeanu in \cite{ripeanubook}.
Agregam apoi rezultatele folosind metoda din capitolul \ref{chap:three}. Procesam productiile conform 
urmatoarelor doua reguli:

\begin{enumerate}
  \item diftongul \textit{iă} nu exista in limba romana si il inlocuim cu \textit{ie}
  \item consoanele duble dispar (spre exemplu \textit{ll} devinte \textit{l})
\end{enumerate}

\section{Evaluare}

\subsection{Producerea de cuvinte cognate}
Evaluarea a fost facuta manual intrucat natura problemei face dificila gasirea unei metode de evaluare
automata. Algoritmul prezentat produce cuvinte romanesti neatestate deci doar un expert in domeniu
le-ar putea valida. Am decis sa folosim doar italiana si latina pentru a reconstrui cuvinte romanesti
deoarece spaniola, franceza, portugheza sunt mai departe de romana din punct de vedere ortografic,
iar metoda prezentata se bazeaza pe analiza schimbarilor grafice.

Rezultatele obtinute prin evaluarea manuala a productiilor de cuvinte romanesti pornind de la limba
latina si italiana:
\begin{center}
  \begin{tabular}{|| l l l ||}
    \hline
    Tipul cognat romanesc & Latina & Italiana \\[0.5ex]
    \hline
    \hline
    Reali                                & 82 (34.8\%)  & 72 (30.6\%) \\
    \hline
    Reali cu interventie lingvistica     & 12 (5.1\%)   & 11 (4.6\%) \\
    \hline
    Virtuali                             & 69 (29.3\%)  & 32 (13.6\%) \\
    \hline
    Virtuali cu interventie lingvistica  & 28 (11.9\%)  & 11 (5.1\%) \\
    \hline
    Inexistenti                          & 51 (21.7\%)  & 111 (47.2\%) \\
    \hline
  \end{tabular}
\end{center}

\begin{itemize}
  \item cognat \textbf{real}: cuvant ce exista in limba romana datorita unor procese de dezvoltare
    interne ale limbii sau prin imprumutarea masiva a acestora din alte limbi
  \item cognat \textbf{real cu interventie lingvistica}: cuvant ce exista in limba romana in urma
    aparitiei unor noi criterii lingvistice
  \item cognat \textbf{virtual}: cuvant ce ar fi putut exista in limba romana dar care nu au fost
    atestate
  \item cognat \textbf{virtual cu interventie lingvistica}: cuvant ce ar fi putut exista in limba
    romana in urma introducerii unor noi criterii lingvistice dar care nu au fost atestate
  \item cognat \textbf{inexsitent}: cele ce nu apartin niciunei categorii de mai sus
\end{itemize}

\subsection{Reconstruirea cuvintelor latinesti neatestate}
In acest caz, setul de date contine cuvantul latinesc propus dar neatestatat. Este vorba de un subset
al setului de date propus de Ripeanu in \cite{ripeanubook}. Deci am putut rula o evaluare automata
a rezultatelor. In urma agregarii cu distanta rank am obtinut o acuratete de $23\%$ in top-10. 
Adica am verificat ca etimonul latinesc sa se afle in primele zece productii.

Dar, daca nu am folosi agregarea ci doar limba italiana, se obtine o acuratete top-10 de $38\%$.
Astfel, metoda confirma mai mult de un sfert din cuvintele neatestate latinesti, reconstruite artificial.
