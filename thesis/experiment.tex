\chapter{Experimente și rezultate}
În capitolul \ref{chap:two} am prezentat pașii metodei propuse pentru reconstrucția cuvintelor
latinești. Pe scurt, vom antrena mai multe sisteme CRF pe un set de date pentru a face mai apoi 
predicții de cuvinte. Din date vor fi extrase anumite atribute pe baza unor alinieri, conform 
secțiunii \ref{subs:one}. Predicțiile sunt apoi agregate pe baza distantei rank ca în capitolul
\ref{chap:three}.

În experimentul rulat ne propunem să reconstruim cuvinte românești neatestate pornind de la un posibil
etimon latinesc.  

\section{Antrenare}
Pentru antrenarea sistemelor CRF am folosit setul de date propus de Ciobanu și Dinu \cite{dataset}.
Acesta conține 3218 grupuri complete de cuvinte cognate pentru cinci limbi romanice (română, italiană, 
franceză, spaniolă, portugheză) și etimoanele lor latinești. Împărțim acest set de date în trei părți,
pentru antrenare, pentru reglarea parametrilor și pentru testare. Dimensiunea proporțiilor este $3:1:1$.

Aliniem fiecare pereche de cuvinte cognate din care extragem 3-grame ca și atribute pentru sistemele CRF.
Folosim implementarea dată de Mallet toolkit \cite{mallet}. Parametrii sunt găsiți printr-o căutare
de tip \textit{grid search} cu numărul de iterații $\{1, 5, 10, 25, 50, 100\}$ și pentru dimensiunea
ferestrei $w$ de $\{1, 2, 3, 4, 5\}$.

Pentru cei mai buni parametri (50 de iterații) obținem acuratețea top-10:
\begin{itemize}
    \item Spaniolă   $61\%$
    \item Franceză   $62\%$
    \item Italiană   $62\%$
    \item Portugheză $51\%$
    \item Latină     $65\%$
\end{itemize}

\section{Testare}
Aplicam modelele învățate pe o lista de 235 de cuvinte cognate ce exista în limbile romanice din 
Occident dar nu și în limba română. Acestea au fost extrase dintr-o listă propusă de Rîpeanu în \cite{ripeanubook}.
Agregăm apoi rezultatele folosind metoda din capitolul \ref{chap:three}. Procesăm producțiile conform 
următoarelor două reguli:

\begin{enumerate}
  \item diftongul \textit{iă} nu există în limba română și îl înlocuim cu \textit{ie}
  \item consoanele duble dispar (spre exemplu \textit{ll} devine \textit{l})
\end{enumerate}

\section{Evaluare}

\subsection{Producerea de cuvinte cognate}
Evaluarea a fost făcută manual întrucât natura problemei face dificilă găsirea unei metode de evaluare
automată. Algoritmul prezentat produce cuvinte romanești neatestate deci doar un expert în domeniu
le-ar putea valida. Am decis să folosim doar italiana și latina pentru a reconstrui cuvinte românești
deoarece spaniola, franceza, portugheza sunt mai departe de română din punct de vedere ortografic,
iar metoda prezentată se bazează pe analiza schimbărilor grafice.

Rezultatele obținute prin evaluarea manuală a producțiilor de cuvinte românești pornind de la limba
latină și italiană:
\begin{center}
  \begin{tabular}{|| l l l ||}
    \hline
    Tipul cognat romanesc & Latină & Italiană \\[0.5ex]
    \hline
    \hline
    Reali                                & 82 (34.8\%)  & 72 (30.6\%) \\
    \hline
    Reali cu intervenție lingvistică     & 12 (5.1\%)   & 11 (4.6\%) \\
    \hline
    Virtuali                             & 69 (29.3\%)  & 32 (13.6\%) \\
    \hline
    Virtuali cu intervenție lingvistică  & 28 (11.9\%)  & 11 (5.1\%) \\
    \hline
    Inexistenți                          & 51 (21.7\%)  & 111 (47.2\%) \\
    \hline
  \end{tabular}
\end{center}

\begin{itemize}
  \item cognat \textbf{real}: cuvânt ce există în limba română datorită unor procese de dezvoltare
    interne ale limbii sau prin împrumutarea masivă a acestora din alte limbi
  \item cognat \textbf{real cu intervenție lingvistică}: cuvânt ce există în limba română în urma
    apariției unor noi criterii lingvistice
  \item cognat \textbf{virtual}: cuvânt ce ar fi putut exista în limba română dar care nu a fost
    atestat
  \item cognat \textbf{virtual cu intervenție lingvistică}: cuvânt ce ar fi putut exista în limba
    română în urma introducerii unor noi criterii lingvistice dar care nu a fost atestat
  \item cognat \textbf{inexistent}: cele ce nu aparțin niciunei categorii de mai sus
\end{itemize}

\subsection{Reconstruirea cuvintelor latinești neatestate}
În acest caz, setul de date conține cuvântul latinesc propus dar neatestat. Este vorba de un subset
al setului de date propus de Rîpeanu în \cite{ripeanubook}. Deci am putut rula o evaluare automată
a rezultatelor. În urma agregării cu distanța rank am obținut o acuratețe de $23\%$ în top-10. 
Adică am verificat ca etimonul latinesc să se afle în primele zece producții.

Dar, dacă nu folosim agregarea ci doar limba italiană, se obține o acuratețe top-10 de $38\%$.
Astfel, metoda confirmă mai mult de un sfert din cuvintele neatestate latinești, reconstruite artificial.
